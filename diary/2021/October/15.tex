\mytitle{What is all this for?}

It's probably a good idea for me to step back every so often and
try to sort out how I got into these topics.
What do I hope to achieve here?

At the end of the day, the only thing I'm ultimately trying to do with any of my work is:
I want to understand, in its entirety, the geometry of number rings.

That might not be the answer anyone would expect,
given the theorems I've proved.
But it's been at the back of everything I've thought seriously about.
All the formalism, all the abstract nonsense, all the complicated structures --
all of it, ultimately, is in the service of understanding the integers
in a manner that is simultaneously complete and simple.

Consistently, my impression has been that the structures that appear will be of the most general sort.
We can't understand something as fundamental as numbers without the hardest version of every structure and every theorem.

Schemes were never enough.
Even though an affine scheme contains the data of its ring of functions,
it was clear from the beginning that there was more to see in the integers than $\Spec\ZZ$ could see.
This is why number theory had elements that weren't
explained adequately by the theory of schemes.

Nevertheless $\Spec$ gave us a good platform to work from.
In my way of thinking, the Zariski topological space of $\Spec \ZZ$ simply stratifies our picture.
We have a infinitely many closed pieces, but our eyes are bad --
we can only keep finitely many in focus at any one time.
By refocusing our eyes, we can always see a bigger finite list of primes, but they always recede away from us into a profinite mist.

So what I'm aiming for is a real understanding of these closed strata
and the way they assemble \ldots



