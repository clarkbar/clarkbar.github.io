%!TeX program = xelatex
%-*- program: xelatex -*-
%-*- encoding: utf-8 -*-
% chktex-file 3
% chktex-file 8
% chktex-file 12
% chktex-file 24
% chktex-file 42
\documentclass[draft=false, leqno]{article}
\usepackage[minionmath]{boilerart}
\usepackage{booktabs}
\usepackage{isodate}
\usetikzlibrary{backgrounds,decorations.markings,hobby,fit}

\hypersetup{
    urlcolor = slate,
}

\isodate
\isodash{.}

\setcounter{secnumdepth}{2}

\definecolor{harvestgold}{rgb}{0.85, 0.57, 0.0}
\definecolor{pakistangreen}{rgb}{0.0, 0.4, 0.0}
\definecolor{rosewood}{rgb}{0.4, 0.0, 0.04}
\definecolor{sangria}{rgb}{0.57, 0.0, 0.04}
\definecolor{taupe}{rgb}{0.28, 0.24, 0.2}
\definecolor{tealgreen}{rgb}{0.0, 0.51, 0.5}
\definecolor{ultramarine}{rgb}{0.07, 0.04, 0.56}

\hyphenation{mon-oid-al}
\hyphenation{Grun-dlehren}
\hyphenation{Birkhäus-er} % Comment out when using PDFLaTeX

% Does sectioning and addition of references to table of contents correctly.
\defbibheading{references}[\refname]{%
	\section*{#1}%
	\addcontentsline{toc}{part}{References} %
	\markboth{#1}{#1}
	}

\addbibresource{universal.bib} % To add the bibliography resource

\title{Annotated bibliography}

\author{Clark Barwick}

\date{Draft: \today}

\begin{document}

\maketitle

\section{Foreword}%
\label{sec:Foreword}
This is a resource list for my Ph.D students.
As such, it's biased toward stuff I like:
large-scale structures, big concepts, that kind of thing.
I include some comments about how I think these works may be read.

\section{Foundations}%
\label{sec:Foundations}

\subsection{Simplicial structures}%
\label{sub:Simplicial structures}
Simplicial structures are a key technical tool in homotopical mathematics.
In the 1950s, Dan Kan noticed that essentially the whole of homotopy theory can be expressed in combinatorial terms.
Once this became available, it was a simple matter to transport homotopical thinking to lots of new contexts.

There is no one perfect reference for this material.

If you know nothing of simplicial sets, but are comfortable with the basic technology of (ordinary) category theory, then the starting place is this short note:
\begin{quotation}
  Riehl, Emily. \enquote{A leisurely introduction to simplicial sets}. \href{https://math.jhu.edu/~eriehl/ssets.pdf}.
\end{quotation}
Riehl is a celebrated expositor, and this is as good an invitation as any to simplicial constructions.

After that, the next thing to read is probably sections 1.1--1.3 of
\begin{quotation}
  Lurie, Jacob. \emph{Kerodon}. \href{https://kerodon.net}.
\end{quotation}
This is an online text that is still under construction.
It will eventually cover a big chunk of higher category theory, but
at this point, the main thing is to get comfortable with these objects.

A more complete reference is
\begin{quotation}
  Goerss, Paul G.; Jardine, John F. \emph{Simplicial homotopy theory}. Progress in Mathematics. Vol. 174. Birkhäuser.
\end{quotation}
This book has some rough patches, but it's certainly the best book on the subject I know of.

An old but still wonderful text on simplicial stuff is
\begin{quotation}
  Curtis, Edward B. \enquote{Simplicial Homotopy Theory}. Advances in Mathematics 6, 107--209 (1971).
\end{quotation}
The notation is oldfashioned, but the emphasis on the connection to unstable homotopy theory is fantastic.
A serious homotopy theory student should feel at home with the first six sections of this paper.
Reading this will  

\bookmark[page=1,level=1]{Contents}
\setcounter{tocdepth}{2}
\tableofcontents

%=========================================================%
%=========================================================%
%=========================================================%
%=========================================================%

\newpage

% Separates bibliography into two parts: the specially labeled references followed by the numbered references. The specially labeled references appear with just the label; numbered references appear as "n."
\DeclareFieldFormat{labelnumberwidth}{#1}
\printbibliography[keyword=alph, heading=references]
% \addcontentsline{toc}{part}{References} % Adds References to table of contents
\DeclareFieldFormat{labelnumberwidth}{{#1\adddot\midsentence}}
\printbibliography[heading=none, notkeyword=alph]

\end{document}
