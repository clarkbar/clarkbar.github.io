%!TEX root = book.tex 
% chktex-file 1
% chktex-file 3
% chktex-file 8
% chktex-file 12
% chktex-file 18
% chktex-file 24
% chktex-file 35 
% chktex-file 42

\section{Indization}%
\label{sec:indization}
\addcontentsline{toc}{section}{Indization}

\begin{definition}
	Let $ \alpha \geq 1 $ be an ordinal number, and
	let $ \kappa < \daleth_{\alpha} $ be a regular cardinal.
	Let $ C $ be a category of echelon $ \leq \alpha $.

	Then $ \Ind_{\kappa}^{\alpha}(C) $ is
	the smallest full subcategory $ D \subseteq \PP^{\daleth_{\alpha}}(C) $
	such that $ D $ contains the image
	of the Yoneda embedding
	$ \yo \colon C \inclusion \PP^{\daleth_{\alpha}}(C) $,
	and $ D $ is stable under
	$ \daleth_{\alpha} $-small, $ \kappa $-filtered colimits.

	Accordingly, if $ C $ is a locally small large-category,
	then $ \Ind_{\kappa}^V(C) $ is
	the smallest full subcategory $ D \subseteq \PP^V(C) $
	such that $ D $ contains the image
	of the Yoneda embedding
	$ \yo \colon C \inclusion \PP^V(C) $,
	and $ D $ is stable under $ \kappa $-filtered colimits.
\end{definition}

\begin{eg}
	The category $ \Ind_{0}^{\alpha}(C) $ is 
	equivalent to the presheaf category
	$ \PP^{\daleth_{\alpha}}(C) $.
\end{eg}

\begin{proposition}
	Let $ \alpha \geq 1 $ be an ordinal number, and
	let $ \kappa < \daleth_{\alpha} $ be a regular cardinal.
	Let $ C $ be a category of echelon $ \leq \alpha $.
	Then the Yoneda embedding
	$ \yo \colon C \inclusion \Ind_{\kappa}^{\alpha}(C) $,
	generates $ \Ind_{\kappa}^{\alpha}(C) $ freely under
	$ \daleth_{\alpha} $-small, $ \kappa $-filtered colimits.

	Similarly, if $ C $ is a locally small large-category,
	then the Yoneda embedding
	$ \yo \colon C \inclusion \Ind_{\kappa}^{V}(C) $,
	generates $ \Ind_{\kappa}^{V}(C) $ freely under
	$ \kappa $-filtered colimits.
\end{proposition}

\begin{nul}
	Let $ \alpha \geq 1 $ be an ordinal number, and
	let $ \kappa < \daleth_{\alpha} $ be a regular cardinal.
	Let $ C $ and $ D $  be categories
	that contain all $ \daleth_{\alpha} $-small, $ \kappa $-filtered colimits.
	Denote by $ \Fun_{\kappa}^{\alpha}(C, D) $
	the full subcategory of $ \Fun(C,D) $ consisting of
	the functors $ C \to D $ that preserve all 
	$ \daleth_{\alpha} $-small, $ \kappa $-filtered colimits.
	
	If $ C' $ is a category of echelon $ \leq \alpha $,
	then restriction along the Yoneda embedding
	induces an equivalence of categories
	\[
		\Fun_{\kappa}^{\alpha}(\Ind_{\kappa}^{\alpha}(C'), D)
		\equivalence \Fun(C', D) \period
	\]
\end{nul}

\begin{eg}
	Let $ \alpha \geq 1 $ be an ordinal number, and
	let $ \kappa < \daleth_{\alpha} $ be a regular cardinal.
	Let $ C $ be a $ \daleth_{\alpha} $-small category.
	Then the category $ \Ind_{\kappa}^{\alpha}(C) $ is
	$ \kappa $-accessible of echelon $ \leq \alpha $.
	
	Hence we obtain a functor
	\[
		\Ind_{\kappa}^{\alpha} \colon
		\Cat_{\alpha} \to \Acc_{\kappa}^{\alpha} \period
	\]
	This functor exhibits $ \Acc_{\kappa}^{\alpha} $ as
	a localization of $ \Cat_{\alpha} $,
	and it restricts to the inverse
	\[ \Cat_{\alpha}^{\idem} \equivalence \Acc_{\kappa}^{\alpha} \]
	of the equivalence $ C \mapsto C_{\alpha}^{(\kappa)} $
	constructed in \eqref{nul:thetafromAcctoCat}.
	It also restricts further to the equivalence
	\[
		\Cat_{\alpha}^{\kappa,\idem} \equivalence
		\Pr^{\alpha,L}_{\kappa}
	\]
	inverse to the restriction of
	$ C \mapsto C_{\alpha}^{(\kappa)} $.
\end{eg}

\begin{construction}
	Let $ \beta > \alpha \geq 1 $ be two ordinal numbers.
	Then we can use indization to define 
	a change-of-universe functor
	\[
		\II_{\alpha}^{\beta} \coloneq
		\Ind_{\daleth_{\alpha}}^{\beta} \period
	\]
	If $ \kappa < \daleth_{\alpha} $ is a regular cardinal,
	then this is a fully faithful functor
	\[
		\II_{\alpha}^{\beta} \colon
		\Acc_{\kappa}^{\alpha} \inclusion
		\Acc_{\kappa}^{\beta} \comma
	\]
	which is equivalent to the inclusion
	$ \Cat_{\alpha}^{\idem} \inclusion \Cat_{\beta}^{\idem} $.
	The functor $ \II_{\alpha}^{\beta} $ restricts to 
	a fully faithful functor
	\[
		\II_{\alpha}^{\beta} \colon
		\Pr^{\alpha,L}_{\kappa} \inclusion
		\Pr^{\beta,L}_{\kappa} \comma
	\]
	which is equivalent to the inclusion
	$ \Cat_{\alpha}^{\kappa,\idem} \inclusion \Cat_{\beta}^{\kappa,\idem} $.
\end{construction}

\begin{eg}
	For any category $ C $ of echelon $ \alpha $,
	one has
	\[
		\II_{\alpha}^{\beta}(\PP^{\daleth_{\alpha}}(C))
		\simeq \PP^{\daleth_{\beta}}(C) \period
	\]
\end{eg}

\begin{notation}
	Let $ \alpha \geq 1 $ be an ordinal number.
	Then we will abberviate
	\[
		\Ind^{\alpha} \coloneq \Ind_{\aleph_0}^{\alpha} \period
	\]
\end{notation}


