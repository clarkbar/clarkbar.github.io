%!TEX root = book.tex 
% chktex-file 1
% chktex-file 3
% chktex-file 8
% chktex-file 12
% chktex-file 18
% chktex-file 24
% chktex-file 35 
% chktex-file 42

\subsection{Simplicial \& cosimplicial}%
\label{sub:simplicialcosimplicial}

\begin{definition}
  Let $C$ be a category.
  A \emph{simplicial object} of $C$ is a functor $\DDelta^{\op}\to C$.
  A \emph{cosimplicial object} of $C$ is a functor $\DDelta \to C$.
  We write
  \[
    sC \coloneq \Fun(\DDelta^{\op},C)
    \andeq
    cC \coloneq \Fun(\DDelta, C) \period
  \]

  If $X \in sC$, then we write $X_n$ for the value $X([n])$, and
  if $Y \in cC$, then we write $Y^n$ for the value $Y([n])$.
  At times it may be convenient to write $X_\bullet$ and $Y^\bullet$ instead of $X$ and $Y$, just to emphasize the variance.
\end{definition}

Thus a simplicial object $X_\bullet$ specifies object $X_0, X_1, \dots$, along with \emph{face maps} $d_j \colon X_n \to X_{n-1}$ for each $j \in [n]$ and \emph{degeneracy maps} $s_i \colon X_n \to X_{n+1}$ for each $i \in [n]$:
\[
  \begin{tikzcd}[sep=1.5em, ampersand replacement=\&]
    \cdots \arrow[r] \arrow[r, shift left=1.5ex] \arrow[r, shift right=1.5ex] \arrow[r, shift left=3ex] \arrow[r, shift right=3ex] \&
    {X_3} \arrow[r, shift left=0.75ex] \arrow[r, shift right=0.75ex] \arrow[r, shift right=2.25ex] \arrow[r, shift left=2.25ex] \arrow[l, shift left=0.75ex] \arrow[l, shift right=0.75ex] \arrow[l, shift left=2.25ex] \arrow[l, shift right=2.25ex] \&
    {X_2} \arrow[l] \arrow[l, shift left=1.5ex] \arrow[l, shift right=1.5ex] \arrow[r] \arrow[r, shift left=1.5ex] \arrow[r, shift right=1.5ex] \&
    {X_1} \arrow[l, shift left=0.75ex] \arrow[l, shift right=0.75ex] \arrow[r, shift left=0.75ex] \arrow[r, shift right=0.75ex] \&
    {X_0} \arrow[l]
  \end{tikzcd}%
  \period
\]

We will be interested in simplicial objects of a number of different categories, but our big interest is the theory of simplicial \emph{sets}.

\begin{eg}
  If $[m], [n] \in \DDelta$, then let us write
  \[
    \Delta_m^n \coloneq \Mor_{\DDelta}([m],[n]) \period
  \]
  Thus for each $[n] \in \DDelta$, we have the simplicial set it represents
  \[
    \Delta^n \coloneq \Mor_{\DDelta}(-,[n]) \colon \DDelta^{\op} \to \Set \period
  \]
  We call $\Delta^n$ the \emph{standard $n$-simplex}.
  The assignment $[n] \mapsto \Delta^n$ is a functor $\DDelta \to s\Set$.
  Equally, for each $[m] \in \DDelta$, we have the cosimplicial set it corepresents
  \[
    \Delta_m \coloneq \Mor_{\DDelta}([m],-) \colon \DDelta \to \Set \comma
  \]
  and the assignment $[m] \mapsto \Delta_m$ is a functor $\DDelta^{\op} \to c\Set$.
\end{eg}

The standard simplices play a critical role in the theory of simplicial sets.
The Yoneda lemma implies that for every simplicial set $X$, we have a natural isomorphism
\[
  X_n = \Mor_{s\Set}(\Delta^n, X) \period
\]
An element $\sigma \in X_n$ -- or equivalently a map $\sigma \colon \Delta^n \to X$ -- is called an \emph{$n$-simplex of $X$}.
If $n=0$, we may call this a \emph{vertex};
if $n=1$, we may call this an \emph{edge}.

Every simplicial set $X \in s\Set$ is the colimit of its simplices.
More precisely, consider the Yoneda embedding $\DDelta \to s\Set$ given by $[n] \mapsto \Delta^n$, and let
\[
  \DDelta_{/X} \coloneq \DDelta \times_{s\Set} s\Set_{/X} \period
\]
We call $\DDelta_{/X}$ the \emph{category of simplices} of $X$.
We now have a canonical isomorphism
\[
  \colim_{[n]\in\DDelta_{/X}} \Delta^n = X \period
\]
Said differently, the left Kan extension of the Yoneda embedding $\DDelta \to s\Set$ is the identity functor on $s\Set$. 
We will discuss this much more carefully in \ref{sub:generatorsrelations}.

\begin{eg}
  Let $A$ be a small category.
  The \emph{nerve} $N_\bullet A$ is the simplicial set that carries $[n]$ to the \emph{set} of functors $[n] \to A$.

  In other words, by regarding each nonempty totally ordered finite set $[n]$ as a category, we obtain a fully faithful inclusion $\DDelta \inclusion \Cat$.
  The nerve $N_\bullet A$ is the composite of $\DDelta^{\op} \inclusion \Cat^{\op}$ with the functor $\Cat^{\op} \to \Set$ represented by $A$.

  Thus $N_0A$ is the set of objects of $A$, and $N_1A$ is the set of morphims.
  If $f \colon x \to y$ is a morphism of $A$, then $d_0(f)=y$, and $d_1(f)=x$.
  For any object $x$ of $A$, the degenerate $1$-simplex $s_0(x)$ is the identity at $x$.

  The set $N_2A$ of $2$-simplices is the set of commutative triangles 
  \[
    \begin{tikzcd}[sep=1.5em, ampersand replacement=\&]
      \& y \arrow[dr, "g"] \& \\
      x \arrow[ur, "f"] \arrow[rr, "h"'] \& \& z 
    \end{tikzcd}
  \]
  in $A$.
  If we call this $2$-simplex $\alpha$, then $d_0(\alpha) = g$, $d_1(\alpha) = h$, and $d_2(\alpha) = f$.
  For every morphism $f \colon x \to y$ of $A$, we also have two degenerate $2$-simplices $s_0(f)$ and $s_1(f)$, which correspond to the commutative triangles
  \[
    \begin{tikzcd}[sep=1.5em, ampersand replacement=\&]
      \& x \arrow[dr, "f"] \& \\
      x \arrow[ur, "\id"] \arrow[rr, "f"'] \& \& y 
    \end{tikzcd}
    \andeq
    \begin{tikzcd}[sep=1.5em, ampersand replacement=\&]
      \& y \arrow[dr, "\id"] \& \\
      x \arrow[ur, "f"] \arrow[rr, "f"'] \& \& y 
    \end{tikzcd}
    \comma
  \]
  respectively.

  If $S$ is the set of objects of $A$ and $R$ is its set of morphisms, then
  we find that $N_n A$ is the $n$-fold fiber product $R^{\times_S n}$.
  The simplicial objects we extracted from relations and groupoids are therefore examples of nerves.

  In particular, the standard $n$-simplex $\Delta^n$ is precisely the nerve $N_\bullet [n]$.
\end{eg}

\begin{proposition}
  The nerve functor $N_\bullet \colon \Cat \to s\Set$ is fully faithful.
\end{proposition}

\begin{proof}
  Let $C$ and $D$ be categories, and let $f \colon N_\bullet C \to N_\bullet D$ be a morphism of simplicial sets.
  We aim to show that there exists a unique functor $F \colon C \to D$ such that $f = N_\bullet F$.

  If $F \colon C \to D$ is a functor such that $N_\bullet F=f$, then
  on objects, $F$ is the map $f_0 \colon N_0C \to N_0D$, and
  on morphisms, $F$ is the map $f_1 \colon N_1C \to N_1D$.
  So our $F$ is certainly unique if it exists.

  So define $F$ accordingly: on objects, take the map $f_0$, and on morphisms, take the map $f_1$.
  The compatibility of $f$ with the face map $d_1 \colon N_2 \to N_1$ shows that $F$ respects composition.
  The compatibility of $f$ with the degeneracy map $s_0 \colon N_0 \to N_1$ shows that $F$ respects identities.
  Hence $F$ is indeed a functor.
\end{proof}

\noindent This proposition guarantees that we lose no information when we pass from categories to simplicial sets.


