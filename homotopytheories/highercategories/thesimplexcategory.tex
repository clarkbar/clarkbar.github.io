%!TEX root = book.tex 
% chktex-file 1
% chktex-file 3
% chktex-file 8
% chktex-file 12
% chktex-file 18
% chktex-file 24
% chktex-file 35 
% chktex-file 42

\subsection{The simplex category}%
\label{sub:thesimplexcategory}

\begin{definition}
  The \emph{simplex category} $\DDelta$ is the category
  whose objects are nonempty, totally ordered, finite sets,
  and whose morphisms the monotonic maps between these.
\end{definition}

Every object of $\DDelta$ is uniquely isomorphic to a finite ordinal
\[
  [n] = \{ 0 < 1 < \cdots < n \}
\]
for some integer $n \geq 0$.
This entitles us to refer to objects of $\DDelta$ as if they are all of this form.

Between these, we have the following morphisms:
\begin{itemize}
  \item for every $j \in [n]$, the \emph{face map} is the injective map $\delta_j \colon [n-1] \to [n]$ whose image does not contain $j$;
  \item for every $i \in [n]$, the \emph{degeneracy map} is the surjective map $\sigma_i \colon [n+1] \to [n]$ that carries $i+1$ to $i$.
\end{itemize}
Every other map in the simplex category can be expressed as a composite of face and degeneracy maps.
It is elementary (but boring) to prove that $\DDelta$ is generated by these face and degeneracy maps, subject only to the following relations, called the \emph{simplicial identities}:
\begin{itemize}
  \item if $i \leq j$, then
  \[
    \delta_i \delta_j = \delta_{j+1} \delta_i \semicolon
  \]
  \item if $i \leq j$, then
  \[
    \sigma_j \sigma_i = \sigma_i \sigma_{j+1} \semicolon
  \]
\item for every $i,j$, 
  \[
    \sigma_j \delta_i = \begin{cases}
      \delta_i \sigma_{j-1} & \text{ if } i < j \semicolon \\
      \id                   & \text{ if } i \in \{j,j+1\} \semicolon \\
      \delta_{i-1} \sigma_j & \text{ if } j+1 < i \period
    \end{cases}
  \]
\end{itemize}
In practice, the generators-and-relations description of $\DDelta$ is usually more trouble than it's worth, but
it does provide a schematic picture of the category $\DDelta$:
\[
  \begin{tikzcd}[sep=1.5em, ampersand replacement=\&]
    {[0]} \arrow[r, shift left=0.75ex] \arrow[r, shift right=0.75ex] \&
    {[1]} \arrow[l] \arrow[r] \arrow[r, shift left=1.5ex] \arrow[r, shift right=1.5ex] \&
    {[2]} \arrow[l, shift left=0.75ex] \arrow[l, shift right=0.75ex] \arrow[r, shift left=0.75ex] \arrow[r, shift right=0.75ex] \arrow[r, shift right=2.25ex] \arrow[r, shift left=2.25ex] \&
    {[3]} \arrow[l] \arrow[l, shift left=1.5ex] \arrow[l, shift right=1.5ex] \arrow[r] \arrow[r, shift left=1.5ex] \arrow[r, shift right=1.5ex] \arrow[r, shift left=3ex] \arrow[r, shift right=3ex] \&
    \cdots \comma \arrow[l, shift left=0.75ex] \arrow[l, shift right=0.75ex] \arrow[l, shift left=2.25ex] \arrow[l, shift right=2.25ex]
  \end{tikzcd}%
\]
as well as its opposite $\DDelta^{\op}$:
\[
  \begin{tikzcd}[sep=1.5em, ampersand replacement=\&]
    \cdots \arrow[r] \arrow[r, shift left=1.5ex] \arrow[r, shift right=1.5ex] \arrow[r, shift left=3ex] \arrow[r, shift right=3ex] \&
    {[3]} \arrow[r, shift left=0.75ex] \arrow[r, shift right=0.75ex] \arrow[r, shift right=2.25ex] \arrow[r, shift left=2.25ex] \arrow[l, shift left=0.75ex] \arrow[l, shift right=0.75ex] \arrow[l, shift left=2.25ex] \arrow[l, shift right=2.25ex] \&
    {[2]} \arrow[l] \arrow[l, shift left=1.5ex] \arrow[l, shift right=1.5ex] \arrow[r] \arrow[r, shift left=1.5ex] \arrow[r, shift right=1.5ex] \&
    {[1]} \arrow[l, shift left=0.75ex] \arrow[l, shift right=0.75ex] \arrow[r, shift left=0.75ex] \arrow[r, shift right=0.75ex] \&
    {[0]} \arrow[l]
  \end{tikzcd}%
\]

\begin{notation}
  For any integer $n \geq 0$, we define $\DDelta_{\leq n} \subset \DDelta$ as the full subcategory spanned by the objects $[k]$ with $k \leq n$.
\end{notation}

We started by contemplating an equivalence relation $R$ on $S$ in a diagrammatic way.
That gave us diagram \eqref{eq:equivrel}, which we now can describe efficiently as a functor $X \colon \DDelta_{\leq 2}^{\op} \to \Set$ that carries $[0]$ to $S$, $[1]$ to $R$, and $[2]$ to $R \times_S R$.

Our job is to replace properties with structure, so
we then considered what happens if you allow the possibility of different ways of being equivalent.
The equivalence relation becomes a groupoid;
$S$ becomes the set of objects;
$R$ becomes the set of morphisms.
The transitivity condition becomes a composition structure.
That new structure was encoded in a map $R \times_S R \to R$,
and we asked it to satisfy a \emph{coherence condition}, which asserted the associativity of composition.
That associativity is then expressed in the larger diagram \eqref{eq:groupoid},
which is a functor $\DDelta_{\leq 3}^{\op} \to \Set$ that carries $[3]$ to $R \times_S R \times_S R$.

But let's examine the meaning of associativity in our groupoid a little more carefully.
The value of associative laws is that they permit us to make sense of composites not only of pairs of morphisms
\[
  x \to y \to z \comma
\]
but also of triples of morphisms
\[
  w \to x \to y \to z \comma
\]
and even of arbitrary finite sequences of morphisms
\[
  x_0 \to x_1 \to \cdots \to x_n \period
\]
That's a subtle shift of perspective here that we want to take a moment to appreciate.
We were never \emph{really} interested in the equation $(\gamma\beta)\alpha = \gamma(\beta\alpha)$;
what we wanted was to say that there was an unambiguous meaning to the expression $\gamma\beta\alpha$ and even of $\alpha_n \alpha_{n-1} \cdots \alpha_1$.

In other words, the data in which we are ultimately interested is not that of a single multiplication law $R \times_S R \to R$, but in fact of a family of multiplication laws
\[
  R^{\times_S n} \coloneq R \times_S R \times_S \cdots \times_S R \to R \comma
\]
one for each $n$.
Associativity is then the expression of the compatibility between these.
This is all packaged up very neatly in the functor $\DDelta^{\op} \to \Set$ that carries $[n]$ to $R^{\times_S n}$. 
This is our first example of a \emph{simplicial object}.


