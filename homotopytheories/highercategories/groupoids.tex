%!TEX root = book.tex 
% chktex-file 1
% chktex-file 3
% chktex-file 8
% chktex-file 12
% chktex-file 18
% chktex-file 24
% chktex-file 35 
% chktex-file 42

%-------------------------------------------------------------------%
%-------------------------------------------------------------------%
%-------------------------------------------------------------------%
\chapter{Grothendieck's homotopy hypothesis}%
\label{cha:Grothendieck's homotopy hypothesis}
%-------------------------------------------------------------------%
%-------------------------------------------------------------------%
%-------------------------------------------------------------------%

Our approach to constructing the homotopy theory of animae follows a general recipe, which will inform our work throughout this book.
The goal is to design a homotopy theory $\EE$, using only partial or imperfect information about $\EE$.
\begin{itemize}
  \item%
    Select a piece $\EE_0$ of $\EE$ that is simple enough that you can understand it completely,
    but complex enough so that any object $X \in \EE$ is completely determined by the sets/groups/whatevers of maps $T \to X$ with $T \in \EE_0$.
    In many cases, you'll want $\EE_0$ to generate $\EE$ under suitable colimits.
  \item%
    Now select a small category $J$, usually combinatorial in nature,
    along with an essentially surjective functor $R \colon J \to \EE_0$ (which need not be fully faithful).
    In many cases, it will be helpful if the objects of $J$ come equipped with some notion of \emph{degree}.
    The category $J$ and functor $R$ should be chosen so that some key salient features of general objects $X \in \EE$ can be read off from mapping objects $\Map(R(j),X)$.
  \item%
    Encode these salient features as a set of properties $P$ of the corresponding functors $\Map(R(-),X)$.
    For example, perhaps you'll want to require that for some particular $j \in J$, the object $\Map(R(j),X)$ is trivial.
    Or perhaps such a property expresses $\Map(R(k),X)$ as a limit or colimit of various other mapping objects $\Map(R(j),X)$.
    The objects of your homotopy theory will be exactly the functors $J^{\op} \to \Set$ (or other enriching category) that enjoy these properties of $P$.
\end{itemize}

In this chapter, we will construct the homotopy theory of animae in this way:
$\EE_0$ will be the homotopy theory of contractible animae (which is of course trivial),
$J$ is the category of \emph{simplices} $\mbfDelta$ (the first topic of this book),
and $P$ is Kan's \emph{horn-filling condition}.

Other options for $J$ (and therefore $P$) are possible:
there is an interesting class of categories -- the \emph{Grothendieck test categories} -- that can could serve as $J$.
These include the category of nonempty finite sets, Joyal's categories $\mbfTheta_n$, various categories of cubes, and the tree category.

One special feature of $\mbfDelta$ is that it can also be used to define the larger homotopy theory of $\infty$-categories itself.
For this homotopy theory, $\EE_0$ is the homotopy theory of finite, linearly ordered sets (which is not trivial, but is still quite simple),
$J$ is again the category $\mbfDelta$,
and $P$ is the \emph{inner horn-filling condition} first identified explicitly by Boardman--Vogt.

%!TEX root = book.tex 
% chktex-file 1
% chktex-file 3
% chktex-file 8
% chktex-file 12
% chktex-file 18
% chktex-file 24
% chktex-file 35 
% chktex-file 42

\section{Simplicial objects}%
\label{sec:Simplicialobjects}

Our aim is to convert equality from a property into a structure.
At a minimum, we want to be able to work with notions of equivalence that are not just set-theoretic equality.
Traditionally, such an alternative is described as an \emph{equivalence relation} on a set $S$.

A \emph{relation} is encoded by a subset $R \subseteq S \times S$, which is the subset of related pairs of elements.
An \emph{equivalence} relation is one that is reflexive, symmetric, and transitive:
\begin{enumerate}
  \item Reflexivity is the condition that $R$ contains the image of the \emph{diagonal map} $S \to S \times S$ given by $x \mapsto (x,x)$.
  \item Symmetry is the condition that $R$ is stable under the involution of $S \times S$ given by $(x,y) \mapsto (y,x)$.
  \item Transitivity states that the projection $S \times S \times S \to S \times S$ given by $(x,y,z) \mapsto (x,z)$ carries the subset $R \times_S R \subseteq S \times S \times S$ to the subset $R$.
\end{enumerate}
An equivalence relation on $S$ can thus be converted into a diagram
\begin{equation}
  \label{eq:equivrel}
  \begin{tikzcd}[sep=1.5em, ampersand replacement=\&]
    R \times_S R \arrow[r] \arrow[r, shift left=1.5ex] \arrow[r, shift right=1.5ex] \& R \arrow[l, shift left=0.75ex] \arrow[l, shift right=0.75ex] \arrow[r, shift left=0.75ex] \arrow[r, shift right=0.75ex] \& S \comma \arrow[l]
  \end{tikzcd}%
\end{equation}
The maps from left to right are various projections.
The maps from right to left are various diagonal maps.
The colimit of this diagram agrees with the coequalizer of the subdiagram 
\[
  \begin{tikzcd}[sep=1.5em, ampersand replacement=\&]
    R \arrow[r, shift left=0.5ex] \arrow[r, shift right=0.5ex] \& S \comma
  \end{tikzcd}%
\]
which is in turn the set of equivalence classes $S/R$.

The case we want to make is that this diagram shape is the start of a more natural \emph{simplicial diagram}, extending infinitely off to the left:
\[
  \simplicial{R_2}{R_1}{R_0}.
\]
This diagram can encode the higher forms of equivalence we sought in the introduction to this chapter.

Our diagram \eqref{eq:equivrel} does not capture the symmetry property of $R$.
It would be a simple matter to incorporate actions of the symmetric groups $\Sigma_2$ and $\Sigma_3$ on $R$ and $R \times_S R$ into our diagram.
This would lead us to the theory of \emph{symmetric simplicial sets}.
We will see in our story that the symmetry is encoded not in the shape of a simplicial diagram of sets, but in the properties we demand of it when it models an anima.
This asymmetry is as much a feature as it is a bug, however:
it will be necessary when we want to connect the theory of simplicial sets to that of categories.

Nothing in the diagram \eqref{eq:equivrel} forces $R$ to be a subset of $S \times S$;
we may demand only that $R$ map to $S \times S$.
The effect of this is to permit the elements of $S$ to be equivalent in many ways.
For example, if $x,y \in S$, then the fiber of the map $R \to S \times S$ is a set
\[
  \enquine{x R y} \comma
\]
which is the set of ways in which $x$ and $y$ are equivalent -- or witnesses to their equivalence.
The middle map from $R \times_S R \to R$ now carries two such witnesses, $\beta \in \enquine{xRy}$ and $\gamma \in \enquine{yRz}$, to a composite witness $\gamma\beta \in \enquine{xRz}$.
But now we encounter a kind of \emph{higher transitivity}: if $\alpha \in \enquine{wRx}$, then there ought to be only one composite witness $\gamma\beta\alpha \in \enquine{wRz}$.
In other words, this composition should be \emph{associative}:
$(\gamma\beta)\alpha = \gamma(\beta\alpha)$.
If it is, then what we have is the data of a \emph{groupoid} whose objects are the elements of $S$ and whose (iso)morphisms are elements of $R$, so that $\Isom(x,y) = \enquine{xRy}$.

The associativity also lets us extend our diagram \eqref{eq:equivrel} to a larger diagram
\begin{equation}%
  \label{eq:groupoid}
  \begin{tikzcd}[sep=1.5em, ampersand replacement=\&]
    R \times_S R \times_S R\arrow[r, shift left=0.75ex] \arrow[r, shift right=0.75ex] \arrow[r, shift right=2.25ex] \arrow[r, shift left=2.25ex] \& R \times_S R \arrow[l] \arrow[l, shift left=1.5ex] \arrow[l, shift right=1.5ex] \arrow[r] \arrow[r, shift left=1.5ex] \arrow[r, shift right=1.5ex] \& R \arrow[l, shift left=0.75ex] \arrow[l, shift right=0.75ex] \arrow[r, shift left=0.75ex] \arrow[r, shift right=0.75ex] \& S \period \arrow[l]
  \end{tikzcd}%
\end{equation}
These diagrams are now becoming sufficiently complicated that
we need to be more pedantic about the indexing categories we're using.

\subsection{The simplex category}%
\label{sub:Thesimplexcategory}

\begin{definition}
  The \emph{simplex category} $\DDelta$ is the category
  whose objects are nonempty, totally ordered, finite sets,
  and whose morphisms the monotonic maps between these.
\end{definition}

Every object of $\DDelta$ is uniquely isomorphic to a finite ordinal
\[
  [n] = \{ 0 < 1 < \cdots < n \}
\]
for some integer $n \geq 0$.
This entitles us to refer to objects of $\DDelta$ as if they are all of this form.

Between these, we have the following morphisms:
\begin{itemize}
  \item for every $j \in [n]$, the \emph{face map} is the injective map $\delta_j \colon [n-1] \to [n]$ whose image does not contain $j$;
  \item for every $i \in [n]$, the \emph{degeneracy map} is the surjective map $\sigma_i \colon [n+1] \to [n]$ that carries $i+1$ to $i$.
\end{itemize}
Every other map in the simplex category can be expressed as a composite of face and degeneracy maps.
It is elementary (but boring) to prove that $\DDelta$ is generated by these face and degeneracy maps, subject only to the following relations, called the \emph{simplicial identities}:
\begin{itemize}
  \item if $i \leq j$, then
  \[
    \delta_i \delta_j = \delta_{j+1} \delta_i \semicolon
  \]
  \item if $i \leq j$, then
  \[
    \sigma_j \sigma_i = \sigma_i \sigma_{j+1} \semicolon
  \]
\item for every $i,j$, 
  \[
    \sigma_j \delta_i = \begin{cases}
      \delta_i \sigma_{j-1} & \text{ if } i < j \semicolon \\
      \id                   & \text{ if } i \in \{j,j+1\} \semicolon \\
      \delta_{i-1} \sigma_j & \text{ if } j+1 < i \period
    \end{cases}
  \]
\end{itemize}
In practice, the generators-and-relations description of $\DDelta$ is usually more trouble than it's worth, but
it does provide a schematic picture of the category $\DDelta$:
\[
  \begin{tikzcd}[sep=1.5em, ampersand replacement=\&]
    {[0]} \arrow[r, shift left=0.75ex] \arrow[r, shift right=0.75ex] \&
    {[1]} \arrow[l] \arrow[r] \arrow[r, shift left=1.5ex] \arrow[r, shift right=1.5ex] \&
    {[2]} \arrow[l, shift left=0.75ex] \arrow[l, shift right=0.75ex] \arrow[r, shift left=0.75ex] \arrow[r, shift right=0.75ex] \arrow[r, shift right=2.25ex] \arrow[r, shift left=2.25ex] \&
    {[3]} \arrow[l] \arrow[l, shift left=1.5ex] \arrow[l, shift right=1.5ex] \arrow[r] \arrow[r, shift left=1.5ex] \arrow[r, shift right=1.5ex] \arrow[r, shift left=3ex] \arrow[r, shift right=3ex] \&
    \cdots \comma \arrow[l, shift left=0.75ex] \arrow[l, shift right=0.75ex] \arrow[l, shift left=2.25ex] \arrow[l, shift right=2.25ex]
  \end{tikzcd}%
\]
as well as its opposite $\DDelta^{\op}$:
\[
  \begin{tikzcd}[sep=1.5em, ampersand replacement=\&]
    \cdots \arrow[r] \arrow[r, shift left=1.5ex] \arrow[r, shift right=1.5ex] \arrow[r, shift left=3ex] \arrow[r, shift right=3ex] \&
    {[3]} \arrow[r, shift left=0.75ex] \arrow[r, shift right=0.75ex] \arrow[r, shift right=2.25ex] \arrow[r, shift left=2.25ex] \arrow[l, shift left=0.75ex] \arrow[l, shift right=0.75ex] \arrow[l, shift left=2.25ex] \arrow[l, shift right=2.25ex] \&
    {[2]} \arrow[l] \arrow[l, shift left=1.5ex] \arrow[l, shift right=1.5ex] \arrow[r] \arrow[r, shift left=1.5ex] \arrow[r, shift right=1.5ex] \&
    {[1]} \arrow[l, shift left=0.75ex] \arrow[l, shift right=0.75ex] \arrow[r, shift left=0.75ex] \arrow[r, shift right=0.75ex] \&
    {[0]} \arrow[l]
  \end{tikzcd}%
\]

\begin{notation}
  For any integer $n \geq 0$, we define $\DDelta_{\leq n} \subset \DDelta$ as the full subcategory spanned by the objects $[k]$ with $k \leq n$.
\end{notation}

We started by contemplating an equivalence relation $R$ on $S$ in a diagrammatic way.
That gave us diagram \eqref{eq:equivrel}, which we now can describe efficiently as a functor $X \colon \DDelta_{\leq 2}^{\op} \to \Set$ that carries $[0]$ to $S$, $[1]$ to $R$, and $[2]$ to $R \times_S R$.

Our job is to replace properties with structure, so
we then considered what happens if you allow the possibility of different ways of being equivalent.
The equivalence relation becomes a groupoid;
$S$ becomes the set of objects;
$R$ becomes the set of morphisms.
The transitivity condition becomes a composition structure.
That new structure was encoded in a map $R \times_S R \to R$,
and we asked it to satisfy a \emph{coherence condition}, which asserted the associativity of composition.
That associativity is then expressed in the larger diagram \eqref{eq:groupoid},
which is a functor $\DDelta_{\leq 3}^{\op} \to \Set$ that carries $[3]$ to $R \times_S R \times_S R$.

But let's examine the meaning of associativity in our groupoid a little more carefully.
The value of associative laws is that they permit us to make sense of composites not only of pairs of morphisms
\[
  x \to y \to z \comma
\]
but also of triples of morphisms
\[
  w \to x \to y \to z \comma
\]
and even of arbitrary finite sequences of morphisms
\[
  x_0 \to x_1 \to \cdots \to x_n \period
\]
That's a subtle shift of perspective here that we want to take a moment to appreciate.
We were never \emph{really} interested in the equation $(\gamma\beta)\alpha = \gamma(\beta\alpha)$;
what we wanted was to say that there was an unambiguous meaning to the expression $\gamma\beta\alpha$ and even of $\alpha_n \alpha_{n-1} \cdots \alpha_1$.

In other words, the data in which we are ultimately interested is not that of a single multiplication law $R \times_S R \to R$, but in fact of a family of multiplication laws
\[
  R^{\times_S n} \coloneq R \times_S R \times_S \cdots \times_S R \to R \comma
\]
one for each $n$.
Associativity is then the expression of the compatibility between these.
This is all packaged up very neatly in the functor $\DDelta^{\op} \to \Set$ that carries $[n]$ to $R^{\times_S n}$. 
This is our first example of a \emph{simplicial object}.

\subsection{Simplicial \& cosimplicial}%
\label{sub:simplicialcosimplicial}

\begin{definition}
  Let $C$ be a category.
  A \emph{simplicial object} of $C$ is a functor $\DDelta^{\op}\to C$.
  A \emph{cosimplicial object} of $C$ is a functor $\DDelta \to C$.
  We write
  \[
    sC \coloneq \Fun(\DDelta^{\op},C)
    \andeq
    cC \coloneq \Fun(\DDelta, C) \period
  \]

  If $X \in sC$, then we write $X_n$ for the value $X([n])$, and
  if $Y \in cC$, then we write $Y^n$ for the value $Y([n])$.
  At times it may be convenient to write $X_\bullet$ and $Y^\bullet$ instead of $X$ and $Y$, just to emphasize the variance.
\end{definition}

Thus a simplicial object $X_\bullet$ specifies object $X_0, X_1, \dots$, along with \emph{face maps} $d_j \colon X_n \to X_{n-1}$ for each $j \in [n]$ and \emph{degeneracy maps} $s_i \colon X_n \to X_{n+1}$ for each $i \in [n]$:
\[
  \begin{tikzcd}[sep=1.5em, ampersand replacement=\&]
    \cdots \arrow[r] \arrow[r, shift left=1.5ex] \arrow[r, shift right=1.5ex] \arrow[r, shift left=3ex] \arrow[r, shift right=3ex] \&
    {X_3} \arrow[r, shift left=0.75ex] \arrow[r, shift right=0.75ex] \arrow[r, shift right=2.25ex] \arrow[r, shift left=2.25ex] \arrow[l, shift left=0.75ex] \arrow[l, shift right=0.75ex] \arrow[l, shift left=2.25ex] \arrow[l, shift right=2.25ex] \&
    {X_2} \arrow[l] \arrow[l, shift left=1.5ex] \arrow[l, shift right=1.5ex] \arrow[r] \arrow[r, shift left=1.5ex] \arrow[r, shift right=1.5ex] \&
    {X_1} \arrow[l, shift left=0.75ex] \arrow[l, shift right=0.75ex] \arrow[r, shift left=0.75ex] \arrow[r, shift right=0.75ex] \&
    {X_0} \arrow[l]
  \end{tikzcd}%
  \period
\]

We will be interested in simplicial objects of a number of different categories, but our big interest is the theory of simplicial \emph{sets}.

\begin{eg}
  If $[m], [n] \in \DDelta$, then let us write
  \[
    \Delta_m^n \coloneq \Mor_{\DDelta}([m],[n]) \period
  \]
  Thus for each $[n] \in \DDelta$, we have the simplicial set it represents
  \[
    \Delta^n \coloneq \Mor_{\DDelta}(-,[n]) \colon \DDelta^{\op} \to \Set \period
  \]
  We call $\Delta^n$ the \emph{standard $n$-simplex}.
  The assignment $[n] \mapsto \Delta^n$ is a functor $\DDelta \to s\Set$.
  Equally, for each $[m] \in \DDelta$, we have the cosimplicial set it corepresents
  \[
    \Delta_m \coloneq \Mor_{\DDelta}([m],-) \colon \DDelta \to \Set \comma
  \]
  and the assignment $[m] \mapsto \Delta_m$ is a functor $\DDelta^{\op} \to c\Set$.
\end{eg}

The standard simplices play a critical role in the theory of simplicial sets.
The Yoneda lemma implies that for every simplicial set $X$, we have a natural isomorphism
\[
  X_n = \Mor_{s\Set}(\Delta^n, X) \period
\]
An element $\sigma \in X_n$ -- or equivalently a map $\sigma \colon \Delta^n \to X$ -- is called an \emph{$n$-simplex of $X$}.
If $n=0$, we may call this a \emph{vertex};
if $n=1$, we may call this an \emph{edge}.

Every simplicial set $X \in s\Set$ is the colimit of its simplices.
More precisely, consider the Yoneda embedding $\DDelta \to s\Set$ given by $[n] \mapsto \Delta^n$, and let
\[
  \DDelta_{/X} \coloneq \DDelta \times_{s\Set} s\Set_{/X} \period
\]
We call $\DDelta_{/X}$ the \emph{category of simplices} of $X$.
We now have a canonical isomorphism
\[
  \colim_{[n]\in\DDelta_{/X}} \Delta^n = X \period
\]
Said differently, the left Kan extension of the Yoneda embedding $\DDelta \to s\Set$ is the identity functor on $s\Set$. 
We will discuss this much more carefully in \ref{sub:generatorsrelations}.

\begin{eg}
  Let $A$ be a small category.
  The \emph{nerve} $N_\bullet A$ is the simplicial set that carries $[n]$ to the \emph{set} of functors $[n] \to A$.

  In other words, by regarding each nonempty totally ordered finite set $[n]$ as a category, we obtain a fully faithful inclusion $\DDelta \inclusion \Cat$.
  The nerve $N_\bullet A$ is the composite of $\DDelta^{\op} \inclusion \Cat^{\op}$ with the functor $\Cat^{\op} \to \Set$ represented by $A$.

  Thus $N_0A$ is the set of objects of $A$, and $N_1A$ is the set of morphims.
  If $f \colon x \to y$ is a morphism of $A$, then $d_0(f)=y$, and $d_1(f)=x$.
  For any object $x$ of $A$, the degenerate $1$-simplex $s_0(x)$ is the identity at $x$.

  The set $N_2A$ of $2$-simplices is the set of commutative triangles 
  \[
    \begin{tikzcd}[sep=1.5em, ampersand replacement=\&]
      \& y \arrow[dr, "g"] \& \\
      x \arrow[ur, "f"] \arrow[rr, "h"'] \& \& z 
    \end{tikzcd}
  \]
  in $A$.
  If we call this $2$-simplex $\alpha$, then $d_0(\alpha) = g$, $d_1(\alpha) = h$, and $d_2(\alpha) = f$.
  For every morphism $f \colon x \to y$ of $A$, we also have two degenerate $2$-simplices $s_0(f)$ and $s_1(f)$, which correspond to the commutative triangles
  \[
    \begin{tikzcd}[sep=1.5em, ampersand replacement=\&]
      \& x \arrow[dr, "f"] \& \\
      x \arrow[ur, "\id"] \arrow[rr, "f"'] \& \& y 
    \end{tikzcd}
    \andeq
    \begin{tikzcd}[sep=1.5em, ampersand replacement=\&]
      \& y \arrow[dr, "\id"] \& \\
      x \arrow[ur, "f"] \arrow[rr, "f"'] \& \& y 
    \end{tikzcd}
    \comma
  \]
  respectively.

  If $S$ is the set of objects of $A$ and $R$ is its set of morphisms, then
  we find that $N_n A$ is the $n$-fold fiber product $R^{\times_S n}$.
  The simplicial objects we extracted from relations and groupoids are therefore examples of nerves.

  In particular, the standard $n$-simplex $\Delta^n$ is precisely the nerve $N_\bullet [n]$.
\end{eg}

\begin{proposition}
  The nerve functor $N_\bullet \colon \Cat \to s\Set$ is fully faithful.
\end{proposition}

\begin{proof}
  Let $C$ and $D$ be categories, and let $f \colon N_\bullet C \to N_\bullet D$ be a morphism of simplicial sets.
  We aim to show that there exists a unique functor $F \colon C \to D$ such that $f = N_\bullet F$.

  If $F \colon C \to D$ is a functor such that $N_\bullet F=f$, then
  on objects, $F$ is the map $f_0 \colon N_0C \to N_0D$, and
  on morphisms, $F$ is the map $f_1 \colon N_1C \to N_1D$.
  So our $F$ is certainly unique if it exists.

  So define $F$ accordingly: on objects, take the map $f_0$, and on morphisms, take the map $f_1$.
  The compatibility of $f$ with the face map $d_1 \colon N_2 \to N_1$ shows that $F$ respects composition.
  The compatibility of $f$ with the degeneracy map $s_0 \colon N_0 \to N_1$ shows that $F$ respects identities.
  Hence $F$ is indeed a functor.
\end{proof}

\noindent This proposition guarantees that we lose no information when we pass from categories to simplicial sets.

\subsection{Interlude: generators \& relations of categories}%
\label{sub:generatorsrelations}
We mentioned above (\ref{}) that every simplicial set is the colimit of its simplices.
In other words, every simplicial set is well approximated by simplices.

More generally, let $\CC$ be a category with all small colimits.
In most cases, $\CC$ will be a large category.%
\footnote{
  In fact, a small category $\CC$ with all small coproducts is automatically a \emph{preorder}.
  That is, if $x,y \in \CC$ are two objects, then we claim that there is at most one map $x \to y$.
  To prove this, suppose that $\kappa$ is the cardinality of the set of morphisms of $\CC$, and consider the coproduct $\kappa \cdot x = \coprod_{\alpha < \kappa} x$.
  Now $\CC(\kappa \cdot x, y) = \CC(x,y)^{\kappa}$, so by Cantor, if $\#\CC(x,y) \geq 2$, then $\#\CC(\kappa\cdot x, y) \geq 2^\kappa > \kappa$, which contradicts our assumption.
}
We are interested in situations in which, even though $\CC$ is large, all of its objects are well approximated -- or even \emph{determined} -- by a small subcategory $A \subseteq \CC$ via colimits.
This will be an idea that plays a big role in our work, so let's discuss it carefully.

\begin{definition}
  Let $\CC$ be a category with all colimits, and let $A \subseteq \CC$ be a small full subcategory.
  We say that $A$ \emph{generates $\CC$ under colimits} if the smallest full subcategory $\DD \subseteq \CC$ that is closed under colimits and contains $A$ is $\CC$ itself.

  We say that $A$ \emph{strongly generates $\CC$ under colimits} -- or that $A$ is \emph{dense} in $\CC$ -- if the left Kan extension of the inclusion $A \subseteq \CC$ along itself is the identity functor on $\CC$.
\end{definition}

For every object $X \in \CC$, we have the category
\[
  A_{/X} \coloneq A \times_{\CC} \CC_{/X} \comma
\]
whose objects are pairs $(a,f)$ consisting of an object $a \in A$ and a morphism $f \colon a \to X$ in $\CC$.
The left Kan extension of the inclusion $A \inclusion \CC$ along itself is the functor that carries an object $X \in \CC$ to the colimit $\colim_{a \in A_{/X}} a$.
Thus $A$ is dense in $\CC$ if and only if, for every object $X \in \CC$, the natural morphism
\[
  \colim_{a \in A_{/X}} a \to X
\]
is an isomorphism.
So if $A$ strongly generates $\CC$, then every object of $\CC$ is a colimit of objects in $A$ in a canonical way.
So certainly if $A$ is dense in $\CC$, then it generates $\CC$ under colimits, but the converse is far from true.
In fact, if $A$ generates $\CC$ under colimits, then it need not even be the case that every object of $\CC$ can be written as a colimit of objects from $A$.

\begin{eg}
  Consider the full subcategory $\{\pt\} \subset \Set$ generated by the one-point set.
  The category $\{\pt\}$ is dense in $\Set$.
  To see this, let $S$ be a set, and
  consider the category $\{\pt\}_{/S}$.
  An object of $\{\pt\}_{/S}$ is nothing more than an element of $S$,
  and there are no non-identity morphisms in $\{\pt\}_{/S}$ (because there are none in $\{\pt\}$!).
  Hence $\{\pt\}_{/S}$ is the set $S$, viewed as a category.
  Thus
  \[
    \colim_{s \in \{\pt\}_{/S}} s = \coprod_{s \in S} \{s\} = S \period
  \]
\end{eg}

\begin{eg}
  Let's generalize this example.
  Let $A$ be a small category, and let $\PP(A)$ be the category of \emph{presheaves} on $A$ --
  \emph{i.e.}, functors $A^{\op} \to \Set$.
  Thus
  \[
    \PP(A) = \Fun(A^{\op}, \Set) \comma
  \]
  and in particular $s\Set = \PP(\Delta)$.
  Now via the Yoneda embedding $\yo \colon A \inclusion \PP(A)$, we consider $A$ as a full subcategory of $\PP(A)$.

  The category $A$ is dense in $\PP(A)$.
  To prove this, consider a presheaf $F \colon A^{\op} \to \Set$.
  An object of the category $A_{/F}$ is a pair $(a,f)$ consisting of an object $a \in A$ and a natural transformation $f \colon \yo_a \to F$.
  By Yoneda, such a natural transformation is the same thing as an element $f \in F(a)$.
  Thus $A_{/F}$ is sometimes called the \emph{category of elements of $F$}.
  To prove that the canonical natural transformation
  \[
    \eta \colon \colim_{a \in A_{/F}} \yo_a \to F
  \]
  is an isomorphism, let $b \in A$ be an object, and consider the map
  \[
    \eta_b \colon \colim_{a \in A_{/F}} \Mor_A(b,a) \to F(b) \period
  \]
  The elements of the colimit on the left are equivalence classes of triples $(a,f,\phi)$ consisting of an object $a \in A$, an element $f \in F(a)$, and a morphism $\phi \colon b \to a$ in $A$.
  The equivalence relation is generated by the demand that if $(a,f,\phi)$ and $(a',f',\phi')$ are such triples, and if there exists a map $\gamma \colon a \to a'$ such that $\phi' = \gamma\phi$ and $F(\gamma)(f') = f$, then $(a,f,\phi)$ and $(a',f',\phi')$ are equivalent.
  The map $\eta_b$ carries such a triple to $F(\phi)(f) \in F(b)$.
  The aim is to show that this map is a bijection.

  So let $g \in F(b)$.
  Then $(b,g) \in A_{/F}$ is an object, and $\id_b \in A(b,b)$ provides an element $(b,g,\id_b)$ of the colimit on the left, and $\eta_b(b,g,\id_b) = g$.
  Furthermore, if $(a,f,\phi)$ is an element of the colimit such that $F(\phi)(f) = \eta_b(a,f,\phi) = g$, then
  by definition, $\phi \colon b \to a$ is a morphism such that $\phi = \phi \id_b$ and $F(\phi)(f) = g$;
  thus $(a,f,\phi)$ is equivalent to $(b,g,\id_b)$.
\end{eg}

\begin{eg}
  Let $\Ab$ denote the category of abelian groups, and
  let $F \subset \Ab$ be the full subcategory spanned by the single object $\ZZ$.
  (In other words, $F$ is the category of free abelian groups of rank $1$.)
  What kinds of colimits can we form from $F$?

  Certainly any free abelian group is a colimit of objects of $F$, since we can form the coproducts
  \[
    \ZZ\{S\} \coloneq \bigoplus_{s \in S} \ZZ
  \]
  for every set $S$.
  Now if $E$ is any abelian group, then we claim that we can write $E$ as a colimit of free abelian groups.

  There's really only one way to prove that an object can be written as a colimit of some \enquote{nice} objects:
  we have to somehow cook up a good supply of maps from the nice objects to our object. 
  In our case, we can get away with a single map, as long as it is an epimorphism.
  We select a set $S \subseteq E$ of generators for $E$.
  (If we're feeling lazy or wasteful, we can choose $S = E$.)
  This defines an epimorphism
  \[
    \phi \colon \ZZ\{S\} \to E \period
  \]
  Now the kernel of $\phi$ is a subgroup $K \subseteq \ZZ\{S\}$ such that $E = \ZZ\{S\}/K$.
  It turns out that $K$ itself is automatically free, but
  that isn't obvious, and in any case we don't really need that fact.
  All we need to do is select some generators $T \subseteq K$.
  Now we have a homomorphism $\ZZ\{T\} \to \ZZ\{S\}$ whose image is $K$, so that $E$ is a pushout: 
  \[
    \begin{tikzcd}[sep=1.5em, ampersand replacement=\&]
      \ZZ\{T\} \ar[r] \ar[d] \& \ZZ\{S\} \ar[d] \\
      0        \ar[r]        \& E 
    \end{tikzcd}
  \]
  
  We have thus shown that every abelian group can be written as a colimit of free groups, and that every free group can be written as a colimit of copies of $\ZZ$.
  This implies that our subcategory $F$ generates $\Ab$ under colimits.

  But $F_1$ does not \emph{strongly} generate $\ZZ$.
  To see why not, let's consider the category $F_{/E}$ for an abelian group $E$.
  An object of $F_{/E}$ is a homomorphism $a \colon \ZZ \to E$;
  such a homomorphism is determined uniquely by the value $a(1)$, so we may regard the objects of $F_{/E}$ as \emph{elements} of $E$.
  In this way, if $a, b \in E$ are two such objects, then
  a morphism $m \colon a \to b$ is an integer $m$ such that $mb = a$ in $E$.
\end{eg}

Suppose that $\CC$ is a category with colimits, and
let us suppose that we want to write down a functor $s\Set \to \CC$ that preserves colimits.


\subsection{Skeletal \& coskeletal}%
\label{sub:Skeletalandcoskeletal}

The simplicial objects we have been discussing so far have all been extended from the finite subcategories $\DDelta_{\leq n}$.
Let's understand the mechanism for these extensions.

We write
\[
  s_{\leq n}C \coloneq \Fun(\Delta_{\leq n}^{\op}, C)
  \andeq
  c^{\leq n}C \coloneq \Fun(\Delta_{\leq n}, C)
  \period
\]
Now restriction along the inclusion $\DDelta_{\leq n} \subset \DDelta$ defines functors
\[
  sC \to s_{\leq n}C
  \andeq
  cC \to c^{\leq n}C \comma
\]
which we will denote by $X \mapsto X_{\leq n}$ and $Y \mapsto Y^{\leq n}$.

\begin{definition}
  If $C$ has finite colimits, then these functors each admit a fully faithful left adjoint given by left Kan extension:
  \[
    \sk_n \colon s_{\leq n}C \inclusion sC
    \andeq
    \sk_n \colon c^{\leq n}C \inclusion cC \period
  \]
  These are called the \emph{$n$-skeleton} functors.
  
  Dually, if $C$ has finite limits, then these functors each admit a fully faithful right adjoint given by right Kan extension:
  \[
    \cosk_n \colon s_{\leq n}C \inclusion sC
    \andeq
    \cosk_n \colon c^{\leq n}C \inclusion cC \period
  \]
  These are called the \emph{$n$-coskeleton} functors.
\end{definition}

We have the usual formulas for these Kan extensions:
if $X \in s_{\leq n}C$, then
\[
  \sk_n(X)_m = \colim_{[k] \in ((\DDelta_{\leq n})_{[m]/})^{\op}} X_k 
  \andeq
  \cosk_n(X)_m = \lim_{[k] \in ((\DDelta_{\leq n})_{/[m]})^{\op}} X_k \comma
\]
and if $Y \in c^{\leq n}C$, then
\[
  \sk_n(Y)^m = \colim_{[k] \in (\DDelta_{\leq n})_{/[m]}} Y^k 
  \andeq
  \cosk_n(Y)^m = \lim_{[k] \in (\DDelta_{\leq n})_{[m]/}} Y^k \period
\]
In the language of coends and ends:
\[
  \sk_n(X)_m = \int^{[k] \in \DDelta_{\leq n}} \Delta^k_m \times X_k
  \andeq
  \cosk_n(X)_m = \int_{[k] \in \DDelta_{\leq n}} X_k^{\Delta^m_k} \comma
\]
\[
  \sk_n(Y)^m = \int^{[k] \in \DDelta_{\leq n}} \Delta^m_k \times Y^k
  \andeq
  \cosk_n(Y)^m = \int_{[k] \in \DDelta_{\leq n}} (Y^k)^{\Delta^k_m} \period
\]

We will sometimes abuse notation slightly by writing $\sk_n(X) = \sk_n(X_{\leq n})$ and $\cosk_n(X) = \cosk_n(X_{\leq n})$ for a simplicial object $X$.
That is, we will often regard the functors $\sk_n$ and $\cosk_n$ as implicitly precomposed with the restriction $sC \to s_{\leq n}C$.
The formulas above remain valid.

If $C$ has all finite limits and colimits, and if $X,Y \in sC$, then we have natural bijections
\[
  \Mor_{sC}(\sk_n(X), Y) = \Mor_{s_{\leq n}C}(X_{\leq n}, Y_{\leq n}) = \Mor_{sC}(X, \cosk_n(Y)) \period
\]

\begin{definition}
  Let $C$ be a category with all finite limits and colimits, and let $n \geq 0$ be an integer.
  A simplicial object $X \in sC$ is \emph{$n$-skeletal} if and only if the natural map $\sk_n(X) \to X$ is an isomorphism.
  Accordingly, $X$ is \emph{$n$-coskeletal} if and only if the natural map $X \to \cosk_n(X)$ is an isomorphism.

  Similar definitions apply for cosimplicial objects.
\end{definition}

\begin{eg}
  Let $X \in sC$ be a simplicial object.
  Our $X$ is $0$-skeletal if and only if it is \emph{constant}.
  It is $0$-coskeletal if and only if it is determined by $X_0$ via the formula
  \[
    X_m = X_0^{\times(m+1)} \period
  \]
\end{eg}

\begin{eg}
  The standard $n$-simplex $\Delta^n$ is $n$-skeletal.
  For a simplicial set $X$, the $n$-skeleton $\sk_n(X)$ is the colimit of the $m$-simplices of $X$ with $m \leq n$:
  \[
    \sk_n(X) = \colim_{[m] \in (\DDelta_{\leq n})_{/X}} \Delta^m \period
  \]
  (Exercise \ref{exercise:nskeletonascolimit}.)
\end{eg}

\begin{eg}
  If $X$ is a simplicial set, then we have a formula for the $m$-simplices of the $n$-coskeleton:
  \[
    \cosk_n(X)_m = \Mor_{s\Set}(\sk_n(\Delta^m), X) \period
  \]
\end{eg}

\begin{eg}
  The nerve of any small category is $2$-coskeletal (Exercise \ref{exercise:nerve2cosk}).
\end{eg}

\subsection{Boundaries \& horns}%
\label{sub:Boundariesandhorns}

\begin{definition}
  Let $X$ be a simplicial set.
  A \emph{simplicial subset} $Y \subseteq X$ is a choice of a subset $Y_n \subseteq X_n$ for each $n \geq 0$ such that for any map $\phi \colon [m] \to [n]$ in $\DDelta$, the induced map $X_n \to X_m$ carries $Y_n$ to $Y_m$.
\end{definition}

\noindent We are particularly interested here in simplicial subsets of the simplicial set $\Delta^n$.

\begin{notation}
  Let $0 \leq i_0 < i_1 < \cdots < i_k \leq n$ be integers.
  Then we write
  \[
    \Delta^{\{i_0, \dots, i_k\}} \subseteq \Delta^n
  \]
  for the corresponding simplicial subset of $\Delta^n$.
  This simplicial subset is itself a $k$-simplex.
\end{notation}

Families of simplicial subsets of a simplicial set $X$ can be intersected or unioned, just as with subsets of a set.

\begin{definition}
  Let $n \geq 0$ be an integer.
  For every integer $0 \leq i \leq n$, the \emph{$i$-th face} is the simplicial subset 
  \[
    \Delta^{\hat{\imath}} \coloneq \Delta^{\{0,\dots,i-1,i+1,\dots,n\}} \subset \Delta^n \period
  \]
  This is the unique $(n-1)$-simplex of $\Delta^n$ that does not contain the vertex $\Delta^{\{i\}}$.

  The \emph{boundary} of the $n$-simplex is the union of all the faces of $\Delta^n$:
  \[
    \partial \Delta^n \coloneq \bigcup_{0 \leq i \leq n} \Delta^{\hat{\imath}} \subset \Delta^n \period
  \]

  For every integer $0 \leq k \leq n$, the \emph{$k$-th horn} is the union of all but the $k$-th face of $\Delta^n$:
  \[
    \Lambda^n_k \coloneq \bigcup_{0 \leq i \leq n, i \neq k} \Delta^{\hat{\imath}} \subset \Delta^n \period
  \]
  This is the union of all the faces of $\Delta^n$ that contain the vertex $\Delta^{\{k\}}$. 
\end{definition}

Equivalently, $\partial \Delta^n$ can be described as the $(n-1)$-skeleton of the $n$-simplex:
\[
  \partial \Delta^n = \sk_{n-1}(\Delta^n) \period
\]
Thus
\[
  \Mor_{s\Set}(\partial\Delta^n, X) = \cosk_{n-1}(X)_n \period
\]

\subsection{Geometric realization}%
\label{sub:geometricrealization}
We have seen that every simplicial set is the colimit of its simplices.

\subsection{Higher groupoids \& higher categories}%
\label{sub:Highergroupoidshighercategories}
Let's start with \emph{$(2,1)$-categories}.
These are categories with the property that between two objects $x$ and $y$, we have a \emph{groupoid} of maps $x \to y$.
Accordingly, between two maps $f,g \colon x \to y$ we have a set of isomorphisms $\Isom(f,g)$.

\subsection{Fibrations}%
\label{sub:Fibrations}

\begin{exercises}
  \item%
    \label{exercise:stronggeneration}
    Use the Yoneda lemma to show that for any small category $A$,
    the left Kan extension of the Yoneda embedding $A \inclusion \Fun(A^{\op}, \Set)$ along itself is the identity functor.
  \item%
    \label{exercise:nskeletonascolimit}
    We have seen that every simplicial set $X$ is the colimit of its simplices.
    Now show that $X$ is $n$-skeletal if and only if the canonical map
    \[
      \colim_{[m] \in \DDelta_{\leq n, /X}} \Delta^m \to X
    \]
    is an isomorphism.
  \item%
    \label{exercise:nerve2cosk}
    Show that the composition of the nerve functor $N \colon \Cat \to s\Set$ with the restriction $s\Set \to s_{\leq 2}\Set$ is fully faithful.
    Conclude that the nerve of every small category is $2$-coskeletal.
\end{exercises}




