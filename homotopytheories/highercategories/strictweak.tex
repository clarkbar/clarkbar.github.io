%!TEX root = book.tex 
% chktex-file 1
% chktex-file 3
% chktex-file 8
% chktex-file 12
% chktex-file 18
% chktex-file 24
% chktex-file 35 
% chktex-file 42

%\section*{From strict to weak}%
%\label{sec:strictweak}%
%\addcontentsline{toc}{subsection}{From strict to weak}

\minidiv

\noindent \emph{Why is higher category theory subtle?}
At first it's hard to see what all the fuss is about.
We already know what $0$-categories are: they're \emph{sets}.
We also know what $1$-categories are: they're \emph{categories}.
So a $1$-category has a collection of objects and, between every pair $x,y$ of objects, one has a $0$-category $\Mor(x,y)$ of maps.
Composition is then a map
\[
  \Mor(x,y) \times \Mor(y,z) \to \Mor(x,z)
\]
given by $(f,g) \mapsto g \circ f$.

The way to iterate this definition seems uncontroversial.
Let $\VV$ be a category with all finite products.
A \emph{category $\CC$ enriched in $\VV$} consists of:
\begin{itemize}
  \item a collection $\Obj \CC$ of objects;
  \item between every pair $x,y \in \Obj \CC$, an object
  \[
    \Mor_{\CC}(x,y) \in \Obj \VV \semicolon
  \]
  \item for every $x \in \Obj \CC$, a morphism $\id_x \colon 1_{\VV} \to \Mor_{\CC}(x,x)$ (where $1_{\VV}$ denotes the terminal object of $\VV$); and
  \item for every $x,y,z \in \Obj \CC$, a morphism
    \[
      \Mor_{\CC}(x,y) \times \Mor_{\CC}(y,z) \to \Mor_{\CC}(x,z) \comma
    \]
    written $(f,g) \mapsto g \circ f$.
\end{itemize}
These data are subject to the usual axioms: composition is associative ($(h \circ g) \circ f = h \circ (g \circ f)$), and unital ($f \circ \id_x = \id_y \circ f = f$).
There's an attached notion of \emph{$\VV$-enriched functor}, whose definition is predictable.

Now we can give an easy iterative definition.
Our base case is the notion of \emph{strict $0$-categories} and the category $\Cat^{\textit{str}}_0 \coloneq \Set$.
Then we define a \emph{strict $n$-category} as a category enriched in $\Cat^{\textit{str}}_{n-1}$,
and we define $\Cat^{\textit{str}}_n$ as the category of $\Cat^{\textit{str}}_{n-1}$-enriched categories and $\Cat^{\textit{str}}_{n-1}$-enriched functors.

Unwinding the recursion, we see that a strict $n$-category $\CC$ has:
\begin{itemize}
  \item a collection $\Obj \CC$ of \emph{objects} or \emph{$0$-morphisms};
  \item for every pair $x,y \in \Obj(\CC)$ of objects, a collection $\Mor_{\CC}(x,y)$ of \emph{morphisms} or \emph{$1$-morphisms} with \emph{source} $x$ and \emph{target} $y$;
  \item for every pair $f,g \in \Mor_{\CC}(x,y)$ of $1$-morphisms (which we say are \emph{parallel} because they have the same source, and they have the same target), a collection $\Mor_{\Mor_{\CC}(x,y)}(f,g)$ of \emph{$2$-morphisms} with source $f$ and target $g$;
  \item for every parallel pair $\alpha,\beta \in \Mor_{\Mor_{\CC}(x,y)}(f,g)$ of $2$-morphisms, a collection
    \[
      \Mor_{\Mor_{\Mor_{\CC}(x,y)}(f,g)}(\alpha,\beta)
    \]
    of \emph{$3$-morphisms} with source $\alpha$ and target $\beta$;
\end{itemize}

\dots

\begin{itemize}
  \item for every parallel pair $\phi, \psi$ of $(n-1)$-morphisms, a collection
    \[
    \Mor_{\Mor_{\;{\ddots}_{\;\Mor_{\CC}(x,y)\;}{\revddots}\;}}(\phi,\psi)
    \]
    $\Mor(\phi,\psi)$ of \emph{$n$-morphisms} with source $\phi$ and target $\psi$.
\end{itemize} 
Each of these kinds of morphisms can be composed, in all sorts of ways.
As $n$ increases, the combinatorics can get a little heady, but it's nothing we can't handle.

But as the word \enquote{strict} suggests, this is demanding too much.
To see how, let's look more carefully at the case $n = 2$.
A strict $2$-category consists of a set $\Obj \CC$ of objects;
between every $x,y \in \Obj \CC$, a set $\Mor_{\CC}(x,y)$ of morphisms $x \to y$; and
between every $f,g \in \Mor_{\CC}(x,y)$, a set $\Mor_{\Mor_{\CC}(x,y)}(f,g)$ of $2$-morphisms $f \to g$.
Morphisms and $2$-morphisms each have composition laws, which are associative and unital.

Associativity means that if $f \colon x \to y$, $g \colon y \to z$, and $h \colon z \to u$ are $1$-morphisms of $\CC$, then $h \circ (g \circ f) = (h \circ g) \circ f$.
There's something strange about this.
On either side of the equals sign here are \emph{objects} of the category $\Mor_{\CC}(x,u)$, and here we are asking them to be \emph{equal}.
This is just the kind of unreasonable request that our structuralist disposition is supposed to deny.
Let's illustrate this issue with an interesting class of examples that \enquote{ought} to be $2$-categories, but are not strict $2$-categories.

We take our inspiration from a construction we meet early in a study of category theory:
if $M$ is a monoid, then we define a category $BM$ with exactly one object $\pt$ and $\Mor_{BM}(\pt,\pt) = M$;
composition in $BM$ is multiplication in $M$.
An action of $M$ on an object of a category $\CC$ is then precisely a functor $BM \to \CC$.
The construction $M \mapsto BM$ identifies the category of monoids with the category of categories with exactly one object.

Let's try to tell this story again one \enquote{category level} up.
There are some interesting examples that look like monoid objects in categories.
For example, consider the category $\Mod(R)$ of modules over our ring $R$.
The tensor product $\otimes_R$ is a multiplication law on $\Mod(R)$.
We cannot quite call this a monoid structure though, because it's not strictly associative.
If $A$, $B$, and $C$ are three $R$-modules, then it depends on some rather pedantic set-theoretic points of your precise definition of the tensor product as to whether we really have \emph{equality}
\[
  (A \otimes_R B) \otimes_R C = A \otimes_R (B \otimes_R C) \period
\]
More dramatically, in our preferred, ladder-free understanding of the tensor product, we came to the conclusion that we should simply \emph{define} it as the object representing the functor $T \mapsto \categ{Bilin}_R(A \times B,T)$.
Of course, representing objects are only unique up to canonical isomorphism, not up to set-theoretic equality.
So from this perspective, strict associativity for $\otimes_R$ is no longer \emph{meaningful}.

On the other hand, there clearly is \emph{some} kind of associativity here, because we have an isomorphism
\[
  \alpha_{A,B,C} \colon (A \otimes_R B) \otimes_R C \cong A \otimes_R (B \otimes_R C)
\]
that is natural in $A$, $B$, and $C$,
since these objects each represent the functor $T \mapsto \categ{Trilin}_R(A \times B \times C, T)$.
In other words, associativity is no longer a \emph{property}, but a piece of \emph{structure} --
namely, the natural isomorphism $\alpha$.
The category $\Mod(R)$ is a \emph{monoidal category}.

We might now try to construct a $2$-category $B\Mod(R)$. 
There will be exactly one object, $\pt$.
A $1$-morphism $A \colon \pt \to \pt$ will be a module over a ring $R$.
A $2$-morphism $\phi \colon A \to B$ will be an $R$-linear map.
In other words, $\Mor(\pt,\pt) = \Mod(R)$.
The composition functor
\[
  \Mor(\pt, \pt) \times \Mor(\pt,\pt) \to \Mor(\pt,\pt)
\]
is the formation of the tensor product $(A,B) \mapsto A \otimes_R B$.
But of course this isn't a strict $2$-category, because this composition is not strictly associative.

The difference between $2$-categories and strict $2$-categories reduces, in the one-object case, to the difference between monoidal categories and monoid objects in categories.
In order to come to grips with $2$-categories in general, it's a good start to understand monoidal $1$-categories.
In particular, how does one deal with the failure of associativity?

\emph{Why is associativity important anyhow?}
This is not a frivilous question.
In a monoid $M$, if we have three elements $x,y,z \in M$, then what associativity buys us is the ability to talk about the element $xyz \in M$ without ambiguity.
More generally, for any $n$, and for any collection $\{x_1,x_2,\dots,x_n\}$, associativity lets us make unique sense of the product
\[
  x_1 x_2 \cdots x_n \in M \period
\]
This even makes sense when $n = 0$, because the empty product in $M$ is the unit for the multiplication.

We are used to thinking of the \emph{structure} of a monoid as an element $e \in M$ and a binary multiplication $M \times M \to M$.
These data are then required to satisfy two \emph{properties}: $xe=x=ex$ and $(xy)z=x(yz)$.
From one perspective, this is a strange thing to do:
the structure we really want out of a monoid is the ability to make sense of any product $x_1 x_2 \cdots x_n \in M$.
We get that as a result of the associativity and unitality, but
if we were more direct in our intentions, we would define the structure of a monoid as a set $M$ along with products
\[
  \prod_{i \in I} \colon M^I \to M \comma
\]
one for every totally ordered finite set $I$, all arriving in a single packet.
The binary multiplication map is what we get for $I = \{1,2\}$, and the unit is what we get for $I = \varnothing$.

Of course, we've just introduced a lot more structure here, and
in general we have to pay for new pieces of structure with more properties.
In our case, we require that if $I$ is a singleton, then $\prod_{i \in I}$ is the identity.
More importantly, we demand that these multiplication maps satisfy a higher associativity:
for every monotonic map $\phi \colon J \to I$ between totally ordered finite sets, we have
\[
  \prod_{i \in I} \prod_{j \in \phi^{-1}\{i\}} x_j = \prod_{j \in J} x_j \period
\]
Applied to the two monotonic surjections $\{1,2,3\} \to \{1,2\}$, we recover associativity $(xy)z = xyz = x(yz)$;
applied to the two injections $\{1\} \to \{1,2\}$, we recover the unit condition $xe=x=ex$.

This sort of definition of \enquote{monoid} is inefficient, but
it is equivalent to the usual definition.
(That is, they define equivalent categories.)
Tom Leinster calls this definition \emph{unbiased}, because
the usual definition exhibits a bias for small totally ordered finite sets -- those of cardinality less than $3$.

Let us \emph{categorify} this story in order to make sense of monoidal categories.
A monoidal category consists of a category $\CC$ and a family of functors
\[
  \otimes_I = \bigotimes_{i \in I} \colon \CC^I \to \CC \comma
\]
one for each totally ordered finite set $I$.
This is not, however, all the structure we need.
We also need to define, for every monotonic map $\phi \colon J \to I$, a natural isomorphism
\[
  \alpha_{\phi} \colon \bigotimes_{i \in I} \bigotimes_{j \in \phi^{-1}\{i\}} x_j \equivalence \bigotimes_{j \in J} x_j \period
\]
This is a great deal of structure, and
it must be made to satisfy a great many conditions, called \emph{coherences}.
These identify $\alpha_{\id}$ with $\id$ and express the relationship among $\alpha_{\psi \circ \phi}$,$\alpha_{\phi}$, and $\alpha_{\psi}$.
A \emph{monoidal functor} between two monoidal categories $C$ and $D$ is then a functor $F \colon C \to D$ along with isomorphisms $\gamma_I \colon \otimes_I \circ F^I \equivalence F \circ \otimes_I$ that are suitably compatible with the associators for $C$ and $D$.

Once we already \emph{have} the categorified notion -- the notion of \enquote{monoidal category} and \enquote{monoidal functor}, our unbiased definition of monoid becomes natural and trivial to state cleanly.
We observe that \emph{concatenation} is a monoidal structure on the category $\OO$ of totally ordered finite sets.
A monoid is then exactly a monoidal functor $\OO \to \Set$, where the monoidal structure on $\Set$ is the cartesian product.
This phenomenon is a little expression of what John Baez called the \emph{macrocosm/microcosm principle}.

In order to establish a workable framework for \emph{weak} higher categories, we can follow a similar strategy to the one taken here to contemplate monoidal categories.
\begin{enumerate}
  \item We identify a large family of composition shapes we can form in a strict $n$-category.
  \item We define a category of these composition shapes.
    The morphisms of this category should control the higher associativity laws of these compositions.
  \item We then weaken these associative laws from \emph{equations} to pieces of \emph{structure} (like isomorphisms) that identify the two sides of the equations.
  \item These isomorphisms will have to satisfy coherences that are controlled by compositions in our category of composition shapes.
    These coherences might themselves be properties that would be more naturally expressed as higher isomorphisms,
    which would in turn be subject to yet higher coherences (controlled by composable sequences of maps), 
    \emph{etc}.
\end{enumerate}
The transition from equations to pieces of structure to exhibit \emph{sameness} is the heart of our approach.
This is what we call the \emph{homotopical turn} in higher category theory.
As a result of this transition, the theory of higher categories will unfold in a more robust and appealing way than the more direct approach -- explicitly weakening the theory of $n$-categories -- can achieve.
