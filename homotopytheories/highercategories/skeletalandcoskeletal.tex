%!TEX root = book.tex 
% chktex-file 1
% chktex-file 3
% chktex-file 8
% chktex-file 12
% chktex-file 18
% chktex-file 24
% chktex-file 35 
% chktex-file 42

\subsection{Skeletal \& coskeletal}%
\label{sub:skeletalandcoskeletal}

The simplicial objects we have been discussing so far have all been extended from the finite subcategories $\DDelta_{\leq n}$.
Let's understand the mechanism for these extensions.

We write
\[
  s_{\leq n}C \coloneq \Fun(\Delta_{\leq n}^{\op}, C)
  \andeq
  c^{\leq n}C \coloneq \Fun(\Delta_{\leq n}, C)
  \period
\]
Now restriction along the inclusion $\DDelta_{\leq n} \subset \DDelta$ defines functors
\[
  sC \to s_{\leq n}C
  \andeq
  cC \to c^{\leq n}C \comma
\]
which we will denote by $X \mapsto X_{\leq n}$ and $Y \mapsto Y^{\leq n}$.

\begin{definition}
  If $C$ has finite colimits, then these functors each admit a fully faithful left adjoint given by left Kan extension:
  \[
    \sk_n \colon s_{\leq n}C \inclusion sC
    \andeq
    \sk_n \colon c^{\leq n}C \inclusion cC \period
  \]
  These are called the \emph{$n$-skeleton} functors.
  
  Dually, if $C$ has finite limits, then these functors each admit a fully faithful right adjoint given by right Kan extension:
  \[
    \cosk_n \colon s_{\leq n}C \inclusion sC
    \andeq
    \cosk_n \colon c^{\leq n}C \inclusion cC \period
  \]
  These are called the \emph{$n$-coskeleton} functors.
\end{definition}

We have the usual formulas for these Kan extensions:
if $X \in s_{\leq n}C$, then
\[
  \sk_n(X)_m = \colim_{[k] \in ((\DDelta_{\leq n})_{[m]/})^{\op}} X_k 
  \andeq
  \cosk_n(X)_m = \lim_{[k] \in ((\DDelta_{\leq n})_{/[m]})^{\op}} X_k \comma
\]
and if $Y \in c^{\leq n}C$, then
\[
  \sk_n(Y)^m = \colim_{[k] \in (\DDelta_{\leq n})_{/[m]}} Y^k 
  \andeq
  \cosk_n(Y)^m = \lim_{[k] \in (\DDelta_{\leq n})_{[m]/}} Y^k \period
\]
In the language of coends and ends:
\[
  \sk_n(X)_m = \int^{[k] \in \DDelta_{\leq n}} \Delta^k_m \times X_k
  \andeq
  \cosk_n(X)_m = \int_{[k] \in \DDelta_{\leq n}} X_k^{\Delta^m_k} \comma
\]
\[
  \sk_n(Y)^m = \int^{[k] \in \DDelta_{\leq n}} \Delta^m_k \times Y^k
  \andeq
  \cosk_n(Y)^m = \int_{[k] \in \DDelta_{\leq n}} (Y^k)^{\Delta^k_m} \period
\]

We will sometimes abuse notation slightly by writing $\sk_n(X) = \sk_n(X_{\leq n})$ and $\cosk_n(X) = \cosk_n(X_{\leq n})$ for a simplicial object $X$.
That is, we will often regard the functors $\sk_n$ and $\cosk_n$ as implicitly precomposed with the restriction $sC \to s_{\leq n}C$.
The formulas above remain valid.

If $C$ has all finite limits and colimits, and if $X,Y \in sC$, then we have natural bijections
\[
  \Mor_{sC}(\sk_n(X), Y) = \Mor_{s_{\leq n}C}(X_{\leq n}, Y_{\leq n}) = \Mor_{sC}(X, \cosk_n(Y)) \period
\]

\begin{definition}
  Let $C$ be a category with all finite limits and colimits, and let $n \geq 0$ be an integer.
  A simplicial object $X \in sC$ is \emph{$n$-skeletal} if and only if the natural map $\sk_n(X) \to X$ is an isomorphism.
  Accordingly, $X$ is \emph{$n$-coskeletal} if and only if the natural map $X \to \cosk_n(X)$ is an isomorphism.

  Similar definitions apply for cosimplicial objects.
\end{definition}

\begin{eg}
  Let $X \in sC$ be a simplicial object.
  Our $X$ is $0$-skeletal if and only if it is \emph{constant}.
  It is $0$-coskeletal if and only if it is determined by $X_0$ via the formula
  \[
    X_m = X_0^{\times(m+1)} \period
  \]
\end{eg}

\begin{eg}
  The standard $n$-simplex $\Delta^n$ is $n$-skeletal.
  For a simplicial set $X$, the $n$-skeleton $\sk_n(X)$ is the colimit of the $m$-simplices of $X$ with $m \leq n$:
  \[
    \sk_n(X) = \colim_{[m] \in (\DDelta_{\leq n})_{/X}} \Delta^m \period
  \]
  (Exercise \ref{exercise:nskeletonascolimit}.)
\end{eg}

\begin{eg}
  If $X$ is a simplicial set, then we have a formula for the $m$-simplices of the $n$-coskeleton:
  \[
    \cosk_n(X)_m = \Mor_{s\Set}(\sk_n(\Delta^m), X) \period
  \]
\end{eg}

\begin{eg}
  The nerve of any small category is $2$-coskeletal (Exercise \ref{exercise:nerve2cosk}).
\end{eg}


