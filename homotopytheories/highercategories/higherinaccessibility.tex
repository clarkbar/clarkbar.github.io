%!TEX root = book.tex 
% chktex-file 1
% chktex-file 3
% chktex-file 8
% chktex-file 12
% chktex-file 18
% chktex-file 24
% chktex-file 35 
% chktex-file 42

\section{Higher inaccessibility}%
\label{sec:higherinaccessibility}
\addcontentsline{toc}{section}{Higher inaccessibility}

\begin{nul}
	We shall endow an ordinal with its order topology.
	This may be described recursively as follows:
	\begin{enumerate}
		\item The ordinal $ 0 $ is the empty topological space.
		\item For any ordinal $ \alpha $ with its order topology,
			the order topology on the ordinal $ \alpha + 1 $
			is the one-point compactification of $ \alpha $.
		\item For any limit ordinal $ \alpha $,
			the order topology is the colimit topology
			$ \colim_{\beta < \alpha} \beta $.
	\end{enumerate}
\end{nul}

We will use terminology 
that treats $ \Ord $ itself as a topological space,
even though it is not small.

\begin{definition}
	If $ W \subseteq \Ord $ is a subclass,
	then a \defn{limit point} of $ A $ is
	an ordinal $ \alpha $ such that $ \alpha = \sup (W \cap \alpha) $.
	The class $ W $ will be said to be \defn{closed} if and only if
	it contains all its limit points.

	An \defn{ordinal function} is a class map $ f \colon \Ord \to \Ord $.
	We say that $ f $ is \defn{continuous} if and only if
	its restriction to any subset is continuous.
	Equivalently, $ f $ is continuous if and only if,
	for every subclass $ W \subseteq \Ord $
	and every limit point $ \alpha $ of $ W $,
	the ordinal $ f(\alpha) $ is a limit point of $ f(W) $.

	We say that $ f $ is \defn{normal} if and only if
	it is continuous and strictly increasing.
\end{definition}

\begin{nul}
	If $ f $ is a normal ordinal function,
	then its image is a closed and unbounded class%
	\footnote{This is often abbreviated \defn{club class} in
	set theory literature.}
	of ordinals.
	Conversely, if $ W \subseteq \Ord $ is a closed and unbounded class,
	then we can define a normal ordinal function $ f $ by
	\[
		f(\alpha) =
		\min \left\{ \gamma \in W :
			(\forall \beta < \alpha)(f(\beta) < \gamma) \right\} \period
	\]
\end{nul}

\begin{definition}
	Let $ f $ be an ordinal function.
	A regular cardinal $ \kappa $ is said to be
	\defn{$ f $-inaccessible} if and only if,
	for every ordinal $ \alpha$, if $ \alpha < \kappa $,
	then $ f(\alpha) < \kappa $ as well.
\end{definition}

\begin{eg}
	If $ f $ is the ordinal function that carries
	an ordinal $ \alpha $ to the cardinal $ 2^{|\alpha|} $,
	then an $ f $-inaccessible cardinal is precisely
	an inaccessible cardinal.
\end{eg}

\begin{construction}
	Let $ f $ be an increasing ordinal function
	such that for every ordinal $ \beta $, one has $ \beta < f(\beta) $.
	For every ordinal $ \xi $,
	the normal ordinal function $ \alpha \mapsto f^{\alpha}(\xi) $
	in uniquely specified by the requirements that
	$ f^0(\xi) = \xi $ and $ f^{\alpha + 1}(\xi) =f(f^{\alpha}(\xi)) $.

	\cite{Jorgensen1970} proves that an $ f $-inaccessible cardinal
	greater than an ordinal $ \xi $
	is precisely a regular cardinal that is a \emph{fixed point}
	for the ordinal function $ \alpha \mapsto f^{\alpha}(\xi) $.
\end{construction}

\begin{eg}
	If $ f $ is the ordinal function $ \beta \mapsto 2^{|\beta|} $,
	then $ f^{\alpha}(\omega) = \beth_{\alpha} $.
	An inaccessible cardinal is thus precisely a regular $ \beth $-fixed point.
	
	If $ f $ is the ordinal function $ \beta \mapsto |\beta|^+ $,
	then $ f^{\alpha}(\omega) = \aleph_{\alpha} $. 
	A weakly inaccessible cardinal is precisely a regular $ \aleph $-fixed  point.
\end{eg}

\begin{eg}
	Assume $ \au $.
	Consider the ordinal function $ f $ that carries
	an ordinal $ \beta $ to the smallest inaccessible cardinal greater than $ \beta $.
	For any ordinal $ \alpha $,
	we have $ \daleth_{\alpha} = f^{\alpha}(\omega) $.

	An $ f $-inaccessible cardinal is precisely a $ \daleth $-fixed point.
	These are called \defn{$ 1 $-inaccessible cardinals}.
	If $ \kappa $ is $ 1 $-inaccessible, then
	$ V_{\kappa} \models (\zfc + \au) $.
	If $ \zfc + \au $ is consistent, then
	the existence of $ 1 $-inaccessible cardinals is not provable
	by methods formalizable in $ \zfc + \au $.
\end{eg}

Iterating this strategy,
one can now proceed to define $ \alpha $-inaccessibility
for every ordinal $ \alpha $.
Iterating the iteration,
one can define notions of
hyperinaccessibility, hyper${}^{\alpha}$inaccessibility, \emph{etc}.
We cut to the chase:

\begin{axiom}
	The \defn{Lévy scheme} ($ \levy $) is the assertion that
	for every ordinal function $ f $ and every ordinal $ \xi $,
	there exists an $ f $-inaccessible cardinal $ \kappa $ such that $ \xi < \kappa $.
\end{axiom}

\begin{theorem}[\cite{Levy1960, Montague1962, Jorgensen1970}]
	The following are equivalent.
	\begin{enumerate}
		\item The Lévy scheme.
		\item Every normal ordinal function
			has a regular cardinal in its image.
		\item Every closed unbounded subclass $ W \subseteq \Ord $
			contains a regular cardinal.
		\item Every normal ordinal function
			has an inaccessible cardinal in its image.
		\item Every closed unbounded subclass $ W \subseteq \Ord $
			contains an inaccessible cardinal.
	\end{enumerate}
\end{theorem}

\begin{nul}
	The Lévy scheme implies the Axiom of Universes,
	and the consistency strength of $ \nbg + \levy $ is strictly greater
	than that of $ \nbg + \au $.

	The consistency strength of the Lévy scheme is
	also strictly greater than
	the existence of $ \alpha $-inaccessible, hyperinaccessible,
	hyper${}^{\alpha}$inaccessible, \emph{etc}., cardinals.
\end{nul}

\begin{definition}
	Let $ \kappa $ be a regular cardinal.
	One says that $ \kappa $ is \defn{Mahlo} if and only if
	every closed unbounded subset $ W \subseteq \kappa $
	contains a regular cardinal.
\end{definition}

\begin{nul}
	Assume that $ \kappa $ is a Mahlo cardinal.
	Then $ \kappa $ is $ f $-inaccessible for every ordinal function $ f $.
	Accordingly, $ \kappa $ is a fixed point of every normal ordinal function.

	Additionally,
	if $ \kappa $ is a Mahlo cardinal, then
	$ V_{\kappa} \models (\zfc + \levy) $, and similarly
	$ V_{\kappa + 1} \models (\nbg + \levy) $.
	The consistency strength of the axiom
	\enquote{a Mahlo cardinal exists}
	is strictly greater than the Lévy scheme.
\end{nul}

\begin{nul}
	The Lévy scheme and its equivalents and slight variants
	have appeared under various names:
	\enquote{Mahlo's principle} \citep{Gloede1973},
	\enquote{Axiom F} \citep{Drake1974},
	\enquote{$ \Ord $ is Mahlo} \citep{Hamkins2003}.
\end{nul}

For our purposes,
one of the main appeals of the Lévy scheme
is the following.

\begin{theorem}
	Assume $ \levy $.
	
\end{theorem}


