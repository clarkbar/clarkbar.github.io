%!TEX root = book.tex 
% chktex-file 1
% chktex-file 3
% chktex-file 8
% chktex-file 12
% chktex-file 18
% chktex-file 24
% chktex-file 35 
% chktex-file 42

\section{Echelons of accessibility}%
\label{sec:echelonsofaccessibility}
\addcontentsline{toc}{section}{Echelons of accessibility}

The notions of smallness,
accessibility, and presentability of categories
can all be relativized to a Grothendieck universe.

\begin{definition}
	The \defn{echelon} of a category $ C $ is
	the smallest ordinal $ \alpha $ such that
	$ C $ is both
	locally $ \daleth_{\alpha} $-small
	and $ \daleth_{\alpha + 1} $-small.
\end{definition}

\begin{eg}
	The category of finite sets is 
	of echelon $ 0 $.
	More generally, for any ordinal number $ \alpha $,
	the category of $ \daleth_{\alpha} $-small sets is
	of echelon $ \alpha $.
\end{eg}

\begin{notation}
	Let $ \alpha $ be an ordinal number.
	We will denote by $ \Cat_{\alpha} $
	the category of categories that are
	$ \daleth_{\alpha} $-small.

	Accordingly, we will denote by
	$ \Set_{\alpha} $ and $ \An_{\alpha} $
	the categories of $ \daleth_{\alpha} $-small
	sets and animae, respectively.

	The categories $ \Cat_{\alpha} $, $ \Set_{\alpha} $, and $ \An_{\alpha} $
	are all of echelon $ \alpha $.
\end{notation}

\begin{definition}
	Let $ \alpha \geq 1 $ be an ordinal number, and
	let $ \kappa < \daleth_{\alpha} $ be a regular cardinal.
	Let $ C $ and $ D $ be categories
	of echelon $ \leq \alpha $.

	A functor $ f \colon C \to D $
	is \emph{$ \kappa $-continuous of echelon  $ \leq \alpha $} if and only if 
	it preserves all $ \daleth_{\alpha} $-small, $ \kappa $-filtered colimits.
	
	An object $ X $ of $ C $
	is said to be \defn{$ \kappa $-compact of echelon $ \leq \alpha $} if and only if
	the functor $ \yo^X \colon C \to \An_{\alpha} $ corepresented by $ X $
	is $ \kappa $-continuous of echelon $ \leq \alpha $.
	We write $ C^{(\kappa)}_{\alpha} \subseteq C $
	for the full subcategory of $ \kappa $-compact objects of echelon $ \leq \alpha $.

	A category $C$ is
	\defn{$ \kappa $-accessible of echelon $ \leq \alpha $} if and only if
	it satisfies the following quartet of conditions:
	\begin{enumerate}
		\item The category $ C $ is of echelon $ \leq \alpha $.
		\item The category $ C $ has all
			$ \daleth_{\alpha} $-small, $ \kappa $-filtered colimits.
		\item The subcategory $ C^{(\kappa)}_{\alpha} \subseteq C $
			is $ \daleth_{\alpha} $-small.
		\item The subcategory $ C^{(\kappa)}_{\alpha} $
			generates $ C $ under $ \daleth_{\alpha} $-small and 
			$ \kappa $-filtered colimits.
	\end{enumerate}
	
	A category $C$ is
	\defn{$\kappa$-presentable of echelon $ \leq \alpha $}
	if and only if
	it is $ \kappa $ accessible of echelon $ \leq \alpha $, and  
	$ C^{(\kappa)}_{\alpha} $ has all $ \kappa $-small colimits.

	A category $ C $ is
	\defn{accessible of echelon $ \leq \alpha $}
	if and only if
	there exists a regular cardinal $ \kappa < \daleth_{\alpha} $
	such that $ C $ is $ \kappa $-accessible of echelon $ \leq \alpha $.
	It is
	\defn{presentable of echelon $ \leq \alpha $}
	if and only if
	there exists a regular cardinal $ \kappa < \daleth_{\alpha} $
	such that $ C $ is $ \kappa $-presentable of echelon $ \leq \alpha $.
\end{definition}

\begin{eg}
	Let $ \alpha \geq 1 $ be an ordinal number.
	Let $ C $ be a $ \daleth_{\alpha} $-small category.
	The category $ \PP^{\daleth_{\alpha}}(C) $ is then
	$ 0 $-presentable of echelon $ \leq \alpha $.

	Conversely, if $ D $ is a $ 0 $-presentable category
	of echelon $ \leq \alpha $, then 
	there exists a $ \daleth_{\alpha} $-small category $ C $
	and an equivalence $ D \simeq \PP^{\daleth_{\alpha}}(C) $.
\end{eg}

\begin{notation}
	Let $ \alpha \geq 1 $ be an ordinal number, and
	let $ \kappa < \daleth_{\alpha} $ be a regular cardinal.
	We shall write $ \Acc_{\kappa}^{\alpha} \subset \Cat_{\alpha+1} $ for
	the following subcategory.
	\begin{enumerate}
		\item The objects of $ \Acc_{\kappa}^{\alpha} $ are
			the $ \kappa $-accessible categories of echelon $ \leq \alpha $.
		\item The morphisms $ f \colon C \to D $ of $ \Acc_{\kappa}^{\alpha} $ are
			the $ \kappa $-continuous functors of echelon $ \leq \alpha $
			such that $ f(C^{(\kappa)}_{\alpha}) \subseteq D^{(\kappa)}_{\alpha} $.
	\end{enumerate}
	Similarly, we shall write $ \Pr^{\alpha,L}_{\kappa} \subset \Acc_{\kappa}^{\alpha}$
	for the following subcategory.
	\begin{enumerate}
		\item The objects of $ \Pr^{\alpha,L}_{\kappa} $ are
			the $ \kappa $-presentable categories of echelon $ \leq \alpha $.
		\item The morphisms of $ \Pr^{\alpha,L}_{\kappa} $ are
			those functors in $ \Acc_{\kappa}^{\alpha} $
			that preserve all $ \daleth_{\alpha} $-small colimits.
	\end{enumerate}

	Now we may write
	\[
		\Acc^{\alpha} = \bigcup_{\kappa < \daleth_{\alpha}} \Acc_{\kappa}^{\alpha}
		\andeq 
		\Pr^{\alpha,L} = \bigcup_{\kappa < \daleth_{\alpha}} \Pr^{\alpha,L}_{\kappa}
	\]
	for the category of accessible categories of echelon $ \alpha $
	and the category of presentable categories of echelon $ \alpha $, respectively.

	We specify some corresponding subcategories of $ \Cat_{\alpha}$.
	Let $ \Cat_{\alpha}^{\idem} \subset \Cat_{\alpha} $ denote
	the full subcategory consisting of 
	the idempotent-complete $ \daleth_{\alpha} $-small categories.
	Let $ \Cat^{\kappa}_{\alpha} \subseteq \Cat_{\alpha} $ denote the subcategory
	whose objects are $ \daleth_{\alpha} $-small categories
	that possess all $ \kappa $-small colimits
	and whose morphisms are functors that preserve $ \kappa $-small colimits.
	Finally, let 
	\[ \Cat^{\kappa,\idem}_{\alpha} \coloneq \Cat^{\kappa}_{\alpha} \cap \Cat_{\alpha} \period \]
	Please observe that if $ \kappa $ in uncountable, then
	in fact $ \Cat^{\kappa,\idem}_{\alpha} = \Cat^{\kappa}_{\alpha} $,
	because one can split an idempotent using a colimit that is
	$ \aleph_1 $-small and $ \aleph_0 $-filtered
	\cite[Corollary 4.4.5.15 \& Example 5.3.1.9]{Lurie2009}
\end{notation}

\begin{nul}%
	\label{nul:thetafromAcctoCat}
	Let $ \alpha \geq 1 $ be an ordinal number,
	and let $ \kappa < \daleth_{\alpha} $ be a regular cardinal.
	The assignment $ C \mapsto C^{(\kappa)}_{\alpha} $
	defines a functor
	\[ \Acc_{\kappa}^{\alpha} \to \Cat^{\idem}_{\alpha} \period \]
	This functor is an equivalence;
	furthermore, it restricts to an equivalence
	\[
		\Pr^{\alpha,L}_{\kappa} \equivalence
		\Cat^{\kappa,\idem}_{\alpha} \period
	\]
	We will describe the inverses of these equivalences in the next section.
\end{nul}


