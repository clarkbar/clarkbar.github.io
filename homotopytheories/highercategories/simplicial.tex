%!TEX root = book.tex 
% chktex-file 1
% chktex-file 3
% chktex-file 8
% chktex-file 12
% chktex-file 18
% chktex-file 24
% chktex-file 35 
% chktex-file 42

%-------------------------------------------------------------------%
%-------------------------------------------------------------------%
\section{Simplicial objects}%
\label{sec:Simplicialobjects}
%-------------------------------------------------------------------%
%-------------------------------------------------------------------%

%-------------------------------------------------------------------%
\subsection{The simplex category}%
\label{sub:Thesimplexcategory}
%-------------------------------------------------------------------%

If our aim is to convert equality from a property to a structure,
then we might recall the notion of \emph{equivalence relation} on a set $S$.
A \emph{relation} is encoded by a subset $R \subseteq S \times S$, which is the subset of related pairs of elements.
An \emph{equivalence} relation is one that is reflexive, symmetric, and transitive:
\begin{enumerate}
  \item Reflexivity is the condition that $R$ contains the image of the \emph{diagonal map} $S \to S \times S$ given by $x \mapsto (x,x)$.
  \item Symmetry is the condition that $R$ is stable under the involution of $S \times S$ given by $(x,y) \mapsto (y,x)$.
  \item Transitivity states that the projection $S \times S \times S \to S \times S$ given by $(x,y,z) \mapsto (x,z)$ carries the subset $R \times_S R \subseteq S \times S \times S$ to the subset $R$.
\end{enumerate}
An equivalence relation on $S$ can thus be converted into a diagram
\begin{equation}
  \label{eq:equivrel}
  \begin{tikzcd}[sep=1.5em, ampersand replacement=\&]
    R \times_S R \arrow[r] \arrow[r, shift left=1.5ex] \arrow[r, shift right=1.5ex] \& R \arrow[l, shift left=0.75ex] \arrow[l, shift right=0.75ex] \arrow[r, shift left=0.75ex] \arrow[r, shift right=0.75ex] \& S \comma \arrow[l]
  \end{tikzcd}%
\end{equation}
The maps from left to right are various projections.
The maps from right to left are various diagonal maps.
The colimit of this diagram agrees with the coequalizer of the subdiagram 
\[
  \begin{tikzcd}[sep=1.5em, ampersand replacement=\&]
    R \arrow[r, shift left=0.5ex] \arrow[r, shift right=0.5ex] \& S \comma
  \end{tikzcd}%
\]
which is in turn the set of equivalence classes $S/R$.

The case we want to make is that this diagram shape is the start of a more natural \emph{simplicial diagram}, extending infinitely off to the left:
\[
  \simplicial{R_2}{R_1}{R_0}.
\]
This diagram can encode the higher forms of equivalence we sought in the introduction to this chapter.

You may have noticed that our diagram \eqref{eq:equivrel} does not capture the symmetry property of $R$.
It would be a simple matter to incorporate actions of the symmetric groups $\Sigma_2$ and $\Sigma_3$ on $R$ and $R \times_S R$ into our diagram.
This would lead us to the theory of \emph{symmetric simplicial sets}.
We will see in our story that the symmetry is encoded not in the shape of a simplicial diagram of sets, but in the properties we demand of it when it models an anima.
This asymmetry is as much a feature as it is a bug, however:
it will be necessary when we want to connect the theory of simplicial sets to that of categories.

Nothing in the diagram \eqref{eq:equivrel} forces $R$ to be a subset of $S \times S$;
we may demand only that $R$ map to $S \times S$.
The effect of this is to permit the elements of $S$ to be equivalent in many ways.
For example, if $x,y \in S$, then the fiber of the map $R \to S \times S$ is a set
\[
  \enquine{x R y} \comma
\]
which is the set of ways in which $x$ and $y$ are equivalent -- or witnesses to their equivalence.
The middle map from $R \times_S R \to R$ now carries two such witnesses, $\alpha \in \enquine{xRy}$ and $\beta \in \enquine{yRz}$, to a witness $\beta\alpha \in \enquine{xRz}$.
It now becomes natural to ask whether this operation is \emph{associative}:
is $\gamma(\beta\alpha) = (\gamma\beta)\alpha$?
If so, then what we have is the data of a \emph{groupoid} whose objects are the elements of $S$ and whose (iso)morphisms are elements of $R$, so that $\Isom(x,y) = \enquine{xRy}$.

The associativity also lets us extend our diagram \eqref{eq:equivrel} to a larger diagram
\begin{equation}%
  \label{eq:groupoid}
  \begin{tikzcd}[sep=1.5em, ampersand replacement=\&]
    R \times_S R \times_S R\arrow[r, shift left=0.75ex] \arrow[r, shift right=0.75ex] \arrow[r, shift right=2.25ex] \arrow[r, shift left=2.25ex] \& R \times_S R \arrow[l] \arrow[l, shift left=1.5ex] \arrow[l, shift right=1.5ex] \arrow[r] \arrow[r, shift left=1.5ex] \arrow[r, shift right=1.5ex] \& R \arrow[l, shift left=0.75ex] \arrow[l, shift right=0.75ex] \arrow[r, shift left=0.75ex] \arrow[r, shift right=0.75ex] \& S \period \arrow[l]
  \end{tikzcd}%
\end{equation}
These diagrams are becoming complex enough that it will help to be a little more exacting about their indexing categories. 

\begin{definition}
  The \emph{simplex category} $\DDelta$ is category
  whose objects are nonempty, totally ordered, finite sets,
  and whose morphisms the monotonic maps between these.
\end{definition}

Every object of $\DDelta$ is uniquely isomorphic to a finite ordinal
\[
  [n] = \{ 0 < 1 < \cdots < n \}
\]
for some integer $n \geq 0$.
This entitles us to refer to objects of $\DDelta$ as if they are all of this form.

Between adjacent objects, we have the following morphisms:
\begin{itemize}
  \item for every $j \in [n]$, the \emph{face map} is the injective map $\delta_j \colon [n-1] \to [n]$ whose image does not contain $j$;
  \item for every $i \in [n]$, the \emph{degeneracy map} is the surjective map $\sigma_i \colon [n+1] \to [n]$ that carries $i+1$ to $i$.
\end{itemize}
Every other map in the simplex category can be expressed as a composite of face and degeneracy maps.
It is elementary (but boring) to prove that $\DDelta$ is generated by these face and degeneracy maps, subject only to the following relations, called the \emph{simplicial identities}:
\begin{itemize}
  \item if $i \leq j$, then
  \[
    \delta_i \delta_j = \delta_{j+1} \delta_i \semicolon
  \]
  \item if $i \leq j$, then
  \[
    \sigma_j \sigma_i = \sigma_i \sigma_{j+1} \semicolon
  \]
\item for every $i,j$, 
  \[
    \sigma_j \delta_i = \begin{cases}
      \delta_i \sigma_{j-1} & \text{ if } i < j \semicolon \\
      \id                   & \text{ if } i \in \{j,j+1\} \semicolon \\
      \delta_{i-1} \sigma_j & \text{ if } j+1 < i \period
    \end{cases}
  \]
\end{itemize}
In practice, the generators-and-relations description of $\DDelta$ is usually more trouble than it's worth, but
it does provide a schematic picture of the category $\DDelta$:
\[
  \begin{tikzcd}[sep=1.5em, ampersand replacement=\&]
    {[0]} \arrow[r, shift left=0.75ex] \arrow[r, shift right=0.75ex] \&
    {[1]} \arrow[l] \arrow[r] \arrow[r, shift left=1.5ex] \arrow[r, shift right=1.5ex] \&
    {[2]} \arrow[l, shift left=0.75ex] \arrow[l, shift right=0.75ex] \arrow[r, shift left=0.75ex] \arrow[r, shift right=0.75ex] \arrow[r, shift right=2.25ex] \arrow[r, shift left=2.25ex] \&
    {[3]} \arrow[l] \arrow[l, shift left=1.5ex] \arrow[l, shift right=1.5ex] \arrow[r] \arrow[r, shift left=1.5ex] \arrow[r, shift right=1.5ex] \arrow[r, shift left=3ex] \arrow[r, shift right=3ex] \&
    \cdots \comma \arrow[l, shift left=0.75ex] \arrow[l, shift right=0.75ex] \arrow[l, shift left=2.25ex] \arrow[l, shift right=2.25ex]
  \end{tikzcd}%
\]
as well as its opposite $\DDelta^{\op}$:
\[
  \begin{tikzcd}[sep=1.5em, ampersand replacement=\&]
    \cdots \arrow[r] \arrow[r, shift left=1.5ex] \arrow[r, shift right=1.5ex] \arrow[r, shift left=3ex] \arrow[r, shift right=3ex] \&
    {[3]} \arrow[r, shift left=0.75ex] \arrow[r, shift right=0.75ex] \arrow[r, shift right=2.25ex] \arrow[r, shift left=2.25ex] \arrow[l, shift left=0.75ex] \arrow[l, shift right=0.75ex] \arrow[l, shift left=2.25ex] \arrow[l, shift right=2.25ex] \&
    {[2]} \arrow[l] \arrow[l, shift left=1.5ex] \arrow[l, shift right=1.5ex] \arrow[r] \arrow[r, shift left=1.5ex] \arrow[r, shift right=1.5ex] \&
    {[1]} \arrow[l, shift left=0.75ex] \arrow[l, shift right=0.75ex] \arrow[r, shift left=0.75ex] \arrow[r, shift right=0.75ex] \&
    {[0]} \arrow[l]
  \end{tikzcd}%
\]

\begin{notation}
  For any integer $n \geq 0$, we define $\DDelta_{\leq n} \subset \DDelta$ as the full subcategory spanned by the objects $[k]$ with $k \leq n$.
\end{notation}

We see that the diagram \eqref{eq:equivrel} is a functor $\DDelta_{\leq 2}^{\op} \to \Set$, and
the larger diagram \eqref{eq:groupoid} is a functor $\DDelta_{\leq 3}^{\op} \to \Set$.
In fact, there is no obstruction to extending it further, to a functor $\DDelta^{\op} \to \Set$ that carries $[n]$ to the $n$-fold fiber product of $R$ over $S$:
\[
  R \times_S R \times_S \cdots \times_S R \period
\]

This is our first example of a \emph{simplicial object}:
\begin{definition}
  Let $C$ be a category.
  A \emph{simplicial object} of $C$ is a functor $\DDelta^{\op}\to C$.
  A \emph{cosimplicial object} of $C$ is a functor $\DDelta \to C$.
  We write
  \[
    sC \coloneq \Fun(\DDelta^{\op},C)
    \andeq
    cC \coloneq \Fun(\DDelta, C) \period
  \]

  If $X \in sC$, then we write $X_n$ for the value $X([n])$, and
  if $Y \in cC$, then we write $Y^n$ for the value $Y([n])$.
  At times it may be convenient to write $X_\bullet$ and $Y^\bullet$ instead of $X$ and $Y$, just to emphasize the variance.
\end{definition}

We will be interested in simplicial objects of a wide range of categories, but our starting place is the theory of simplicial \emph{sets}.

\begin{eg}
  If $[m], [n] \in \DDelta$, then let us write
  \[
    \Delta_m^n \coloneq \Mor_{\DDelta}([m],[n]) \period
  \]
  Thus for each $[n] \in \DDelta$, we have the simplicial set it represents
  \[
    \Delta^n \coloneq \Mor_{\DDelta}(-,[n]) \colon \DDelta^{\op} \to \Set \period
  \]
  We call $\Delta^n$ the \emph{standard $n$-simplex}.
  The assignment $[n] \mapsto \Delta^n$ is a functor $\DDelta \to s\Set$.
  Equally, for each $[m] \in \DDelta$, we have the cosimplicial set it corepresents
  \[
    \Delta_m \coloneq \Mor_{\DDelta}([m],-) \colon \DDelta \to \Set \comma
  \]
  and the assignment $[m] \mapsto \Delta_m$ is a functor $\DDelta^{\op} \to c\Set$.
\end{eg}

The standard simplices play a critical role in the theory of simplicial sets.
The Yoneda lemma implies that for every simplicial set $X$, we have a natural isomorphism
\[
  X_n = \Mor_{s\Set}(\Delta^n, X) \period
\]
An element $\sigma \in X_n$ -- or equivalently a map $\sigma \colon \Delta^n \to X$ -- is called an \emph{$n$-simplex of $X$}.
This implies that every simplicial set $X \in s\Set$ is the colimit of its simplices.
That is, consider the Yoneda embedding $\DDelta \to s\Set$ given by $[n] \mapsto \Delta^n$, and let
\[
  \DDelta_{/X} \coloneq \DDelta \times_{s\Set} s\Set_{/X} \period
\]
We call $\DDelta_{/X}$ the \emph{category of simplices} of $X$.
We now have a canonical isomorphism
\[
  \colim_{[n]\in\DDelta_{/X}^{\op}} \Delta^n = X \period
\]
Said differently, the left Kan extension of the Yoneda embedding $\DDelta \to s\Set$ is the identity functor on $s\Set$. 

\begin{eg}
  Let $A$ be a small category.
  The \emph{nerve} $NA$ is the simplicial set that carries $[n]$ to the \emph{set} of functors $[n] \to A$.

  In other words, by regarding each nonempty totally ordered finite set $[n]$ as a category, we obtain a fully faithful inclusion $\DDelta \inclusion \Cat$.
  The nerve $NA$ is the composite of $\DDelta^{\op} \inclusion \Cat^{\op}$ with the functor $\Cat^{\op} \to \Set$ represented by $A$.
\end{eg}

The simplicial objects we have been discussing so far have all been extended from finite subcategories $\DDelta_{\leq n}$.
Let's understand how that works.
We write
\[
  s_{\leq n}C \coloneq \Fun(\Delta_{\leq n}^{\op}, C)
  \andeq
  c^{\leq n}C \coloneq \Fun(\Delta_{\leq n}, C)
  \period
\]
Now restriction along the inclusion $\DDelta_{\leq n} \subset \DDelta$ defines functors
\[
  sC \to s_{\leq n}C
  \andeq
  cC \to c^{\leq n}C \period
\]

If $C$ has finite colimits, then these functors each admit a fully faithful left adjoint given by left Kan extension:
\[
  \sk_n \colon s_{\leq n}C \inclusion sC
  \andeq
  \sk_n \colon c^{\leq n}C \inclusion cC \period
\]
These are called the \emph{$n$-skeleton} functors.

If $C$ has finite limits, then these functors each admit a fully faithful right adjoint given by right Kan extension:
\[
  \cosk_n \colon s_{\leq n}C \inclusion sC
  \andeq
  \cosk_n \colon c^{\leq n}C \inclusion cC \period
\]
These are called the \emph{$n$-coskeleton} functors.

We have the usual formulas for these Kan extensions;
for example, in the language of coends and ends:
if $X \in s_{\leq n}C$, then
\[
  \sk_n(X)_m = \int^{k \in \DDelta_{\leq n}} \Delta^k_m \times X_k
  \andeq
  \cosk_n(X)_m = \int_{k \in \DDelta_{\leq n}} X_k^{\Delta^m_k} \comma
\]
and if $Y \in c^{\leq n}C$, then
\[
  \sk_n(Y)^m = \int^{k \in \DDelta_{\leq n}} \Delta^m_k \times Y^k
  \andeq
  \cosk_n(Y)^m = \int_{k \in \DDelta_{\leq n}} (Y^k)^{\Delta^k_m} \comma
\]

We will regularly abuse notation by speaking of $\sk_n(X)$ for a simplicial object $X$;
in this case, one should regard $\sk_n$ as implicitly precomposed with the restriction $sC \to s_{\leq n}C$.

\begin{definition}
  Let $C$ be a category with all finite colimits, and let $n \geq 0$ be an integer.
  A simplicial object $X \in sC$ is \emph{$n$-skeletal} if and only if the natural map $\sk_n(X) \to X$ is an isomorphism.
  Accordingly, $X$ is \emph{$n$-coskeletal} if and only if the natural map $X \to \cosk_n(X)$ is an isomorphism.

  Similar definitions apply for cosimplicial objects.
\end{definition}

\begin{eg}
  Let $X \in sC$ be a simplicial object.
  Our $X$ is $0$-skeletal if and only if it is \emph{constant}.
  It is $0$-coskeletal if and only if it is determined by $X_0$ via the formula
  \[
    X_m = X_0^{\times(m+1)} \period
  \]
\end{eg}



\begin{exercises}
  \item%
    Use the Yoneda lemma to show that for any small category $A$,
    the left Kan extension of the Yoneda embedding $A \inclusion \Fun(A^{\op}, \Set)$ along itself is the identity functor.
\end{exercises}


%-------------------------------------------------------------------%
\subsection{Simplicial and cosimplicial objects}%
\label{sub:Simplicialandcosimplicialobjects}
%-------------------------------------------------------------------%




