%!TEX root = book.tex 
% chktex-file 1
% chktex-file 3
% chktex-file 8
% chktex-file 12
% chktex-file 18
% chktex-file 24
% chktex-file 35 
% chktex-file 42

\chapter*{Higher categories \& Wittgenstein's ladder}%
\label{cha:introduction}

\epigraph{
  My propositions elucidate in the sense that anyone who understands me will eventually recognize them as nonsense, once they have climbed through them -- on them -- beyond them.
  (They must, as it were, throw the ladder away after they have climbed up it.)

  They must surmount these propositions;
  then they will see the world aright.
}{-- Ludwig Wittgenstein, \textit{Tractatus Logico-Philosophicus}, 6.54.}

We cannot yet give a thorough and precise account of higher category theory in the same terms that experienced practitioners use it.
Instead, we have to construct a \emph{model} of higher categories within established set-theoretic foundations.
We will then work within that model to develop a slate of fundamental definitions, constructions, and theorems.
Once enough of this development is complete,
the corresponding \emph{higher category of higher categories} is then unique.
At that point, we are free to ignore the set-level specifics of the chosen model and to work contentedly in a natively higher-categorical way.

This process of abstraction permeates mathematics.
For example, according to set theory, the number $3$ might refer to:
\begin{itemize}
  \item a set with $3$ elements (when viewed as a natural number),
  \item a set with countably many elements (when viewed as an integer or as a rational number), or
  \item a set with uncountably many elements (when viewed as a real number).
\end{itemize}
Mathematicians do not fret much about this ambiguity, because they know there are preferred injections
\[
  \NN \inclusion \ZZ \inclusion \QQ \inclusion \RR \period
\]
These injections are in fact \emph{unique} with certain simple properties.
For example, there is exactly one way to embed $\QQ$ as a subfield of $\RR$.

Better still, each of these number systems satisfies a universal property:
\begin{itemize}
  \item $\NN$ is the free commutative monoid on one generator;
  \item $\ZZ$ is the group completion of $\NN$;
  \item $\QQ$ is the field of fractions of $\ZZ$;
  \item $\RR$ is the Cauchy completion of $\QQ$.
\end{itemize}
Each of these universal properties characterizes maps from these objects in certain categories.

To \emph{construct} these objects set-theoretically and to prove that they have the desired properties is sometimes a chore.
There are often many constructions of the same object.
For example, are real numbers Dedekind cuts, or are they equivalence classes of Cauchy sequences of rational numbers?
If we ask questions that involve the \emph{elements} of a real number, then we can observe differences between these two options. 
But if the kinds of questions we study only involve the things for which we usually use real numbers -- adding, subtracting, multiplying, dividing, and taking limits -- then any differences among the various constructions of $\RR$ are unimportant.

Even further, working mathematicians don't generally regard the question \enquote{is $3 \in \pi$?} as \emph{meaningful}, even though it technically will have an answer.
It is precisely \emph{because} the answer depends upon the set-theoretic model one chooses of the real numbers that we aren't supposed to ask that sort of question.
That is, we adopt a \emph{structuralist} attitude to the real numbers.
Every complete ordered field has a unique element that deserves the name $\pi$, and
what we call the \emph{meaningful} questions about $\pi$ make reference only to that structure.
Since any pair of complete ordered fields are topologically isomorphic in exactly one way, the answers do not depend on any particular construction of the complete ordered field.

Note that we may still \emph{answer} meaningful questions by appealing to a particular model.
One may deduce facts about $\pi$ by understanding it as an equivalence class of Cauchy sequences.
But those facts are only meaningful (in our sense) if they can be stated only with reference to the structures present in a complete ordered field.

This pattern of mathematical development resembles Wittgenstein's parable.
We climb a ladder by constructing a mathematical object $X$ in a set-theoretical manner.
Once we've done this, we identify the salient \emph{structures} and \emph{properties} of $X$.
We write enough of these down to characterize $X$ uniquely, up to an appropriate notion of \emph{isomorphism}.
The structures we specify will, perhaps, reveal to us a category $\CC$ of which it should be viewed as an object, and the properties will identify it uniquely up to unique isomorphism in $\CC$.
We then restrict our attention to the questions about $X$ that can be formulated in terms provided by $\CC$,
and we regard questions about $X$ that do not admit such formulations as meaningless.
Details about our initial construction of $X$ become uninterrogable, and thus we throw away the ladder.

The theory of higher categories functions also follows this pattern, to a point. 
In fact, there are many ways to construct a theory of higher categories.
These constructions vary significantly in detail.
However, there are also today a number of different \emph{unicity theorems};
these state that
any two purported models of higher categories satisfying a handful of reasonable axioms are equivalent in an appropriate sense.
In other words, all of these are models of the same structure.
That structure is the \emph{higher category of higher categories}.
We will have more to say about this below.

Before we do, though, we must acknowledge that this does seem to introduce a little wrinkle in our story.
After all, $\RR$ is not itself a real number;
one can define complete ordered fields, without making reference to an existing theory of real numbers.
But higher categories do form a higher category.
Is it not paradoxical to assert that there is a unique higher category of higher categories?
Are we not trying to throw away our ladder while we are still standing on it?
After all, how can one characterize the theory of higher categories as a higher category before one has fully worked out what a higher category is?
And how can one claim to have fully worked out what a higher category is without having characterized the theory of higher categories?

The strategy is to begin with a definition (in set-theoretic terms) of \emph{putative} higher categories.
We show that putative higher categories form a putative higher category $\categ{Put}$.
Then we uniquely characterize an object $\categ{True}$ of $\categ{Put}$, making reference only to the structure available in $\categ{Put}$.
We call $\categ{True}$ the putative higher category of \emph{true} higher categories.
Then we prove that $\categ{Put}$ satisfies the conditions of our characterization.
Hence the putative higher category $\categ{Put}$ of putative higher categories is equivalent (in a unique fashion) to the putative higher category $\categ{True}$ of true higher categories.

It is easy to tell a degenerate version of this story.
I define a putative higher category to be a one-point set.
The category $\categ{Put}$ of one-point sets is, up to equivalence, a one-point set itself.
We characterize $\categ{True}$ as the unique object of $\categ{Put}$ (up to unique isomorphism).
Necessarily, $\categ{Put} \cong \categ{True}$.
Obviously, we want a theory of higher categories with more content than this one.

We conclude that the theory of higher categories, as we can develop it today, differs from many other mathematical abstractions in one important respect.
Whereas we are eventually able to find new (and often clearer) \emph{definitions} of many mathematical objects by adopting structuralist attitudes,
we do not have a way of constructing a theory of higher categories without introducing a model of the theory at some point.
We can, for example, give a structuralist definition of \enquote{real number} as an element of a complete ordered field $\RR$;
such a definition doesn't exhibit any preference for any particular construction of $\RR$.
By contrast, we do not currently know how to give a completely structuralist definition of \enquote{higher category}.
What we have is a structuralist \emph{characterization} of the theory of higher categories, when this theory is regarded as an object in some extant model of higher categories.
So it appears that we cannot avoid the difficult task of constructing and studying an explicit model of higher categories.
Such a model remains entangled with the theory of higher categories,
even when the goal is a model-independent understanding of the theory.

What this means for us is that, if we are to undertake a well-motivated study of higher categories, then we will have to spend some time developing good intuitions about these objects. 

\section*{The landscape of higher categories}%
\label{sec:landscape}

At first it's hard to see what all the fuss is about.
We already know what $0$-categories are: they're sets.
We also know what $1$-categories are: they're categories.
So a $1$-category has a collection of objects and, between every pair $x,y$ of objects, one has a $0$-category (=set) $\Mor(x,y)$ of maps.
Composition is then a map
\[
  \Mor(x,y) \times \Mor(y,z) \to \Mor(x,z)
\]
given by $(f,g) \mapsto g \circ f$.
The right way to iterate this definition seems uncontroversial.

Let $\VV$ be a category with all finite products.
A \emph{category $\CC$ enriched in $\VV$} consists of:
\begin{itemize}
  \item a collection $\Obj \CC$ of objects;
  \item between every pair $x,y \in \Obj \CC$, an object
  \[
    \Mor_{\CC}(x,y) \in \Obj \VV \semicolon
  \]
  \item for every $x \in \Obj \CC$, a morphism $\id_x \colon 1_{\VV} \to \Mor_{\CC}(x,x)$ (where $1_{\VV}$ denotes the terminal object of $\VV$); and
  \item for every $x,y,z \in \Obj \CC$, a morphism
    \[
      \Mor_{\CC}(x,y) \times \Mor_{\CC}(y,z) \to \Mor_{\CC}(x,z) \comma
    \]
    written $(f,g) \mapsto g \circ f$.
\end{itemize}
These data are subject to the usual axioms: composition is associative ($(h \circ g) \circ f = h \circ (g \circ f)$), and unital ($f \circ \id_x = \id_y \circ f = f$).
There's an attached notion of \emph{enriched functor}, whose definition is predictable.

Now we can give an easy recursive definition.
Our base case is the notion of \emph{strict $0$-categories} and the category $\Cat^{\textit{str}}_0 \coloneq \Set$.
Then we define a \emph{strict $n$-category} as a category enriched in $\Cat^{\textit{str}}_{n-1}$,
and we define $\Cat^{\textit{str}}_n$ as the category of $\Cat^{\textit{str}}_{n-1}$-enriched categories and $\Cat^{\textit{str}}_{n-1}$-enriched functors.

Thus a strict $2$-category consists of a set $\Obj \CC$ of objects;
between every $a,b \in \Obj \CC$, a set $\Mor_{\CC}(a,b)$ of morphisms $a \to b$; and
between every $f,g \in \Mor_{\CC}(a,b)$, a set $\Mor_{\Mor_{\CC}(a,b)}(f,g)$ of $2$-morphisms $f \to g$.
Morphisms and $2$-morphisms each have composition laws, which are associative and unital.
Since $\Mor_{\CC}(a,b)$ is actually a category, the composition law $\Mor_{\CC}(a,b) \times \Mor_{\CC}(b,c) \to \Mor_{\CC}(a,c)$ is a functor.
In particular, if $f,g \colon a \to b$ and $u,v \colon b \to c$ are morphisms,
and if $\eta \colon f \to g$ and $\theta \colon u \to v$ are $2$-morphisms, then we have a \emph{horizontal composition} $\theta \ast \eta \colon u \circ f \to v \circ g$. 
The horizontal composition is related to the vertical composition by means of an \emph{interchange formula}:
\[
  (\nu \ast \mu) \circ (\theta \ast \eta) = (\nu \circ \theta) \ast (\mu \circ \eta) \period
\]

Unwinding the recursion, we see that a strict $n$-category $\CC$ has:
\begin{itemize}
  \item a collection $\Obj \CC$ of \emph{objects} or \emph{$0$-morphisms};
  \item for every pair $a,b \in \Obj(\CC)$ of objects, a collection $\Mor_{\CC}(a,b)$ of \emph{morphisms} or \emph{$1$-morphisms} with \emph{source} $a$ and \emph{target} $b$;
  \item for every pair $f,g \in \Mor_{\CC}(a,b)$ of $1$-morphisms (which we say are \emph{parallel} because they have the same source, and they have the same target), a collection $\Mor_{\Mor_{\CC}(a,b)}(f,g)$ of \emph{$2$-morphisms} with source $f$ and target $g$;
  \item for every parallel pair $\alpha,\beta \in \Mor_{\Mor_{\CC}(a,b)}(f,g)$ of $2$-morphisms, a collection
    \[
      \Mor_{\Mor_{\Mor_{\CC}(a,b)}(f,g)}(\alpha,\beta)
    \]
    of \emph{$3$-morphisms} with source $\alpha$ and target $\beta$;
\end{itemize}

\dots

\begin{itemize}
  \item for every parallel pair $\phi, \psi$ of $(n-1)$-morphisms, a collection
    \[
    \Mor_{\Mor_{\;{\ddots}_{\;\Mor_{\CC}(a,b)\;}{\revddots}\;}}(\phi,\psi)
    \]
    $\Mor(\phi,\psi)$ of \emph{$n$-morphisms} with source $\phi$ and target $\psi$.
\end{itemize} 
Each of these kinds of morphisms can be composed 


There is a theory of \emph{$(n,k)$-categories} for $0 \leq k \leq n \leq +\infty$.
\emph{When are two mathematical objects the same?}
Let's start with natural numbers $a$ and $b$.
Perhaps you want to prove that $a$ and $b$ are the same.
\emph{How might you go about that?}
Well, it depends upon how $a$ and $b$ arose in your thinking.
Since they are natural numbers, it's likely you found them by counting something.
In other words, $a$ is the cardinality of a finite set $A$, and
$b$ is the cardinality of a finite set $B$.

Perhaps $A$ is the set of nontrivial ways of (correctly and nonredundantly) parenthesizing an expression like $abcd$ with two pairs of matching parentheses,
such as $(a(bc))d$.
Perhaps $B$ is the set of full binary trees with $4$ leaves.
It would be perfectly valid to prove that $a=b$ by computing each of these numbers and compare the answers, but
one feels as though one has 

At first it seems a little strange to speak about the category of categories.
These kinds of constructions seem useful and interesting.
