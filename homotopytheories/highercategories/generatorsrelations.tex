%!TEX root = book.tex 
% chktex-file 1
% chktex-file 3
% chktex-file 8
% chktex-file 12
% chktex-file 18
% chktex-file 24
% chktex-file 35 
% chktex-file 42

\subsection{Interlude: generators of categories \inc}%
\label{sub:generatorsrelations}
We mentioned above that every simplicial set is the colimit of its simplices.
In other words, every simplicial set is well approximated by simplices.

More generally, let $\CC$ be a category with all small colimits.
In most cases, $\CC$ will be a large category.%
\footnote{
  This is an old argument of Peter Freyd.
  In fact, a small category $\CC$ with all small coproducts is automatically a \emph{preorder}.
  That is, if $x,y \in \CC$ are two objects, then we claim that there is at most one map $x \to y$.
  To prove this, suppose that $\kappa$ is the cardinality of the set of morphisms of $\CC$, and consider the coproduct $\kappa \cdot x = \coprod_{\alpha < \kappa} x$.
  Now $\CC(\kappa \cdot x, y) = \CC(x,y)^{\kappa}$, so by Cantor, if $\#\CC(x,y) \geq 2$, then $\#\CC(\kappa\cdot x, y) \geq 2^\kappa > \kappa$, which contradicts our assumption.
}
We are interested in situations in which, even though $\CC$ is large, all of its objects are well approximated -- or even \emph{determined} -- by a small subcategory $A \subseteq \CC$ via colimits.
This will be an idea that plays a big role in our work, so let's discuss it carefully.

\begin{definition}
  Let $\CC$ be a category with all colimits, and let $A \subseteq \CC$ be a small full subcategory.
  We say that $A$ \emph{generates $\CC$ under colimits} if the smallest full subcategory $\DD \subseteq \CC$ that is closed under colimits and contains $A$ is $\CC$ itself.

  We say that $A$ \emph{strongly generates $\CC$ under colimits} -- or that $A$ is \emph{dense} in $\CC$ -- if the left Kan extension of the inclusion $A \subseteq \CC$ along itself is the identity functor on $\CC$.
\end{definition}

For every object $X \in \CC$, we have the category
\[
  A_{/X} \coloneq A \times_{\CC} \CC_{/X} \comma
\]
whose objects are pairs $(a,f)$ consisting of an object $a \in A$ and a morphism $f \colon a \to X$ in $\CC$.
The left Kan extension of the inclusion $A \inclusion \CC$ along itself is the functor that carries an object $X \in \CC$ to the colimit $\colim_{a \in A_{/X}} a$.
Thus $A$ is dense in $\CC$ if and only if, for every object $X \in \CC$, the natural morphism
\[
  \colim_{a \in A_{/X}} a \to X
\]
is an isomorphism.
So if $A$ strongly generates $\CC$, then every object of $\CC$ is a colimit of objects in $A$ in a canonical way.
So certainly if $A$ is dense in $\CC$, then it generates $\CC$ under colimits, but the converse is far from true.
In fact, if $A$ generates $\CC$ under colimits, then it need not even be the case that every object of $\CC$ can be written as a colimit of objects from $A$.

\begin{eg}
  Consider the full subcategory $\{\pt\} \subset \Set$ generated by the one-point set.
  The category $\{\pt\}$ is dense in $\Set$.
  To see this, let $S$ be a set, and
  consider the category $\{\pt\}_{/S}$.
  An object of $\{\pt\}_{/S}$ is nothing more than an element of $S$,
  and there are no non-identity morphisms in $\{\pt\}_{/S}$ (because there are none in $\{\pt\}$!).
  Hence $\{\pt\}_{/S}$ is the set $S$, viewed as a category.
  Thus
  \[
    \colim_{s \in \{\pt\}_{/S}} s = \coprod_{s \in S} \{s\} = S \period
  \]
\end{eg}

\begin{eg}
  Let's generalize this example.
  Let $A$ be a small category, and let $\PP(A)$ be the category of \emph{presheaves} on $A$ --
  \emph{i.e.}, functors $A^{\op} \to \Set$.
  Thus
  \[
    \PP(A) = \Fun(A^{\op}, \Set) \comma
  \]
  and in particular $s\Set = \PP(\Delta)$.
  Now via the Yoneda embedding $\yo \colon A \inclusion \PP(A)$, we consider $A$ as a full subcategory of $\PP(A)$.

  The category $A$ is dense in $\PP(A)$.
  To prove this, consider a presheaf $F \colon A^{\op} \to \Set$.
  An object of the category $A_{/F}$ is a pair $(Y,f)$ consisting of an object $Y \in A$ and a natural transformation $f \colon \yo_Y \to F$.
  By Yoneda, such a natural transformation is the same thing as an element $f \in F(Y)$.
  Thus $A_{/F}$ is sometimes called the \emph{category of elements of $F$}.
  To prove that the canonical natural transformation
  \[
    \eta \colon \colim_{Y \in A_{/F}} \yo_Y \to F
  \]
  is an isomorphism, let $Z \in A$ be an object, and consider the map
  \[
    \eta_Z \colon \colim_{Y \in A_{/F}} \Mor_A(Z,Y) \to F(Z) \period
  \]
  The elements of the colimit on the left are equivalence classes of triples $(Y,f,\phi)$ consisting of an object $Y \in A$, an element $f \in F(Y)$, and a morphism $\phi \colon Z \to Y$ in $A$.
  The equivalence relation is generated by the demand that if $(Y,f,\phi)$ and $(Y',f',\phi')$ are such triples, and if there exists a map $\gamma \colon Y \to Y'$ such that $\phi' = \gamma\phi$ and $F(\gamma)(f') = f$, then $(Y,f,\phi)$ and $(Y',f',\phi')$ are equivalent.
  The map $\eta_Z$ carries such a triple to $F(\phi)(f) \in F(Z)$.
  The aim is to show that this map is a bijection.

  So let $g \in F(Z)$.
  Then $(Z,g) \in A_{/F}$ is an object, and $\id_Z \in A(Z,Z)$ provides an element $(Z,g,\id_Z)$ of the colimit on the left, and $\eta_Z(Z,g,\id_Z) = g$.
  Furthermore, if $(Y,f,\phi)$ is an element of the colimit such that $F(\phi)(f) = \eta_Z(Y,f,\phi) = g$, then
  by definition, $\phi \colon Z \to Y$ is a morphism such that $\phi = \phi \id_Z$ and $F(\phi)(f) = g$;
  thus $(Y,f,\phi)$ is equivalent to $(Z,g,\id_Z)$.
\end{eg}

\begin{eg}
  Let $\Ab$ denote the category of abelian groups, and
  let $F \subset \Ab$ be the full subcategory spanned by the single object $\ZZ$.
  (In other words, $F$ is the category of free abelian groups of rank $1$.)
  What kinds of colimits can we form from $F$?

  Certainly any free abelian group is a colimit of objects of $F$, since we can form the coproducts
  \[
    \ZZ\{S\} \coloneq \bigoplus_{s \in S} \ZZ
  \]
  for every set $S$.
  Now if $E$ is any abelian group, then we claim that we can write $E$ as a colimit of free abelian groups.

  There's really only one way to prove that an object can be written as a colimit of some \enquote{nice} objects:
  we have to somehow cook up a good supply of maps from the nice objects to our object. 
  In our case, we can get away with a single map, as long as it is an epimorphism.
  We select a set $S \subseteq E$ of generators for $E$.
  (If we're feeling lazy or wasteful, we can choose $S = E$.)
  This defines an epimorphism
  \[
    \phi \colon \ZZ\{S\} \to E \period
  \]
  Now the kernel of $\phi$ is a subgroup $K \subseteq \ZZ\{S\}$ such that $E = \ZZ\{S\}/K$.
  It turns out that $K$ itself is automatically free, but
  that isn't obvious, and in any case we don't really need that fact.
  All we need to do is select some generators $T \subseteq K$.
  Now we have a homomorphism $\ZZ\{T\} \to \ZZ\{S\}$ whose image is $K$, so that $E$ is a pushout: 
  \[
    \begin{tikzcd}[sep=1.5em, ampersand replacement=\&]
      \ZZ\{T\} \ar[r] \ar[d] \& \ZZ\{S\} \ar[d] \\
      0        \ar[r]        \& E 
    \end{tikzcd}
  \]
  
  We have thus shown that every abelian group can be written as a colimit of free groups, and that every free group can be written as a colimit of copies of $\ZZ$.
  This implies that our subcategory $F$ generates $\Ab$ under colimits.

  But $F$ does not \emph{strongly} generate $\ZZ$.
  To see why not, let's consider the category $F_{/E}$ for an abelian group $E$.
  An object of $F_{/E}$ is a homomorphism $a \colon \ZZ \to E$;
  such a homomorphism is determined uniquely by the value $a(1)$, so we may regard the objects of $F_{/E}$ as \emph{elements} of $E$.
  In this way, if $a, b \in E$ are two such objects, then
  a morphism $m \colon a \to b$ is an integer $m$ such that $mb = a$ in $E$.
  Now if $E = \ZZ \oplus \ZZ$, then the elements of the form $(a,1)$ or $(1,b)$ each give rise to infinite order elements in the colimit.
  Hence the colimit is much larger than $\ZZ \oplus \ZZ$.
\end{eg}

\begin{proposition}
  Let $\CC$ be a category with all colimits, and let $A \subseteq \CC$ be a small full subcategory.
  Write $i \colon A \inclusion \CC$ for the inclusion functor.
  Then $A$ is dense in $\CC$ if and only if the restricted Yoneda
  \[
    i^{\ast} \yo \colon \CC \to \PP(A) \comma
  \]
  which carries $X \in \CC$ to the functor $Y \mapsto \Mor_{\CC}(Y,X)$, is fully faithful.
\end{proposition}

\begin{proof}
  The left Kan extension of $i \colon A \to \CC$ along the Yoneda embedding is a left adjoint $\yo_!i \colon \PP(A) \to \CC$ to the functor $i^{\ast} \yo$.
  This functor can be evaluated as
  \[
    (\yo_!i)(F) = \colim_{Y \in A_{/(F \circ i)}} Y = \int^{Y \in A} F(Y) \times Y \period
  \]
  The functor $i^{\ast}\yo$ is fully faithful if and only if the counit $\eta \colon (\yo_!i)(i^{\ast}\yo) \to \id_{\CC}$ is an isomorphism.
  For every object $X \in \CC$, we observe that the category
  \[
    A_{/i^{\ast}\yo_X}
  \]
  is equivalent to $A_{/X}$;
  hence
  \[
    (\yo_!i)(i^{\ast}\yo_X) \cong \colim_{Y \in A_{/X}} Y \comma
  \]
  and the morphism $\eta_X$ is the natural maps.
  Since $A \subseteq \CC$ is dense, it follows that this map is an isomorphism.
\end{proof}

Suppose that $\CC$ is a category with colimits, and
let us suppose that we want to write down a functor $s\Set \to \CC$ that preserves colimits.
We can restrict this functor along Yoneda to a functor $\DDelta \to \CC$ along the Yoneda embedding.
It turns out that this restriction loses no information.

\begin{proposition}%
  \label{prp:PAfreelygeneratedbyA}
  Let $\DD$ be a category with all colimits,
  and let $A$ be a small category.
  Let
  \[
    \Fun^L(\PP(A),\DD) \subseteq \Fun(\PP(A),\DD)
  \]
  be the full subcategory spanned by those functors that preserve colimits.
  Then the restriction along the Yoneda embedding
  \[
    \Fun^L(\PP(A),\DD) \to \Fun(A,\DD)
  \]
  is an equivalence of categories.
\end{proposition}

\begin{proof}
  Exercise \ref{exercise:PAfreelygenerated}.
\end{proof}
