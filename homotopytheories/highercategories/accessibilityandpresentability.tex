%!TEX root = book.tex 
% chktex-file 1
% chktex-file 3
% chktex-file 8
% chktex-file 12
% chktex-file 18
% chktex-file 24
% chktex-file 35 
% chktex-file 42

\section{Accessibility \& presentablity}%
\label{sec:accessibilityandpresentablity}
\addcontentsline{toc}{section}{Accessibility \& presentability}

\begin{definition}
	Let $ \kappa $ be a regular cardinal.
	A category $ \Lambda $ is \defn{$ \kappa $-filtered} if and only if
	it satisfies the following equivalent conditions:
	\begin{enumerate}
		\item For every $ \kappa $-small category $ J $,
			every functor $ f \colon J \to \Lambda $ can be extended
			to a functor $ F \colon J^{\rhd} \to \Lambda $.
		\item For every $ \kappa $-small category $ J $
			and every functor $ H \colon \Lambda \times J \to \An^V$,
			the natural morphism
			\[
				\colim_{\lambda \in \Lambda} \lim_{j \in J} H(\lambda,j)
				\to
				\lim_{j \in J} \colim_{\lambda \in \Lambda} H(\lambda,j)
			\]
			is an equivalence.
		\item For every $ \kappa $-small category $ J $,
			the diagonal functor
			$ \Lambda \to \Fun(J, \Lambda) $
			is cofinal.
	\end{enumerate}
\end{definition}

\begin{eg}
	For any regular cardinal $ \kappa $,
	the ordinal $ \kappa $, regarded as a category,
	is $ \kappa $-filtered.

	More generally, a poset is $ \kappa $-filtered if and only if
	every $ \kappa $-small subset thereof is dominated by some element.
\end{eg}

\begin{eg}
	A $ \kappa $-small category is $ \kappa $-filtered
	if and only if it contains a terminal object.
\end{eg}

\begin{eg}
	Since no category is $ 0 $-small,
	every category is $ 0 $-filtered.
\end{eg}

\begin{definition}
	Let $ \kappa $ be a regular cardinal.
	A functor $ f \colon C \to D $ between large-categories
	will be said to be \defn{$ \kappa $-continuous}
	if and only if it preserves $ \kappa $-filtered colimits.
	
	An object $ X $ of a locally small large-category $ C $
	is said to be \defn{$ \kappa $-compact} if and only if
	the functor $ \yo^X \colon C \to \An^V$ corepresented by $ X $
	(i.e., the functor $ Y \mapsto \Map_C(X,Y) $)
	is $ \kappa $-continuous.
	We write $ C^{(\kappa)} \subseteq C $
	for the full subcategory of $ \kappa $-compact objects.

	A large-category $C$ is \defn{$\kappa$-accessible} if and only if
	it satisfies the following conditions:
	\begin{enumerate}
		\item The category $ C $
			is locally small.
		\item The category $ C $
			has all $ \kappa $-filtered colimits.
		\item The subcategory $ C^{(\kappa)} \subseteq C $ is small. 
		\item The subcategory $ C ^{(\kappa)} \subseteq C $ generates $ C $
			under $ \kappa $-filtered colimits.
	\end{enumerate}
	
	A $ \kappa $-accessible large-category $C$ is
	\defn{$\kappa$-presentable}%
	\footnote{Some authors use the phrase
		\defn{$ \kappa $-compactly generated} instead.}
	if and only if $ C^{(\kappa)} $
	has all $ \kappa $-small colimits.
\end{definition}

\begin{eg}
	A $ 0 $-continuous functor is one that preserves all colimits.
	Hence a $ 0 $-compact object $ X $ is one in which the natural map 
	\[
		\Map(X, \colim_{\alpha \in A} Y_{\alpha}) \equivalence
		\colim_{\alpha \in A} \Map(X, Y_{\alpha})
	\]
	is an equivalence,
	irrespective of the category $ A $ or the diagram $ Y \colon A \to C $.

	The following are equivalent for a large-category $ C $.
	\begin{enumerate}
		\item There exists a small full subcategory $ D \subseteq C $ whose
			inclusion extends along the Yoneda embedding
			to an equivalence of categories
			\eqref{not:Vsetsanimaepresheavescategories}
			\[ \PP^V(D) \equivalence C \period \]
		\item The full subcategory $ C^{(0)} \subseteq C $ of $ 0 $-compact objects
			is small, and
			its inclusion extends along the Yoneda embedding
			to an equivalence of categories
			\[ \PP^V(C^{(0)}) \equivalence C \period \]
		\item The large-category $ C $ is $ 0 $-accessible.
		\item The large-category $ C $ is $ 0 $-presentable.
	\end{enumerate}
\end{eg}

\begin{eg}
	Let $ \kappa $ be an uncountable regular cardinal.
	Then the following are equivalent for a category $ C $.
	\begin{enumerate}
		\item The category $ C $ is $ \kappa $-small.
		\item The set of equivalence classes
			of objects of $C$ is $ \kappa $-small,
			and for every morphism
			$ f \colon X \to Y $ of $ C $
			and every $ n \in \NN_0$,
			the set $ \pi_n(\Map_C(X,Y),f) $ is
			$ \kappa $-small.
		\item The category $ C $ is $ \kappa $-compact
			as an object of $ \Cat^V $;
			that is, $ \Cat^{\kappa} = \Cat^{V,(\kappa)} $.
	\end{enumerate}
	In particular, an anima $ X $ is $ \kappa $-small if and only if
	all its homotopy sets are $ \kappa $-small, if and only if
	it is $ \kappa $-compact as an object of $ \An^V $.
\end{eg}

\begin{eg}
	The equivalence above is doubly false if $ \kappa = \aleph_0 $.

	First, we certainly have a containment
	\[ \Cat^{\aleph_0} \subset \Cat^{V,(\aleph_0)} \comma \]
	but this containment is proper.
	An $ \aleph_0 $-compact anima is a \emph{retract} of
	an $ \aleph_0 $-small anima, but 
	it may not be $ \aleph_0 $-small itself.
	If $ X $ is $ \aleph_0 $-compact and \emph{simply connected},
	then $ X $ is $ \aleph_0 $-small, but
	for non-simply-connected animae,
	we have the \emph{de Lyra--Wall finiteness obstruction},
	which lies in the reduced $ K_0 $ of the group ring $ \ZZ[\pi_1(X)] $.

	Second, the homotopy sets of an $ \aleph_0 $-small $ X $ anima
	are not generally finite.
	By a theorem of Serre,
	if each connected component $ Y \subseteq X $
	has finite fundamental group, then
	its homotopy groups are finitely generated.
	But if $ \pi_1(X) $ isn't finite,
	this too fails;
	for example, $ \pi_3(S^1 \vee S^2) $ is not finitely generated.

	It is still true that the category $ \Cat^V $ is
	$ \aleph_0 $-presentable.
\end{eg}

\begin{nul}
	Let $ \kappa \leq \lambda $ be regular cardinals.
	A $ \kappa $-small category is $ \lambda $-small.
	A $ \lambda $-filtered category is $ \kappa $-filtered.
	A $ \kappa $-continuous functor is $ \lambda $-continuous.
	In general, however, there are $ \kappa $-accessible categories
	that are not $ \lambda $-accessible.
\end{nul}

\begin{definition}
	Let $ \kappa $ and $ \lambda $ be regular cardinals.
	We write $ \kappa \ll \lambda $
	if and only if,
	for every pair of cardinals
	$ \kappa_0 < \kappa $ and $ \lambda_0 < \lambda $,
	one has $ \lambda_0^{\kappa_0} < \lambda $.
	Equivalently, $ \kappa \ll \lambda $ if and only if,
	for every $ \kappa $-small set $A$
	and every $ \lambda $-small set $ B $,
	the set $ \Map(A, B) $ is $ \lambda $-small.
\end{definition}

\begin{eg}
	For every regular cardinal $ \kappa $,
	one has $ 0 \ll \kappa $.
\end{eg}

\begin{eg}
	For every infinite regular cardinal $ \kappa $,
	one has $ \aleph_0 \ll \kappa $.
\end{eg}

\begin{nul}
	Let $ \kappa $ and $ \lambda $ be regular cardinals.
	How is the condition $ \kappa \ll \lambda $ used in practice?
	The answer comes down to the following pair of manoeuvres,
	which we can do whenever $ \kappa \ll \lambda $.

	If $ J $ is a $ \lambda $-small poset,
	then we can write
	\[ J = \bigcup_{\el \in \Lambda} J_{\el} \comma \]
	where $ \Lambda $ is a $ \lambda $-small and $ \kappa $-filtered poset,
	and each $ J_{\el} \subseteq J $ is a $ \kappa $-small poset.
	In this way, we may express
	any $ \lambda $-small colimit as
	a $ \lambda $-small and $ \kappa $-filtered colimit of
	$ \kappa $-small colimits:
	\[ \colim_{j \in J} X(j) \simeq \colim_{\el \in \Lambda} \colim_{j \in J_{\el}} X(j) \]
	\citep[Corollary 4.2.3.11]{Lurie2009}.

	On the other hand, if $ M $ is a $ \kappa $-filtered poset,
	then we can write
	\[ M = \bigcup_{k \in K} M_k \comma \]
	where $ K $ is a $ \lambda $-filtered poset,
	and each $ M_k \subseteq M $ is $ \lambda $-small and $ \kappa $-filtered.
	In this way, we may express
	any $ \kappa $-filtered colimit as
	a $ \lambda $-filtered colimit of
	$ \lambda $-small and $ \kappa $-filtered colimits:
	\[ \colim_{m \in M} Y(m) \simeq \colim_{k \in K} \colim_{m \in M_k} Y(m) \]
	\citep[Lemma 5.4.2.10]{Lurie2009}.
\end{nul}

\begin{proposition}[\protect{\citealp[Proposition 5.4.2.11]{Lurie2009}}]
	If $ \kappa \ll \lambda $ are regular cardinals,
	then every $ \kappa $-accessible category
	is $ \lambda $-accessible.
	Similarly, every $ \kappa $-presentable category
	is $ \lambda $-presentable.
\end{proposition}

\begin{definition}
	A large-category $ C $ is \defn{accessible} if and only if
	there exists a regular cardinal $ \kappa $ such that
	$ C $ is $ \kappa $-accessible.

	We shall say that $ C $ is \defn{presentable} if and only if
	there exists a regular cardinal $ \kappa $ such that
	$ C $ is $ \kappa $-presentable.
\end{definition}

\begin{eg}
	A small category is accessible if and only if it is idempotent-complete
	\citep[Corollary 5.4.3.6]{Lurie2009}.
\end{eg}

\begin{nul}
	A large-category is presentable if and only if
	it is accessible and has all colimits.
	A presentable large-category automatically has all limits as well.
\end{nul}

\begin{definition}%
	\label{dfn:locallypresentable}
	A large-category $ C $ is \defn{locally presentable}%
	\footnote{In the $1$-category literature,
		the phrase \emph{locally presentable category} is used for
		what we call \emph{presentable category}.}
	if and only if every object $ X \in C $
	is contained in a presentable full large-subcategory $ C' \subseteq C $
	such that the inclusion $ C' \inclusion C $ preserves colimits.
\end{definition}

\begin{nul}
	In other words, a large-category $ C $ is locally presentable
	just in case it can be expressed as
	a class-indexed union of presentable large-categories,
	each of which is embedded in $ C $ via
	a colimit-preserving, fully faithful functor.
\end{nul}

\section{Presheaf categories}%
\label{sec:presheaf_categories}

Let $ C $ be a large-category.
What happens if we seek to make sense in $ \nbg $
of the category $ \tau_0\PP^V(C) $ of presheaves of sets
$ C^{\op} \to \Set^V $?

Right away we encounter a problem:
if the objects of $ C $ form a proper class $ C_ 0 $,
then there is no class of class maps $ \Map(C_0, V) $.
Indeed, on one hand, in $ \nbg $, every element of a class is itself a set,
and on the other hand, a class map $ f \colon C_0 \to V $ cannot be a set.%
\footnote{Worse still, the \enquote{very large}
	category of classes is not cartesian closed,
	so there's no hope of defining
	$ \Map(C_0, V) $ by means of some other artifice.}

\begin{nul}
	If $ C $ is a small category, then 
	the large-category $ \tau_0\PP^V(C) $ is locally small,
	and it enjoys many of the same good properties
	enjoyed by $ \Set^V $ itself.
	For every regular cardinal $ \kappa $,
	it is $ \kappa $-presentable,
	and it is \defn{cartesian closed}:
	for every pair of presheaves $ X, Y \colon C^{\op} \to \Set^V $,
	the morphisms $ X \to Y $
	form a presheaf $ \MOR(X, Y) \colon D^{\op} \to \Set^V $.
	The category $ \tau_0\PP^V(C) $ is a \defn{1-topos}.

	Similarly, the category $ \PP^V(C) $ of presheaves
	$ C^{\op} \to \An^V $
	is a $ \kappa $-presentable topos
	for every regular cardinal $ \kappa $.
\end{nul}

\begin{eg}
	Let $ C $ be a locally small category.
	If $ Y \in C $ is an object,
	then $ \yo_Y \colon C^{\op} \to \Set^V $ is
	the presheaf $ X \mapsto \Map_C(X,Y) $ represented by $ Y $.

	Dually, if $ X \in C $ is an objects,
	then $ \yo^X \colon C \to \Set^V $ is the functor
	$ Y \mapsto \Map_C(X,Y) $ corepresented by $ X $.
\end{eg}

\begin{definition}
	Let $ C $ be a locally small large-category.
	A \defn{small presheaf} of sets on $ C $ is
	a functor $ C^{\op} \to \Set^V $
	that is left Kan extended from its restriction
	to some small full subcategory $ D \subseteq C $.
	We write $ \tau_0\PP^V(C) $
	for the locally small large-category
	of small presheaves of sets.
	
	Similarly, a \defn{small presheaf} (of animae) is
	a functor $ C^{\op} \to \An^V $
	that is left Kan extended from its restriction to
	some small full subcategory $ D \subseteq C $.
	We write $ \PP^V(C) $
	for the locally small large-category
	of small presheaves.
\end{definition}

\begin{eg}
	Of course if $ C $ is a small category, then
	every presheaf on $ C $ is small.
	Thus the notation above does not conflict with the one established in
	Notation \ref{not:Vsetsanimaepresheavescategories}.
\end{eg}

\begin{nul}
	Let $ C $ be a locally small large-category.
	For any small full subcategory $ D \subseteq C $,
	we may contemplate the large-category $ \PP^V(D) $
	of presheaves $ D^{\op} \to \An^V $.
	If we have an inclusion of full subcategories
	$ D' \subseteq D \subset C $, then
	left Kan extension identifies $ \PP^V(D') $ with
	a full subcategory of $ \PP^V(D) $.
	
	The (class-indexed) filtered union
	$ \bigcup_D \PP^V(D) $
	over the class of small full subcategories of $ C $
	is precisely the large-category $ \PP^V(C) $.
	
	The categories $ \tau_0\PP^V(C) $ and $ \PP^V(C) $ are thus
	locally presentable large-categories.
\end{nul}

\begin{eg}
	Let $ C $ be a locally small large-category.
	For any object $ Y \in C $,
	the representable presheaf $ \yo_Y $ is left Kan extended
	from any full subcategory that contains $ Y $.
	In particular, $ \yo_Y $ is small.

	Thus the assignment $ Y \mapsto \yo_Y $ is
	the fully faithful \defn{Yoneda embedding}
	\[ \yo \colon C \inclusion \PP^V(C) \period \]
\end{eg}

\begin{eg}
	Let $ C $ be a locally small large-category.
	If $ C^{\op} $ is accessible,
	then $ \tau_0\PP^V(C) $ and $ \PP^V(C) $
	are the categories of accessible functors
	$ C^{\op} \to \Set^V $ and $ C^{\op} \to \An^V $, respectively.
\end{eg}

\begin{nul}
	Let $ C $ be a locally small large-category.
	The categories $ \tau_0\PP^V(C) $ and $ \PP^V(C) $
	may not enjoy all the same good features that
	$ \Set^V $ and $ \An^V$ have.
	The categories $ \tau_0\PP^V(C) $ and $ \PP^V(C) $
	possess all colimits,
	but they do not generally have all limits.
	For example, if $C$ has no nonidentity arrows, then
	there is no terminal object in $ \tau_0\PP^V(C) $.
	
	If $ C^{\op} $ is accessible or small,
	then $ \tau_0\PP^V(C) $ and $ \PP^V(C) $ do have all limits.
\end{nul}

\begin{definition}
	Let $ A $ be a class of categories.
	Let $ C $ be a locally small large-category,
	and let $ C' \subseteq C $ be a full subcategory.
	Then we say that
	\defn{$ C' $ generates $ C $ freely under $ A $-shaped colimits}
	if and only if,
	for every large-category $ D $
	that has all $ A $-shaped colimits,
	the following assertions obtain.
	\begin{enumerate}
		\item Every functor $ C' \to D $ extends
			to a functor $ C \to D $
			that preserves $ A $ -shaped colimits.
		\item For every pair of functors
			$ F, G \colon C \to D $
			that preserve $ A $-shaped colimits,
			the map
			$ \Map(F, G) \to \Map(F|C', G|C' ) $
			is an equivalence.
	\end{enumerate}
	If $ f \colon C'' \inclusion C $ is a fully faithful functor,
	then we will say that
	\defn{$ f $ generates $ C $ freely under $ A $-shaped colimits}
	if and only if its image $ f(C'') \subseteq C $ does so.
\end{definition}

\begin{remark}
	If $ C $ is not small, then in $ \nbg $
	we can make sense neither of $ \Fun(C,D) $,
	nor of the full subcategory
	$ \Fun^{A}(C, D) \subseteq \Fun(C,D) $
	consisting of those functors that preserve
	$ A $-shaped colimits.
	If however we are in a situation in which
	these objects \emph{can} be made sensible,
	then $ C' $ generates $ C $ freely under
	$ A $-shaped colimits
	if and only if the restriction induces an equivalence
	\[ \Fun^A(C, D) \equivalence \Fun(C', D) \period \]
\end{remark}

\begin{proposition}%
	\label{prp:PPsmisfreelygenerated}
	Let $ C $ be a locally small large-category.
	Then the Yoneda embedding
	$ \yo \colon C \inclusion \PP^V(C) $
	generates $ \PP^V(C) $ freely under all colimits.
\end{proposition}

The theory of small presheaves can be relativized
to a regular cardinal $ \kappa $:

\begin{definition}
	Let $ \kappa $ be a regular cardinal.
	Let $ C $ be a locally $ \kappa $-small large-category.
	A \defn{$ \kappa $-small presheaf} of sets on $ C $
	is a functor $ C^{\op} \to \Set^{\kappa} $
	that is left Kan extended from its restriction to
	some $ \kappa $-small full subcategory $ D \subseteq C $.
	The large-category of $ \kappa $-small presheaves of sets
	will be denoted $ \tau_0\PP^{\kappa}(C) $.

	Similarly, a \defn{$ \kappa $-small presheaf}
	(of animae) is a functor $ C^{\op} \to \An^{\kappa} $
	that is left Kan extended from its restriction to 
	some $ \kappa $-small full subcategory $ D \subseteq C $.
	The large-category of $ \kappa $-small presheaves
	will be denoted $ \PP^{\kappa}(C) $.
\end{definition}

\begin{nul}
	If $ C $ is small, then so is $ \PP^{\kappa}(C) $.
\end{nul}

\begin{nul}
	Since we have assumed that $ C $ is locally $ \kappa $-small,
	it follows that the Yoneda embedding lands in $ \PP^{\kappa}(C) $.
	We can therefore characterize $ \PP^{\kappa}(C) $
	as the smallest full large-subcategory of $ \PP^V(C) $
	containing $ \yo(C) $ and closed under $ \kappa $-small colimits.
\end{nul}

\begin{proposition}%
	\label{prp:PPkappaisfreelygenerated}
	Let $ \kappa $ be a regular cardinal.
	Let $ C $ be a locally $\kappa$-small large-category.
	Then the Yoneda embedding
	$ \yo \colon C \inclusion \PP^{\kappa}(C) $
	generates $ \PP^{\kappa}(C) $ freely under
	$ \kappa $-small colimits.
\end{proposition}

\begin{nul}
	The category $ \PP^{\kappa}(C) $
	has all $ \kappa $-small colimits, but
	in general, it does not have $ \kappa $-small limits, and
	it is not cartesian closed.
	To ensure these properties as well,
	we must turn to a discussion of inaccessible cardinals.
\end{nul}


