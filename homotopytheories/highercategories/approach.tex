%!TEX root = book.tex 
% chktex-file 1
% chktex-file 3
% chktex-file 8
% chktex-file 12
% chktex-file 18
% chktex-file 24
% chktex-file 35 
% chktex-file 42

%\section*{Our approach}%
%\label{sec:approach}
%\addcontentsline{toc}{subsection}{Our approach}

\minidiv

\noindent Frustratingly, we can't yet give a thorough and precise account of higher category theory in the same terms that experienced practitioners use it.
Instead, we have to construct a model of higher categories within established set-theoretic foundations.
We will then work within that model to develop a slate of fundamental definitions, constructions, and theorems.
Once enough of this development is complete,
the corresponding higher category of higher categories is then unique.
At that point, we are free to throw our ladder away, to ignore set-level specifics of the chosen model, and to work contentedly in a natively higher-categorical way.

Accordingly, this text is divided into two parts.
The first part climbs the ladder by developing the theory of quasicategories as a model of $(\infty,1)$-categories.
The second part throws the ladder away by treating these objects with minimal reference to the explicit definitions.
