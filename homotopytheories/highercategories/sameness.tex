%!TEX root = book.tex 
% chktex-file 1
% chktex-file 3
% chktex-file 8
% chktex-file 12
% chktex-file 18
% chktex-file 24
% chktex-file 35 
% chktex-file 42

%\section*{The question of sameness \inc}%
%\label{sec:sameness}%
%\addcontentsline{toc}{subsection}{The question of sameness \inc}

\minidiv

\noindent \emph{When are two mathematical objects the same?}
Let's start with natural numbers $a$ and $b$.
Perhaps you want to prove that $a$ and $b$ are the same.
\emph{How might you go about that?}
Well, it depends upon how $a$ and $b$ arose in your thinking.
Since they are natural numbers, it's likely you found them by counting something.
In other words, $a$ is the cardinality of a finite set $A$, and
$b$ is the cardinality of a finite set $B$.

Perhaps $A$ is the set of nontrivial ways of (correctly and nonredundantly) parenthesizing an expression like $abcd$ with two pairs of matching parentheses,
such as $(a(bc))d$.
Perhaps $B$ is the set of full binary trees with $4$ leaves.
It would be perfectly valid to prove that $a=b$ by computing each of these numbers and comparing the answers, but
this sort of proof is unsatisfying, even contentless:
such a proof doesn't actually inform us of anything about \emph{why} these two numbers are equal.
The equation $a=b$ is a shadow of something more interesting (and more general):
a \emph{bijection} between the sets $A$ and $B$.

Whereas \enquote{$a=b$} is a \emph{property}, a bijection between $A$ and $B$ is \emph{structure}.
A bijection $\phi \colon A \to B$ is an isomorphism in the category of sets;
its inverse is map $\psi \colon B \to A$ such that $\psi\phi = \id_A$ and $\phi\psi = \id_B$.

If we now want to operate in a $2$-category, these equalities are no longer a natural thing to demand.
For example, we know we should not expect two categories to be isomorphic.
The correct notion of sameness for categories (or any pair of objects in a $2$-category is \emph{equivalence}.
An equivalence $\phi \colon A \to B$ has an inverse $\psi \colon B \to A$ in the sense that $\psi\phi$ is isomorphic to $\id_A$ and $\phi\psi$ is isomorphic to $\id_B$.
That means that there are $2$-morphisms
\[
  \alpha \colon \id_A \to \psi\phi \comma \qquad \beta \colon \psi\phi \to \id_A 
\]
such that $\beta\alpha = \id_{\id_A}$ and $\alpha\beta = \id_{\psi\phi}$,
and there are $2$-morphisms
\[
  \gamma \colon \id_B \to \phi\psi \comma \qquad \delta \colon \phi\psi \to \id_B 
\]
such that $\delta\gamma = \id_{\id_B}$ and $\gamma\delta = \id_{\phi\psi}$.



