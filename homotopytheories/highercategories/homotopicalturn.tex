%!TEX root = book.tex 
% chktex-file 1
% chktex-file 3
% chktex-file 8
% chktex-file 12
% chktex-file 18
% chktex-file 24
% chktex-file 35 
% chktex-file 42

%\section*{The homotopical turn}%
%\label{sec:homotopical}
%\addcontentsline{toc}{subsection}{The homotopical turn}

\minidiv

\noindent The mission of homotopy theory is to iteratively enhance every \emph{property} of a mathematical object to \emph{structure} on it.
The homotopy theorist deconstructs the equals sign by this process:
they no longer regard \enquote{$x = y$} as a property that $x$ and $y$ together possess,
but rather as piece of structure that connects $x$ and $y$.
That structure is then a \emph{path} between $x$ and $y$.
Semantically, we consider this path as a \enquote{reason} for -- or as a \enquote{witness} to -- the equality $x = y$.

If $\alpha$ and $\beta$ are paths connecting $x$ and $y$,
then again we do not wish to speak of \enquote{$\alpha = \beta$} as a property,
but as a further piece of structure -- a \emph{homotopy} between $\alpha$ and $\beta$.
We iterate: two homotopies are no longer merely \enquote{equal}, but they may be connected by \emph{higher homotopies};
two higher homotopies may be connected by further higher homotopies, \emph{etc}., \emph{etc}., \emph{ad infinitum}.

The data of all these points and paths and homotopies and higher homotopies, taken together, constitute an \emph{anima} (pl. \emph{animae}).
Animae are also called \emph{spaces}, \emph{homotopy types}, \emph{Kan complexes}, or \emph{$\infty$-groupoids}.
These terms each reflect a certain attitude toward these objects.
Terms like \enquote{space} and \enquote{homotopy type} acknowledge that these objects were first modelled and understood using topological spaces and topological notions of homotopy.
A \enquote{Kan complex} (named for Dan Kan) is then a combinatorial blueprint for these homotopies and their relations.
The phrase \enquote{$\infty$-groupoid} then reinterprets the (higher) homotopies as (higher) \emph{symmetries}.
The fact that these terms can all be used interchangeably is a nontrivial insight -- Grothendieck's \emph{homotopy hypothesis}.
We will formulate and prove a version of this sentence in this chapter -- surely not the version Grothendieck had in mind,
but one that is better-adapted to the needs of contemporary mathematicians.
In our formulation, it becomes a theorem of Kan.

Our use of the term \emph{anima} reflects our desire not to favor any one of these attitudes.
Animae play the same role in \emph{homotopical mathematics} that sets play in `ordinary' mathematics:
virtually all objects in homotopical mathematics are described in terms of animae.

Symmetries of objects were certainly central to the mathematics of the 19th century, but
it's a distinctly 20th century notion that symmetries might meaningfully have their own symmetries.
(It is interesting to reflect on the origins of this idea, but
it would be difficult to pinpoint the first person who seriously considered this possibility.)
In any case, the 20th century provided three realizations about homotopy theory's iterative enhancement process.

First was the promise of interesting new phenomena to study.
Homological algebra appears at first to be a relatively featureless outgrowth of linear algebra.
The Bockstein homomorphism is an early illustration that ordinary modules interact in new ways in derived settings.
But it's the nontriviality of the Hopf element $\eta$ in the stable homotopy group $\pi^s_1$ that decisively separates \emph{homological algebra} -- ordinary algebra that is then derived -- from \emph{homotopical algebra} -- algebra done in a natively homotopical setting.
The first signal from the mysterious world of homotopical algebra was a short message: \enquote{$\eta \neq 0$}.

In spite of our vague description of homotopy theory as an inductive enhancement of properties into structure,
this can all be made precise.
In fact, it can be made precise in different ways.
On one hand, we can model homotopical structures entirely via ordinary mathematical objects.
We'll do that here: we'll adopt the Kanian approach and encode homotopy types as \emph{simplicial sets}.
Alternatively -- and in a spirit closer to that of this introduction -- one might instead attempt to rewrite the logical foundations of mathematics in a way that bakes in our preference for structure over properties.
This is the approach of \emph{homotopy type theory}, a stirring vision of new foundations in mathematics.
As of this writing, homotopy type theory is still in its infancy.
In later editions of this book, perhaps the combinatorics of simplices will be replaced by fundamental facts about type theory.

Once it was understood how to model homotopical structures accurately, new questions arose:
\emph{how do we construct models that are maximally useful?}
\emph{what does it mean to say that two models represent the same homotopy theory?}
\emph{shouldn't this notion of sameness, whatever it is, be subject to the same inductive refinement process that got us here?}
These questions lay at the heart of the many foundational developments in homotopy theory starting in the 1970s.
By the end of the millennium it was clear that one would need to take a further step, and contemplate a \emph{homotopy theory of homotopy theories}.
Just as Grothendieck had seen that ordinary homotopy theory should be equivalent to the theory of $\infty$-groupoids,
people like Joyal, Kan, Rezk, Simpson, and Toen recognized that the homotopy theory of homotopy theories should be equivalent to the theory of \emph{$\infty$-categories}.


