%!TEX root = book.tex 
% chktex-file 1
% chktex-file 3
% chktex-file 8
% chktex-file 12
% chktex-file 18
% chktex-file 24
% chktex-file 35 
% chktex-file 42

\section{Regularity \& smallness}%
\label{sec:regularityandsmallness}
\addcontentsline{toc}{section}{Regularity \& smallness}

\begin{definition}
	If $ \kappa $ is a cardinal,
	then a set $ S $ is \defn{$ \kappa $-small}
	if and only if $ |S| < \kappa $.
	We shall write $ \Set^{\kappa} \subset \Set^V$
	for the category of $ \kappa $-small sets,
	Thus $ \Set^V $ is the filtered union of the categories $ \Set^{\kappa} $
	over the proper class of regular cardinals.
	
	A cardinal $ \kappa $ is \defn{regular}
	if and only if, for every map $ f \colon S \to T $
	in which $ T $ and every fiber $ f^{-1}\{t\} $ are all $ \kappa $-small,
	the set $ S $ is $ \kappa $-small as well.
	Equivalently, $ \kappa $ is regular if and only if
	$ \Set^{\kappa} \subset \Set^V $ is stable under colimits indexed by $ \kappa $-small posets.
\end{definition}

\begin{eg}
	Under this definition, $ 0 $ is a regular cardinal.%
	\footnote{Many texts require that a regular cardinal be infinite.}
	There are no $ 0 $-small sets.
\end{eg}

\begin{eg}
	The $ \aleph $ family of cardinals is defined by
	a function from the class of ordinal numbers
	to the class of cardinal numbers, by transfinite induction:
	\begin{enumerate}
		\item The cardinal $ \aleph_0 $ is the ordinal number
			$ \omega $ consisting of all finite ordinals.
		\item For any ordinal $ \alpha $,
			one defines $ \aleph_{\alpha + 1} $ to be
			the smallest cardinal number strictly greater than $ \aleph_{\alpha} $.
		\item For any limit ordinal $ \alpha $,
			one defines $ \aleph_{\alpha} \coloneq \sup \left\{ \aleph_{\beta} : \beta < \alpha \right\} $.
	\end{enumerate}

	The countable cardinal $ \aleph_0 $ is regular.
	Every infinite successor cardinal is regular;
	consequently, $ \aleph_n $ for $ n \in \NN $ is regular as well.
	The cardinal $ \aleph_{\omega} $ is
	the smallest infinite cardinal that is not regular.
\end{eg}

\begin{eg}
	A set is $ \aleph_0 $-small if and only if it is finite.
\end{eg}

\begin{nul}
	If $ \kappa $ is a regular cardinal,
	then $ \Set^{\kappa} \subset \Set^V $ is the full subcategory
	generated by the singleton $ \{ 0 \}$
	under colimits over $ \kappa $-small posets.
\end{nul}

\begin{definition}
	If $ \kappa $ is a regular cardinal,
	then we shall write $ \An^{\kappa} \subset \An^V $
	for the full subcategory generated by $ \{ 0 \} $
	under colimits over $ \kappa $-small posets.
	The objects of $ \An^{\kappa} $ will be called
	\defn{$ \kappa $-small animae}.

	Similarly, we shall write $ \Cat^{\kappa} \subset \Cat^V $
	for the full subcategory generated 
	by $ \{ 0 \} $ and $ \{ 0 < 1 \} $
	under colimits over $ \kappa $-small posets.
	The objects of $ \Cat^{\kappa} $ will be called
	\defn{$ \kappa $-small categories}.

	Finally, a large-category $ C $
	is said to be \defn{locally $ \lambda $-small}
	if and only if,
	for every $ \lambda $-small subset $ C'_0 \subseteq C_0 $
	of objects of $ C $, the full subcategory $ C' \subseteq C $
	that it spans is $ \lambda $-small.
\end{definition}

\begin{eg}
	This turn of phrase above is slightly ambiguous when $ \kappa = 0 $.
	In that case, we take the phrase
	\enquote{subcategory generated by \ldots\ under colimits
	over the empty collection of posets}
	to mean the empty category.
	With this convention, there are no $ 0 $-small animae or categories:
	\[ \Set^0 = \An^0 = \Cat^0 = \varnothing \period \]
\end{eg}

\begin{eg}
	An anima is $ \aleph_0 $-small if and only if
	it is weak homotopy equivalent to a simplicial set
	with only finitely many nondegenerate simplices.

	A category $ C $ is $ \aleph_0 $-small if and only if
	it is Joyal equivalent to a simplicial set
	with only finitely many nondegenerate simplices. 
\end{eg}

\begin{eg}
	Why does regularity arise so often in category theory?
	What role does this hypothesis play?
	Here is the sort of scenario that is often
	lurking in the background when we appeal to the
	regularity of a cardinal.

	Let $ C $ be a large-category.
	Suppose that we have a \emph{diagram of diagrams} in $ C $,
	in the following sense.
	We have a category $ A $;
	a functor $ B \colon A \to \Cat^V $;
	and for each $ \alpha \in A $,
	a functor $ X_{\alpha} \colon B_{\alpha} \to C $.
	Furthermore, the colimits of each of these functors
	organize themselves into a functor
	$ A \to C $:
	\[
		\alpha \mapsto \colim_{\beta \in B_{\alpha}}
		X_{\alpha}(\beta) \period
	\]
	We will often be in situations in which
	we need to analyze the \emph{colimit of colimits}:
	\[
		\colim_{\alpha \in A}
		\colim_{\beta \in B_{\alpha}} X_{\alpha}(\beta) \period
	\]
	In this case, we may reorganize these data.
	We first construct the cocartesian fibration
	corresponding to the functor $ B $,
	which we will abusively write $ B \to A $,
	since the fibers are the categories $ B_{\alpha} $.
	We now have a single functor $ X \colon B \to C $
	whose restriction to any fiber $ B_{\alpha} $
	is the functor $ X_{\alpha} $.
	Now the colimit of colimits above is a single colimit:
	\[
		\colim_{\alpha \in A}
		\colim_{\beta \in B_{\alpha}} X_{\alpha}(\beta) \simeq
		\colim_{\gamma \in B} X(\gamma) \period
	\]

	Now let $ \kappa $ be a cardinal.
	If $ A $ is $ \kappa $-small,
	and if each category $ B_{\alpha} $ is $ \kappa $-small,
	then what can we conclude about $ B $?
	In general, nothing.
	However, if $ \kappa $ is a regular cardinal,
	then $ B $ is also $ \kappa $-small.

	The motto here, then, is that
	\emph{%
		if $ \kappa $ is regular,
		then $ \kappa $-small colimits
		of $ \kappa $-small colimits
		are $ \kappa $-small colimits.
	}
\end{eg}


