%!TEX root = book.tex 
% chktex-file 1
% chktex-file 3
% chktex-file 8
% chktex-file 12
% chktex-file 18
% chktex-file 24
% chktex-file 35 
% chktex-file 42

\section{Strong limit \& inaccessible cardinals}%
\label{sec:stronglimitandinaccessiblecardinals}
\addcontentsline{toc}{section}{Strong limit \& inaccessible cardinals}

\begin{definition}
	One says that $ \kappa $ is a \defn{weak limit cardinal} if and only if,
	for every cardinal $ \xi $,
	if $ \xi < \kappa $, then $ \xi^+ < \kappa $.

	A cardinal $\kappa$ is said to be a \defn{strong limit cardinal} if and only if,
	for every cardinal $ \xi $,
	if $ \xi < \kappa $, then $ 2^{\xi} < \kappa $ as well.
	Equivalently, $ \kappa $ is a strong limit cardinal if and only if,
	for every pair of $ \kappa $-small sets $ X $ and $ Y $,
	the set $ \Map(X, Y) $ of maps $ X \to Y $ is $ \kappa $-small as well. 

	One says that $\kappa$ is \defn{weakly inaccessible} if and only if
	it is a regular, uncountable, weak limit cardinal.

	One says that $\kappa$ is \defn{inaccessible}%
	\footnote{Some authors say \defn{strongly inaccessible}
	instead of \defn{inaccessible}.}
	if and only if it is a regular, uncountable,%
	\footnote{We include the condition of uncountability only for convenience.
		It is not unreasonable to regard $ 0 $ and $ \aleph_0 $
		as inaccessible as well.}
	strong limit cardinal.
	Equivalently, an uncountable cardinal $ \kappa $ is inaccessible if and only if
	$ \Set^{\kappa} $ has all $ \kappa $-small colimits and is cartesian closed.
	Equivalently again, an uncountable cardinal $ \kappa $ is inaccessible if and only if 
	$ \Set^{\kappa} $ has all $ \kappa $-small colimits and all $ \kappa $-small colimits.
\end{definition}

The Generalized Continuum Hypothesis ($ \gch $) is equivalent to the statement that
the classes of strong and weak limit cardinals coincide,
and similarly the classes of inaccessible and weakly inaccessible cardinal coincide.

\begin{eg}
	A cardinal $ \kappa $ is a weak limit cardinal if and only if,
	for some limit ordinal $ \alpha $, one has $ \kappa = \aleph_{\alpha} $.
\end{eg}

\begin{eg}
	The $\beth$ family of cardinals is defined by a function from the class of ordinal numbers to the class of cardinal numbers.
	It's defined by transfinite induction:
	\begin{enumerate}
		\item By definition, $ \beth_0 = \aleph_0 $.
		\item For any ordinal $ \alpha $, one defines $ \beth_{\alpha+1} \coloneq 2^{\beth_{\alpha}} $.
		\item For any limit ordinal $ \alpha $,
			one defines $ \beth_{\alpha} \coloneq \sup \left\{ \beth_{\beta} : \beta < \alpha \right\} $.
	\end{enumerate}
	The cardinal $ \beth_{\alpha} $ is the cardinality of $ V_{\omega+\alpha} $.

	The Generalized Continuum Hypothesis ($ \gch $) is equivalent to the statement that
	$ \aleph_{\alpha} = \beth_{\alpha} $ for each ordinal $ \alpha $,	
	
	A cardinal $ \kappa $ is a strong limit cardinal if and only if,
	for some limit ordinal $ \alpha $, one has $ \kappa = \beth_{\alpha} $.

	The cardinal $ \beth_{\omega} $ is
	the smallest uncountable strong limit cardinal.
	It is not inaccessible, however,
	because it is not regular.

	An inaccessible cardinal $ \kappa $ is
	a \defn{$\beth$-fixed point}:
	that is, $ \beth_{\kappa} = \kappa $.
\end{eg}

\begin{nul}
	A regular uncountable cardinal $ \kappa $
	is inaccessible if and only if
	one has $ \kappa \ll \kappa $.
\end{nul}

\begin{definition}[\protect{\citeauthor[Exposé I, \S 0 and Appendix]{SGA4-1}}]%
\label{dfn:uni}
	An uncountable set $ U $ is a \defn{Grothendieck universe} if it satisfies the following conditions.
	\begin{enumerate}
		\item The set $ U $ is \defn{transitive}:
			if $ X \in Y \in U $, then $ X \in U $ as well.
		\item If $ X, Y \in U $, then $ \{X,Y\} \in U $ as well.
		\item If $ X \in U $, then the powerset $ P(X) \in U $ as well.
		\item If $ A \in U $ and $ X \colon A \to U $ is a map, then
		\[ \bigcup_{\alpha\in A}X(\alpha) \in U \]
		as well.
	\end{enumerate}
\end{definition}

Grothendieck universes are essentially the same thing as inaccessible cardinals.
This was effectively proved by \cite{Tarski1938}.
See also Bourbaki, \citeauthor[Exposé I, Appendix]{SGA4-1}.
\begin{proposition}
	If $\kappa$ is an inaccessible cardinal, then
	the set $V_{\kappa}$ of all sets of rank less than $\kappa$ is
	a Grothendieck universe of rank and cardinality $\kappa$.

	If $ U $ is a Grothendieck universe, then
	there exists an inaccessible cardinal $ \kappa $ such that $ U = V_{\kappa} $.
\end{proposition}

\begin{theorem}
	If $ \kappa $ is an inaccessible cardinal,
	then $V_{\kappa} \models \zfc $, and
	$ V_{\kappa+1} \models \nbg $.
	
	Assuming that $ \zfc $ (respectively, $ \nbg $) is consistent, then
	the existence of inaccessible cardinals is not provable
	by methods formalizable in $ \zfc $ (resp., $ \nbg $).
\end{theorem}

\begin{axiom}%
\label{axm:AU}
	The \defn{Axiom of Universes} ($ \au $) is the assertion that
	every cardinal is dominated by an inaccessible cardinal,
	or, equivalently, every set is an element of some Grothendieck universe.
	\defn{Tarski--Grothendieck set theory} is the schema $\tg = \nbg+\au$.
\end{axiom}

Under $ \au $, the proper class of inaccessible cardinals
can be well ordered.
It will be helpful for us to have a notation for this.

\begin{definition}
	Assume $ \au $.
	Let us define the $ \daleth $ family of cardinals
	as a function from the class of ordinal numbers
	to the class of cardinal numbers:
	\begin{enumerate}
		\item By definition, $ \daleth_0 = \aleph_0 $.
		\item For any ordinal $ \alpha $,
			one defines $ \daleth_{\alpha+1} $ as
			the smallest inaccessible cardinal 
			greater than $ \daleth_{\alpha} $.
		\item For any limit ordinal $ \alpha $,
			one defines
			$ \daleth_{\alpha} \coloneq \sup \left\{ \daleth_{\beta} :
			\beta < \alpha \right\} $.
	\end{enumerate}
	Thus $ \daleth_0 = \aleph_0 $, and
	for $ \alpha \geq 1$ an ordinal,
	$ \daleth_{\alpha} $ is the
	\enquote{$ \alpha $-th inaccessible cardinal}.
\end{definition}


