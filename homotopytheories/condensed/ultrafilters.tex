%!TEX root = book.tex 
% chktex-file 1
% chktex-file 3
% chktex-file 8
% chktex-file 12
% chktex-file 18
% chktex-file 24
% chktex-file 35 
% chktex-file 42

%-------------------------------------------------------------------%
%-------------------------------------------------------------------%
\section{Ultrafilters}%
\label{sec:ultrafilters}
%-------------------------------------------------------------------%
%-------------------------------------------------------------------%

%-------------------------------------------------------------------%
\subsection{Ultrafilters on sets}%
\label{sub:ultrafilters_on_sets}
%-------------------------------------------------------------------%

\begin{definition}%
	\label{dfn:ultrafilter}
	Let $ S $ be a set,
	and consider the functor
	$ \yo^S \colon \Set^V \to Set^V $
	corepresented by $ S $.
	Let $ i \colon \Set_0 \inclusion \Set^V $
	denote the inclusion of the full subcategory 
	of finite sets.

	An \defn{ultrafilter} on $ S $ is
	a natural transformation
	\[ \yo^S \circ i \to i \period \]
\end{definition}

\begin{notation}
	Let $ S $ be a set, and
	let $ \mu $ be an ultrafilter on $ S $.
	We find it expressive to write
	\[ \int_S (\cdot) \ d\mu \colon \yo^S \circ i \to i \]
	for the natural transformation.
	If $ I $ is a finite set,
	and if $ f \colon S \to I $ is a map,
	then this natural transformation
	specifies an element
	\[ \int_S f \ d\mu \in I \period \]
	The naturality condition ensures that
	if $ \phi \colon I \to J $ is a map of finite sets,
	then
	\[
		\phi \left( \int_S f \ d\mu \right) = 
		\int_S \phi \circ f \ d\mu \period
	\]
\end{notation}

\begin{eg}
	Let $ S $ be a set, and let $ s \in S $ be an element.
	The \defn{principal ultrafilter} $ \delta_s $ is then defined so that
	\[
		\int_S f \ d \delta_s = f(s) \period
	\]
	By Yoneda,
	every ultrafilter on a finite set is principal,
	but as we shall see,
	infinite sets have ultrafilters that are not principal.
\end{eg}

\begin{notation}
	If $ S $ is a set, then
	we write $ \beta(S) $ for the set
	of ultrafilters on $ S $:
	\[ \beta(S) \coloneq \Map(\yo^S \circ i, i) \period \]

	If $ \alpha $ is an ordinal number such that
	$ S \in \Set_{\alpha} $, then $ \yo^S $
	takes values in $ \Set_{\alpha} $,
	so $ \beta(S) \in \Set_{\alpha} $ as well.
	Thus $ \beta $ defines a functor
	$ \Set_{\alpha} \to \Set_{\alpha} $.

	The formation of the principal ultrafilter
	defines a natural transformation
	$ \delta \colon \id \to \beta $.
\end{notation}

\begin{nul}
	Let $ \phi \colon S \to T $ be a map,
	and let $ \mu $ be an ultrafilter on $ S $.
	The induced ultrafilter $ \phi_{\ast}\mu $ on $ T $
	is then given by
	\[
		\int_T f \ d (\phi_{\ast}\mu) =
		\int_S (f \circ \phi) \ d \mu \period
	\]
\end{nul}

\begin{nul}
	Let $ S $ be a set.
	We can express $ \beta(S) $ as an end:
	\[
		\beta(S) =
		\int_{I \in \Set_0} \Map(\Map(S,I),I) \period
	\]
	Hence for an ordinal number $ \alpha $,
	the functor $ \beta \colon \Set_{\alpha} \to \Set_{\alpha} $
	is the right Kan extension of
	$ i \colon \Set_0 \inclusion \Set_{\alpha} $
	along itself.
	Equivalently, we can express $ \beta(S) $
	as the limit over all finite sets
	to which $ S $ maps:
	\[ \beta(S) = \lim_{I \in (\Set_0)_{S/}} I \period \]
\end{nul}

Thus far, there is nothing to guarantee that we aren't
simply speaking about the identity functor.
We have not laid eyes an example of an ultrafilter
that is not principal.
It turns out that infinite sets have plenty of
non-principal ultrafilters.
To prove the existence of these,
let us look at a more conventional way
of describing ultrafilters.

The connection between
the definition of ultrafilters in \ref{dfn:ultrafilter} and
their more conventional definition
is a primitive variant of
the Riesz−Markov−Kakutani Representation Theorem.
If an ultrafilter on a set $ S $ is
a functional $ f \mapsto \int_S f \ d \mu $,
then we can look for the \emph{measure} $ \mu $ on $ S $
relative to which this functional is the integral.
As in the functional analysis setting,
one connects these two perspectives through
characteristic functions.

\begin{definition}
	Let $ S $ be a set, and
	let $ T \subseteq S$ be a subset.
	The \defn{characteristic map}
	$ \chi_T \colon S \to \{ 0,1 \}$
	is defined by the formula
	\[
		\chi_T(s) =
		\begin{cases}
			1 & \text{if } s \in T \semicolon \\
			0 & \text{if } s \notin T \period
		\end{cases}
	\]
	
	Now let $ \mu $ be an ultrafilter on $ S $.
	Let us write
	\[
		\mu(T) \coloneq
		\int_S \chi_T \ d \mu \in
		\{ 0,1 \} \period
	\]
	
	We shall say that $ T $ is \defn{$ \mu $-thick}
	if and only if $\mu(T) = 1$.
	Accordingly, we say that $ T $ is \defn{$ \mu $-thin}
	if and only if $ \mu(T) = 0 $.
	We let $ \mathscr{F}_{\mu} \subseteq P(S) $ denote
	the subset consisting of all $ \mu $-thick subsets of $ S $.
\end{definition}

\begin{eg}
	For an element $ s \in S$,
	the principal ultrafilter $ \delta_s $ is
	the unique ultrafilter relative to which
	the singleton $ \{ s \} $ is thick.
	That is, $ \{ s \} \in \mathscr{F}_{\mu} $
	if and only if $ \mu = \delta_s $.
\end{eg}

\begin{eg}
	Let $ \phi \colon S \to T $ be a map,
	let $ \mu \in \beta(S) $ be an ultrafilter, and
	let $ U \subseteq T $ be a subset.
	One has
	\[ (\phi_{\ast} \mu)(U) = \mu (\phi^{-1}(U)) \period \]
	Hence $ U $ is $ \phi_{\ast} \mu $-thick
	if and only if $ \phi^{-1} U $ is $ \mu $-thick.
	In other words,
	\[
		\mathscr{F}_{\phi_{\ast} \mu} =
		(\phi^{-1})^{-1} \mathscr{F}_{\mu} \period
	\]
\end{eg}

The proof of the following is routine.

\begin{lemma}
	Let $ S $ be a set, and
	let $ \mu \in \beta(S) $ be an ultrafilter.
	\begin{enumerate}
		\item The set $ S $ is $ \mu $-thick.
		\item Supersets of $ \mu $-thick sets are $ \mu $-thick.
		\item The intersection of two $ \mu $-thick sets
			is $ \mu $-thick.
		\item The complement of a $ \mu $-thick set
			is $ \mu $-thin.
	\end{enumerate}
\end{lemma}

\begin{eg}%
	\label{eg:thicknesscriterion}
	Let $ \mu $ be an ultrafilter on $ S $.
	A subset $ T \subseteq S $ is $ \mu $-thick if and only if
	it intersects any $ \mu $-thick subset of $ S $.
	One direction follows from the fact that $ \mu $-thick subsets
	are closed under finite intersection.
	In the other direction, if $ T $ intersects
	every $ \mu $-thick subset,
	then $ S \setminus T $ cannot be $ \mu $-thick, so
	$ T $ must be.
\end{eg}

\begin{definition}
	Let $ S $ be a set.
	A \defn{finitely additive measure} on $ S $
	is a collection $ \mathscr{F} \subseteq P(S) $
	of subsets of $ S $
	(the subsets of positive measure)
	such that for every partition
	\[
		S = S_1 \coproduct \cdots \coproduct S_n \comma
	\]
	there is a unique $ i $ such that $ S_i \in \mathscr{F} $.

	Equivalently, $ \mathscr{F} \subseteq P(S) $ is
	a finitely additive measure
	if and only the following conditions obtain.
	\begin{enumerate}
		\item $ S \in \mathscr{F} $.
		\item If $ U \subseteq T \subseteq S $ and if
			$ U \in \mathscr{F} $, then
			$ T \in \mathscr{F} $.
		\item If $ T, U \in \mathscr{F} $, then
			$ T \cap U \in \mathscr{F} $.
		\item If $ T \in \mathscr{F} $, then
			$ S \setminus T \in \mathscr{F} $.
	\end{enumerate}

	Equivalently, $ \mathscr{F} \subseteq P(S) $ is
	a finitely additive measure if and only if
	$ \mathscr{F} = \mu^{-1}\{1\} $ for some map
	$ \mu \colon P(S) \to \{0, 1\} $
	satisfying the following.
	\begin{enumerate}
		\item $ \mu(S) = 1 $.
		\item For every family $ T_1, \dots, T_n $
			of pairwise disjoint subsets of $ S $,
			one has
			\[
				\mu(T_1 \coproduct
				\cdots
				\coproduct T_n) = 
				\mu(T_1) +
				\cdots
				+ \mu(T_n) \period
			\]
	\end{enumerate}
\end{definition}

\begin{construction}
	Let $ S $ be a set.
	Let $ m(S) $ be the set of
	finitely additive measures on $ S $.

	Attached to an ultrafilter $ \mu $ on $ S $ is 
	the collection $ \mathscr{F}_{\mu} $ of $ \mu $-thick subsets.
	In the other direction,
	attached to a finitely additive measure $ \mathscr{F} $ 
	is the ultrafilter $\mu_{\mathscr{F}}$ that
	carries a finite set $ I $ and a map $ f \colon S \to I $
	to the unique element $ i = \int_S f \ d \mu \in I $
	such that $ S_i \in \mathscr{F}$.
	The assignments $ \mu \mapsto \mathscr{F}_{\mu} $ and
	$ \mathscr{F} \mapsto \mu_{\mathscr{F}} $
	together define a bijection $ \beta(S) \cong m(S) $.
\end{construction}

\begin{eg}
	Let $ T \subseteq S $ be a subset.
	Then the inclusion induces an inclusion
	$ \beta(T) \inclusion \beta(S) $
	that identifies ultrafilters on $ T $ with
	ultrafilters on $ S $ relative to which
	$ T $ is thick.
\end{eg}

\begin{definition}
	Let $ S $ be a set, and let $ \mathscr{G} \subseteq P(S) $.
	We say that $ \mathscr{G} $ has the
	\defn{finite intersection property} if and only if
	no finite intersection of elements of $ \mathscr{G} $
	is empty.

	An ultrafilter $ \mu $ on $ S $ is said to be
	\defn{supported on $ \mathscr{G} $} if and only if
	every element of $ \mathscr{G} $ is $ \mu $-thick,
	that is, $ \mathscr{G} \subseteq \mathscr{F}_{\mu} $.
\end{definition}

\begin{eg}
	Let $ \mu $ be an ultrafilter on $ S $.
	Since $ \mu $-thick subsets are closed under
	finite intersections,
	if $ \mathscr{G} \subseteq P(S) $ is a finite collection,
	then $ \mu $ is supported on $ \mathscr{G} $
	if and only if it is supported on
	the intersection of the elements of $ \mathscr{G} $.
	In other words, the set
	$ \beta(T_1 \cap \cdots \cap T_n) $ is naturally identified with
	the set of ultrafilters on $ S $ that are 
	supported on $ \{T_1, \ldots, T_n\} $.

	Thus the notion of an ultrafilter supported
	on a family $ \mathscr{G} \subseteq P(S) $
	is really only interesting if $ \mathscr{G} $ is infinite.
\end{eg}

If there exists an ultrafilter supported on $ \mathscr{G} $,
then certainly $ \mathscr{G} $ has the finite intersection property.
We now prove that the converse is true, and 
this will provide us with a good supply of ultrafilters
on an infinite set.

\begin{lemma}%
	\label{lem:generateultrafilters}
	Let $ S $ be a set, and let $ \mathscr{G} \subseteq P(S) $
	be a family that has the finite intersection property.
	Then there exists an ultrafilter on $ S $
	supported on $ \mathscr{G} $.
\end{lemma}

\begin{proof}
	Consider the families $ \mathscr{F} \subseteq P(S) $
	that contain $ \mathscr{G} $ and
	have the finite intersection property.
	By Zorn's lemma there is a maximal such family, $ \mathscr{F} $.
	We claim that $ \mathscr{F} $ is a finitely additive measure.
	
	Let $ S = S_1 \coproduct \cdots \coproduct S_n $
	be a finite partition of $ S $.
	The finite intersection property ensures that
	at most one of the summands $ S_i $ lies in $ \mathscr{F} $.
	
	Now suppose that
	none of the summands $ S_i $ lies in $ \mathscr{F} $.
	Consider, for each $ i $,
	the family $ \mathscr{F} \cup \{ S_i \} \subseteq P(S) $;
	the maximality of $ \mathscr{F} $ implies that
	this family fails to have the finite intersection property. 
	Thus for each $ i $, there is a finite intersection
	$ T_i = T_{i1} \cap \cdots \cap T_{im_i} $
	of elements of $ \mathscr{F} $ such that 
	\[
		S_i \cap T_i = \emptyset \period
	\]
	But this implies that
	\[
		T_1 \cap \cdots \cap T_n = 
		T_{11} \cap \cdots \cap T_{1m_1} \cap
		\cdots
		T_{n1} \cap \cdots \cap T_{nm_n}
		= \emptyset \comma
	\]
	which contradicts the finite intersection property
	for $ \mathscr{F} $ itself.
	Hence one of the $ S_i $ lies in $ \mathscr{F} $.

	Thus $ \mathscr{F} $ is a finitely additive measure. 
\end{proof}

\begin{eg}
	Let $ S $ be an infinite set.
	Let $ \mathscr{G} $ be the collection of
	cofinite subsets of $ S $ −
	those subsets $ T \subseteq S $ such that
	$ S \setminus T $ is finite.
	Since $ \mathscr{G} $ has the finite intersection property,
	it follows that there is an ultrafilter on $ S $
	relative to which 
	every cofinite subset is thick.
	Such an ultrafilter is necessarily nonprincipal.
\end{eg}

\begin{nul}
	It is not quite accurate to say that
	the Axiom of Choice is \emph{necessary}
	to produce nonprincipal ultrafilters, but
	it is true that their existence is
	independent of Zermelo--Fraenkel set theory.
\end{nul}

\begin{proposition}
	The functor $ \beta \colon \Set^V \to \Set^V $
	preserves finite colimits.
\end{proposition}

\begin{proof}
	If $ \mu $ is an ultrafilter on
	$ S = S_1 \sqcup \cdots \sqcup S_n $,
	then exactly one of the $ S_i $ is $ \mu $-thick.
	Thus $ \mu $ is induced by a unique ultrafilter
	on this unique summand $ S_i $.
	This proves that $ \beta $ preserves finite coproducts.

	Now let $ S $ be a set, and let $ R \parallelto S $
	be an equivalence relation.
	Applying $ \beta $, we obtain a relation
	$ \beta(R) \parallelto \beta(S) $.
	We must prove that the natural map
	$ \beta(S)/\beta(R) \to \beta(S/R) $
	is a bijection.
	It is automatically a surjection,%
	\footnote{Every surjection has a section (\ac),
		and so every functor $ \Set \to \Set $ carries
		surjections to surjections.}
	so it remains to show that it is injective.
	
	Given ultrafilters $ \mu $ and $ \nu $ on $ S $,
	let us write $ \mu R \nu $ if
	the pair $ (\mu, \nu) \in \beta(S) \times \beta(S) $
	lies in the image of $ \beta(R) $.
	Thus $ \mu R \nu $ if and only if
	there exists an ultrafilter $ \tau $ on $ R $
	such that:
	\begin{enumerate}
		\item for any $ \mu $-thick subset $ U \subseteq S $,
			the set $ (U \times S) \cap R $
			is $ \tau $-thick, and
		\item for any $ \nu $-thick subset $ V \subseteq S $,
			the set $ (S \times V) \cap R $
			is $ \tau $-thick.
	\end{enumerate}
	In other words, $ \mu R \nu $ if and only if
	there exists an ultrafilter $ \tau $ on $ R $
	supported on the family
	\[
		\mathscr{G} \coloneq
			\{ (U \times V) \cap R :
			(U \in \mathscr{F}_{\mu})
			\wedge (V \in \mathscr{F}_{\nu}) \} \period
	\]
	By Lemma \ref{lem:generateultrafilters},
	$ \mu R \nu $ if and only if
	$ \mathscr{G} $ enjoys the finite intersection property.
	Since $ \mathscr{F}_{\mu} $ and $ \mathscr{F}_{\nu} $
	are closed under finite intersection,
	so is $ \mathscr{G} $.
	So $ \mu R \nu $ if and only if
	$ \mathscr{G} $ does not contain the empty set,
	which holds if and only if,
	for every $ \mu $-thick $ U \subseteq S $ and
	every $ \nu $-thick $ V \subseteq S $,
	we have $ (U \times V) \cap R \neq \emptyset $.

	For any subset $ W \subseteq S $,
	let us write $ \overline{W} \subseteq S $
	for the saturation of $ W $
	under the equivalence relation $ R $.
	We have shown that $ \mu R \nu $ if and only if
	for every $ \mu $-thick subset $ U \subseteq S $
	and every $ \nu $-thick subset $ V \subseteq S $,
	the set $ \overline{U} \cap V \neq \emptyset $,
	or equivalently, $ U \cap \overline{V} \neq \emptyset $.

	Using the observation of Example \ref{eg:thicknesscriterion},
	we thus conclude that the following are equivalent:
	\begin{enumerate}
		\item $ \mu R \nu $;
		\item the saturation of every $ \mu $-thick subset
			is $ \nu $-thick;
		\item the saturation of every $ \nu $-thick subset
			is $ \mu $-thick;
		\item a saturated subset is $ \mu $-thick if and only if
			it is $ \nu $-thick;
		\item $ \mu $ and $ \nu $ induce the same ultrafilter
			on $ S/R $.
	\end{enumerate}
	In other words, the map $ \beta(S)/\beta(R) \to \beta(S/R) $
	is an injection.
\end{proof}

\begin{construction}
	Let $ \alpha $ be an ordinal number.
	For entirely formal reasons,
	the functor $ \beta \colon \Set_{\alpha} \to \Set_{\alpha} $
	is a monad with unit $ \delta \colon \id \to \beta $.

	To explain this, note that $ \delta $
	restricts to a natural isomorphism
	$ i^{\ast}\delta \colon i \equivalence \beta \circ i $.
	Furthermore, $ \beta $ enjoys the following universal property:
	for every functor
	$ \phi \colon \Set_{\alpha} \to \Set_{\alpha} $,
	every natural transformation
	$ \eta \colon \phi \circ i \to i $,
	extends to a unique natural transformation
	$ \etabar \colon \phi \to \beta $,
	by which we mean that
	$ i^{\ast}\etabar = i^{\ast}\delta \circ \eta $.
	In other words, restriction induces a natural bijection
	\[
		\Map(\phi, \beta) = \Map(\phi \circ i, i) \period
	\]
	
	The identification $ i^{\ast} \delta $ gives rise to an
	identification $ \beta^2 \circ i \equivalence i $,
	which in turn extends uniquely to a natural transformation
	$ \lambda \colon \beta^2 \to \beta $.

	Let's unpack this natural transformation.
	For the sake of brevity, if $ X $ and $ Y $ are sets,
	let us write $ Y^X $ for $ \Map(X,Y) $.
	Let $ X $ be a set, and let $ I $ be a finite set.
	The key operation is the evaluation map
	\[
		\epsilon_X \colon X \to I^{I^X} \period
	\]
	Of course this is the composite
	\[
		\begin{tikzcd}[sep=1.5em]
			X \arrow[r, "\delta_X"] &
			\beta(X) =
			\int_{I \in F} I^{I^X} \arrow[r, "\pr_I"] &
			I^{I^X}
		\end{tikzcd}
	\]	
	The map $ \epsilon_X $ induces a map
	\[
		\epsilon_X^{\ast} \colon I^{I^{I^X}} \to I^X \comma
	\]
	and we apply it to the set $ X = I^S $ for a set $ S $.
	This produces a natural map
	\[
		\epsilon_{I^S}^{\ast} \colon
		I^{I^{I^{I^S}}} \to I^{I^S} \period
	\]
	Explicitly, if $ f \colon I^{I^{I^S}} \to I $ is a map,
	then its image under $ \epsilon_{I^S}^{\ast} $ is
	the map $ I^S \to I $
	that carries $ \phi $ to $ f(\epsilon_S(\phi)) $.
	Now the map $ \lambda_S \colon \beta^2(S) \to \beta(S) $ 
	is the composite
	\[
		\begin{tikzcd}[sep=1.5em]
			\beta(\beta(S)) =
			\int_{J \in F} J^{J^{\int_{I \in F} I^{I^S}}}
			\arrow[r, "c"] &
			\int_{J \in F} \int_{I \in F} J^{J^{I^{I^S}}}
			\arrow[r, "p"] &
			\int_{I \in F} I^{I^{I^{I^S}}}
			\arrow[r, "\int_I \epsilon_{I^S}^{\ast}"] &
			\int_{I \in F} I^{I^S} = \beta(S) \comma
		\end{tikzcd}
	\]
	where $ c $ and $ p $ are the canonical maps.

	Now let $ \tau $ be an ultrafilter on $ \beta(S) $.
	The induced ultrafilter $ \lambda_S(\tau) $ on $ S $
	is defined so that
	\[
		\int_{S} f \ d\lambda_S(\tau) =
		\int_{\mu \in \beta(S)}
		\left( \int_{S} f \ d\mu \right) 
		d\tau \period
	\]
	Said differently, if $ T \subseteq S $ is a subset,
	then the induced map $ \beta(T) \to \beta(S) $
	identifies $ \beta(T) $ with
	the set of ultrafilters on $ S $ supported on $ \{T\} $, and
	the ultrafilter $ \lambda(\tau) $ on $ S $
	is defined so that  
	\[
		\lambda(\tau)(T) = \tau(\beta(T)) \period
	\]
\end{construction}

%-------------------------------------------------------------------%
\subsection{Completeness of ultrafilters}%
\label{sub:completeness_of_ultrafilters}
%-------------------------------------------------------------------%

%-------------------------------------------------------------------%
\subsection{Ultrafilters on posets}%
\label{sub:ultrafilters_on_posets}
%-------------------------------------------------------------------%

%-------------------------------------------------------------------%
\subsection{Ultraproducts}%
\label{sub:ultraproducts}
%-------------------------------------------------------------------%




