%!TEX root = book.tex 
% chktex-file 1
% chktex-file 3
% chktex-file 8
% chktex-file 12
% chktex-file 18
% chktex-file 24
% chktex-file 35 
% chktex-file 42

%-------------------------------------------------------------------%
%-------------------------------------------------------------------%
\section{Cardinals}%
\label{sec:cardinals}
%-------------------------------------------------------------------%
%-------------------------------------------------------------------%

Mathematicians' \enquote{stock} set theory, $\zfc$
(Zermelo--Fraenkel set theory $\zf$ plus the Axiom of Choice $\ac$)
doesn't quite have the expressive power one needs for work with categories and higher categories.
The issue ultimately comes down to Cantor's diagonal argument:
there is no surjection of a set onto its powerset.
This is ultimately why no one can contemplate a set of all sets,
and it's also the key to Freyd's observation that
if $C$ is a category and $\kappa$ is the cardinality of its set of arrows,
then $C$ has all $\kappa$-indexed products only if $C$ is a poset.
This, in turn, is what's behind the \enquote{solution set condition}
in representability theorems or the Adjoint Functor Theorem.
Hence one really must distinguish between \enquote{large} and \enquote{small} objects.

One improves matters by passing to von Neumann--Bernays--Gödel set theory ($\nbg$),
which is a conservative extension of $\zfc$.
In $ \nbg $, the formal language consists of the symbols $\in$ and $=$;
a constant $V$; suitable variables;
the usual connectives of first-order logic ($\neg$, $\wedge$, $\vee$, $\implies$, and $\iff$);
and the quantifiers $\forall$ and $\exists$.
The objects of the theory are called \defn{classes}.
A class $ x $ is called a \defn{set} if and only if $x \in V$;
a \defn{proper class} is a class $ X $ such that $ X \notin V $.
We summarize the axioms of $ \nbg $ in informal language:
\begin{description}
	\item[\textbf{Extensionality}] Classes $ X $ and $ Y $ are equal if and only if,
		for any $ z $, one has $ z \in X $ if and only if $ z \in Y $.
	\item[\textbf{Regularity}] For every class $ X $,
		there exists an element $ z \in X $ such that $ z \cap X = \varnothing $.
	\item[\textbf{Infinity}] There is an infinite set.
	\item[\textbf{Union}] If $ x $ is a set, then $ \bigcup x = \bigcup_{z \in x} z $ is a set as well.
	\item[\textbf{Pairing}] If $ x $ and $ y $ are sets, then $ \{x,y\} $ is a set as well.
	\item[\textbf{Powerset}] If $ x $ is a set, then the powerset $ P(x) $ is a set as well.
	\item[\textbf{Limitation of size}] A class $X$ is a proper class if and only if
		there is a bijection between $X$ and $V$.
	\item[\textbf{Class comprehension}] For every first-order formula $ \phi(x) $ with a free variable $ x $ in which the quantifiers are over sets,
		there exists a class $ \{ x \in V : \phi(x) \} $ whose elements are exactly those sets $ x $ such that $ \phi(x) $.
\end{description}

In matters of set theory,
we will generally follow the notations and terminological conventions
of the comprehensive monograph of \cite{Jech2003}.
We will also refer to the texts of \cite{Drake1974} and \cite{Kanamori2009}.
In particular, $ \Ord $ denotes the proper class of ordinal numbers.
For any ordinal $ \alpha $, the set $ V_{\alpha} $ is defined recursively as follows:
\begin{enumerate}
	\item If $ \alpha = 0 $, then $ V_{\alpha} \coloneq \emptyset $;
	\item if $ \alpha = \beta + 1 $ for an ordinal number $ \beta $, then $ V_{\alpha} \coloneq P(V_{\beta}) $;
	\item if $ \alpha $ is a limit ordinal, then $ V_{\alpha} \coloneq \bigcup_{\beta < \alpha} V_{\beta} $.
\end{enumerate}
If $ x $ is a set, then the \defn{rank} of $ x $ is the smallest ordinal number $ \alpha $ such that $ x \in V_{\alpha} $.
The proper class $ V $ is then the union $ \bigcup_{\alpha \in \Ord} V_{\alpha} $.

In matters of (higher) category theory,
we will generally follow the notations and terminological conventions
of \cite{Lurie2009}.
However, we will simplify some pieces of language and notation,
and our set-theoretic conventions are slightly different:
\begin{definition}
	A \defn{large-category}%
	\footnote{We use the term \enquote{category}
		for what other authors might call
		\enquote{$ \infty $-category},
		\enquote{$(\infty,1)$-category}, or
		\enquote{quasicategory}.
		We will use the term \enquote{1-category}
	when the specification is needed.}
	$C$ consists of a sequence $(C_n)_{n\in\NN_0}$ of classes,
	along with a family of class maps $ \phi^{\ast} \colon C_m \to C_n $
	for each map $ \phi \colon n \to m $,
	subject to all the formulas that express the statement that $C_n$ is a simplicial class
	that satisfies the inner Kan condition.

	If the large-category $C$ contains a full subcategory $ C' \subseteq C $
	such that each $ C'_n $ is a set and
	such that every object of $ C $ is equivalent to an object of $ C' $,
	then we will call $ C $ a \defn{category}
	or, for emphasis, a \defn{small category}.%
	\footnote{These are sometimes called \defn{essentially small}.}
	
	A large-category $ C $ is said to be \defn{locally small} if and only if,
	for every subset $ C'_0 \subseteq C_0$,
	the full subcategory $ C' \subseteq C $ spanned by the elements of $ C'_0 $ is small.
	
	Limits and colimits are only considered for functors $ J \to C $ in which
	the $ J $ is a small category.
	Hence when we refer to \defn{all limits} or \defn{all colimits},
	we mean all limits or colimits of diagrams indexed on small categories.
	
	If $ C $ is a category and $ D $ is a large-category, 
	then the large-category $ \Fun(C, D) $ is defined in the usual way,
	so that $ \Fun(C, D)_n $ is the class of simplicial maps
	$ C \times \Delta^n \to D $.
\end{definition}

\begin{notation}%
	\label{not:Vsetsanimaepresheavescategories}
	We shall write $ \Set^V $ for the large-category of all sets.
	The objects of $ \Set^V$ are thus precisely the elements of $V$.

	We shall write $ \An^V $ for the large-category of all animae.%
	\footnote{We follow Clausen and Scholze,
		and we use the term \enquote{anima} for
		what other authors might call
		\enquote{space},
		\enquote{$ \infty $-groupoid}, or
		\enquote{$ (\infty,0) $-category}.}	
	The objects of $ \An^V $ are thus precisely
	the Kan complexes $ \Delta^{\op} \to \Set^V $.

	If $ C $ is a category, then
	we shall write $ \PP^V(C) \coloneq \Fun(C^{\op}, \An^V) $
	and $ \tau_0\PP^V(C) \coloneq \Fun(C^{\op}, \Set^V) $.
	We will write
	\[ \yo \colon C \inclusion \PP^V(C) \]
	for the Yoneda embedding $ X \mapsto \yo_X $.
	Thus if $ X $ is an object of $ C $, then
	$ \yo_X \colon C^{\op} \to \An^V $ is
	the functor represented by $ X $, so that
	$ \yo_X(U) = \Map(U, X) $.

	We shall write $ \Cat^V $ for the large-category of all categories.
	The objects of $ \Cat^V $ are thus precisely
	the weak Kan complexes $ \Delta^{\op} \to \Set^V $.
\end{notation}

The Class Comprehension Axiom Schema implies the \defn{Axiom of Global Choice},
which ensures the existence of a choice function $ \tau \colon V \to V $ such that $ \tau(x) \in x $.
One needs this to make sense of a construction
like \enquote{the} functor $ - \times u \colon C \to C $ for a large-category $ C $
with all finite products and a fixed object $ u \in C $.

Here, we will work with $\nbg$ as our base theory,
so that we may speak of proper classes and large-categories
whenever the occasion arises.

However, the whole project of higher category theory turns on the principle that
we want to be able to deal with the collections of all objects of a given kind
as a mathematical object in its own right,
and that the passage up and down these category levels
is a fruitful way to understand even completely \enquote{decategorified} objects.%
\footnote{The \emph{microcosm principle} of Baez--Dolan is a
precise illustration of this principle.} 
If we have a large-category $C$,
$ \nbg $ provides us with no mechanism
to view $C$ itself as an object
of a still larger category of all categories.
This limits the kinds of operations we are permitted to do to $ C $.

To give ourselves the room to pass up and down category levels,
we need to have a hierarchy of \enquote{scales} at which we can work.
These scales will be identified by inaccessible cardinals
or, equivalently, Grothendieck universes.
The existence of these cardinals is independent of $ \nbg $.

But now a different sort of concern arises.
We will also want to have assurance
that the results we obtain at one scale remain valid at other scales.
We might prefer to prove sentences about
\emph{all} sets, animae, groups, etc. --
not just those within a universe.
This sort of \enquote{scale-invariance of truth}
is expressed by a \defn{reflection principle} (\ref{sec:reflection_principles}).
The reflection principle we will use here states roughly that
there is an inaccessible cardinal $ \kappa $ such that
statements of set theory hold in the universe $ V_{\kappa} $
just in case they hold in $V$ itself.
This is the \defn{Lévy scheme} $ \levy $,
which we will first formulate as a large cardinal axiom.
In effect, this permits us to focus our attention on \emph{categories}
(as opposed to large-categories).
This scheme has a strictly higher conistency strength
than the existence of a proper class of inaccessible cardinals;
however, its consistency strength is strictly lower
than that of the existence of a single Mahlo cardinal.

Following beautiful work of \cite{Hamkins2003},
we will later prove that the Lévy scheme is equivalent to
(not only equiconsistent with)
a \defn{maximality principle},
which states roughly that every sentence of set theory
that is true for a sieve of forcing extensions is true.

%-------------------------------------------------------------------%
\subsection{Regularity and smallness}%
\label{sub:regularity_and_smallness}
%-------------------------------------------------------------------%

\begin{definition}
	If $ \kappa $ is a cardinal,
	then a set $ S $ is \defn{$ \kappa $-small}
	if and only if $ |S| < \kappa $.
	We shall write $ \Set^{\kappa} \subset \Set^V$
	for the category of $ \kappa $-small sets,
	Thus $ \Set^V $ is the filtered union of the categories $ \Set^{\kappa} $
	over the proper class of regular cardinals.
	
	A cardinal $ \kappa $ is \defn{regular}
	if and only if, for every map $ f \colon S \to T $
	in which $ T $ and every fiber $ f^{-1}\{t\} $ are all $ \kappa $-small,
	the set $ S $ is $ \kappa $-small as well.
	Equivalently, $ \kappa $ is regular if and only if
	$ \Set^{\kappa} \subset \Set^V $ is stable under colimits indexed by $ \kappa $-small posets.
\end{definition}

\begin{eg}
	Under this definition, $ 0 $ is a regular cardinal.%
	\footnote{Many texts require that a regular cardinal be infinite.}
	There are no $ 0 $-small sets.
\end{eg}

\begin{eg}
	The $ \aleph $ family of cardinals is defined by
	a function from the class of ordinal numbers
	to the class of cardinal numbers, by transfinite induction:
	\begin{enumerate}
		\item The cardinal $ \aleph_0 $ is the ordinal number
			$ \omega $ consisting of all finite ordinals.
		\item For any ordinal $ \alpha $,
			one defines $ \aleph_{\alpha + 1} $ to be
			the smallest cardinal number strictly greater than $ \aleph_{\alpha} $.
		\item For any limit ordinal $ \alpha $,
			one defines $ \aleph_{\alpha} \coloneq \sup \left\{ \aleph_{\beta} : \beta < \alpha \right\} $.
	\end{enumerate}

	The countable cardinal $ \aleph_0 $ is regular.
	Every infinite successor cardinal is regular;
	consequently, $ \aleph_n $ for $ n \in \NN $ is regular as well.
	The cardinal $ \aleph_{\omega} $ is
	the smallest infinite cardinal that is not regular.
\end{eg}

\begin{eg}
	A set is $ \aleph_0 $-small if and only if it is finite.
\end{eg}

\begin{nul}
	If $ \kappa $ is a regular cardinal,
	then $ \Set^{\kappa} \subset \Set^V $ is the full subcategory
	generated by the singleton $ \{ 0 \}$
	under colimits over $ \kappa $-small posets.
\end{nul}

\begin{definition}
	If $ \kappa $ is a regular cardinal,
	then we shall write $ \An^{\kappa} \subset \An^V $
	for the full subcategory generated by $ \{ 0 \} $
	under colimits over $ \kappa $-small posets.
	The objects of $ \An^{\kappa} $ will be called
	\defn{$ \kappa $-small animae}.

	Similarly, we shall write $ \Cat^{\kappa} \subset \Cat^V $
	for the full subcategory generated 
	by $ \{ 0 \} $ and $ \{ 0 < 1 \} $
	under colimits over $ \kappa $-small posets.
	The objects of $ \Cat^{\kappa} $ will be called
	\defn{$ \kappa $-small categories}.

	Finally, a large-category $ C $
	is said to be \defn{locally $ \lambda $-small}
	if and only if,
	for every $ \lambda $-small subset $ C'_0 \subseteq C_0 $
	of objects of $ C $, the full subcategory $ C' \subseteq C $
	that it spans is $ \lambda $-small.
\end{definition}

\begin{eg}
	This turn of phrase above is slightly ambiguous when $ \kappa = 0 $.
	In that case, we take the phrase
	\enquote{subcategory generated by \ldots\ under colimits
	over the empty collection of posets}
	to mean the empty category.
	With this convention, there are no $ 0 $-small animae or categories:
	\[ \Set^0 = \An^0 = \Cat^0 = \varnothing \period \]
\end{eg}

\begin{eg}
	An anima is $ \aleph_0 $-small if and only if
	it is weak homotopy equivalent to a simplicial set
	with only finitely many nondegenerate simplices.

	A category $ C $ is $ \aleph_0 $-small if and only if
	it is Joyal equivalent to a simplicial set
	with only finitely many nondegenerate simplices. 
\end{eg}

\begin{eg}
	Why does regularity arise so often in category theory?
	What role does this hypothesis play?
	Here is the sort of scenario that is often
	lurking in the background when we appeal to the
	regularity of a cardinal.

	Let $ C $ be a large-category.
	Suppose that we have a \emph{diagram of diagrams} in $ C $,
	in the following sense.
	We have a category $ A $;
	a functor $ B \colon A \to \Cat^V $;
	and for each $ \alpha \in A $,
	a functor $ X_{\alpha} \colon B_{\alpha} \to C $.
	Furthermore, the colimits of each of these functors
	organize themselves into a functor
	$ A \to C $:
	\[
		\alpha \mapsto \colim_{\beta \in B_{\alpha}}
		X_{\alpha}(\beta) \period
	\]
	We will often be in situations in which
	we need to analyze the \emph{colimit of colimits}:
	\[
		\colim_{\alpha \in A}
		\colim_{\beta \in B_{\alpha}} X_{\alpha}(\beta) \period
	\]
	In this case, we may reorganize these data.
	We first construct the cocartesian fibration
	corresponding to the functor $ B $,
	which we will abusively write $ B \to A $,
	since the fibers are the categories $ B_{\alpha} $.
	We now have a single functor $ X \colon B \to C $
	whose restriction to any fiber $ B_{\alpha} $
	is the functor $ X_{\alpha} $.
	Now the colimit of colimits above is a single colimit:
	\[
		\colim_{\alpha \in A}
		\colim_{\beta \in B_{\alpha}} X_{\alpha}(\beta) \simeq
		\colim_{\gamma \in B} X(\gamma) \period
	\]

	Now let $ \kappa $ be a cardinal.
	If $ A $ is $ \kappa $-small,
	and if each category $ B_{\alpha} $ is $ \kappa $-small,
	then what can we conclude about $ B $?
	In general, nothing.
	However, if $ \kappa $ is a regular cardinal,
	then $ B $ is also $ \kappa $-small.

	The motto here, then, is that
	\emph{%
		if $ \kappa $ is regular,
		then $ \kappa $-small colimits
		of $ \kappa $-small colimits
		are $ \kappa $-small colimits.
	}
\end{eg}

%-------------------------------------------------------------------%
\subsection{Accessibility and presentablity}%
\label{sub:accessibility_and_presentablity}
%-------------------------------------------------------------------%

\begin{definition}
	Let $ \kappa $ be a regular cardinal.
	A category $ \Lambda $ is \defn{$ \kappa $-filtered} if and only if
	it satisfies the following equivalent conditions:
	\begin{enumerate}
		\item For every $ \kappa $-small category $ J $,
			every functor $ f \colon J \to \Lambda $ can be extended
			to a functor $ F \colon J^{\rhd} \to \Lambda $.
		\item For every $ \kappa $-small category $ J $
			and every functor $ H \colon \Lambda \times J \to \An^V$,
			the natural morphism
			\[
				\colim_{\lambda \in \Lambda} \lim_{j \in J} H(\lambda,j)
				\to
				\lim_{j \in J} \colim_{\lambda \in \Lambda} H(\lambda,j)
			\]
			is an equivalence.
		\item For every $ \kappa $-small category $ J $,
			the diagonal functor
			$ \Lambda \to \Fun(J, \Lambda) $
			is cofinal.
	\end{enumerate}
\end{definition}

\begin{eg}
	For any regular cardinal $ \kappa $,
	the ordinal $ \kappa $, regarded as a category,
	is $ \kappa $-filtered.

	More generally, a poset is $ \kappa $-filtered if and only if
	every $ \kappa $-small subset thereof is dominated by some element.
\end{eg}

\begin{eg}
	A $ \kappa $-small category is $ \kappa $-filtered
	if and only if it contains a terminal object.
\end{eg}

\begin{eg}
	Since no category is $ 0 $-small,
	every category is $ 0 $-filtered.
\end{eg}

\begin{definition}
	Let $ \kappa $ be a regular cardinal.
	A functor $ f \colon C \to D $ between large-categories
	will be said to be \defn{$ \kappa $-continuous}
	if and only if it preserves $ \kappa $-filtered colimits.
	
	An object $ X $ of a locally small large-category $ C $
	is said to be \defn{$ \kappa $-compact} if and only if
	the functor $ \yo^X \colon C \to \An^V$ corepresented by $ X $
	(i.e., the functor $ Y \mapsto \Map_C(X,Y) $)
	is $ \kappa $-continuous.
	We write $ C^{(\kappa)} \subseteq C $
	for the full subcategory of $ \kappa $-compact objects.

	A large-category $C$ is \defn{$\kappa$-accessible} if and only if
	it satisfies the following conditions:
	\begin{enumerate}
		\item The category $ C $
			is locally small.
		\item The category $ C $
			has all $ \kappa $-filtered colimits.
		\item The subcategory $ C^{(\kappa)} \subseteq C $ is small. 
		\item The subcategory $ C ^{(\kappa)} \subseteq C $ generates $ C $
			under $ \kappa $-filtered colimits.
	\end{enumerate}
	
	A $ \kappa $-accessible large-category $C$ is
	\defn{$\kappa$-presentable}%
	\footnote{Some authors use the phrase
		\defn{$ \kappa $-compactly generated} instead.}
	if and only if $ C^{(\kappa)} $
	has all $ \kappa $-small colimits.
\end{definition}

\begin{eg}
	A $ 0 $-continuous functor is one that preserves all colimits.
	Hence a $ 0 $-compact object $ X $ is one in which the natural map 
	\[
		\Map(X, \colim_{\alpha \in A} Y_{\alpha}) \equivalence
		\colim_{\alpha \in A} \Map(X, Y_{\alpha})
	\]
	is an equivalence,
	irrespective of the category $ A $ or the diagram $ Y \colon A \to C $.

	The following are equivalent for a large-category $ C $.
	\begin{enumerate}
		\item There exists a small full subcategory $ D \subseteq C $ whose
			inclusion extends along the Yoneda embedding
			to an equivalence of categories
			\eqref{not:Vsetsanimaepresheavescategories}
			\[ \PP^V(D) \equivalence C \period \]
		\item The full subcategory $ C^{(0)} \subseteq C $ of $ 0 $-compact objects
			is small, and
			its inclusion extends along the Yoneda embedding
			to an equivalence of categories
			\[ \PP^V(C^{(0)}) \equivalence C \period \]
		\item The large-category $ C $ is $ 0 $-accessible.
		\item The large-category $ C $ is $ 0 $-presentable.
	\end{enumerate}
\end{eg}

\begin{eg}
	Let $ \kappa $ be an uncountable regular cardinal.
	Then the following are equivalent for a category $ C $.
	\begin{enumerate}
		\item The category $ C $ is $ \kappa $-small.
		\item The set of equivalence classes
			of objects of $C$ is $ \kappa $-small,
			and for every morphism
			$ f \colon X \to Y $ of $ C $
			and every $ n \in \NN_0$,
			the set $ \pi_n(\Map_C(X,Y),f) $ is
			$ \kappa $-small.
		\item The category $ C $ is $ \kappa $-compact
			as an object of $ \Cat^V $;
			that is, $ \Cat^{\kappa} = \Cat^{V,(\kappa)} $.
	\end{enumerate}
	In particular, an anima $ X $ is $ \kappa $-small if and only if
	all its homotopy sets are $ \kappa $-small, if and only if
	it is $ \kappa $-compact as an object of $ \An^V $.
\end{eg}

\begin{eg}
	The equivalence above is doubly false if $ \kappa = \aleph_0 $.

	First, we certainly have a containment
	\[ \Cat^{\aleph_0} \subset \Cat^{V,(\aleph_0)} \comma \]
	but this containment is proper.
	An $ \aleph_0 $-compact anima is a \emph{retract} of
	an $ \aleph_0 $-small anima, but 
	it may not be $ \aleph_0 $-small itself.
	If $ X $ is $ \aleph_0 $-compact and \emph{simply connected},
	then $ X $ is $ \aleph_0 $-small, but
	for non-simply-connected animae,
	we have the \emph{de Lyra--Wall finiteness obstruction},
	which lies in the reduced $ K_0 $ of the group ring $ \ZZ[\pi_1(X)] $.

	Second, the homotopy sets of an $ \aleph_0 $-small $ X $ anima
	are not generally finite.
	By a theorem of Serre,
	if each connected component $ Y \subseteq X $
	has finite fundamental group, then
	its homotopy groups are finitely generated.
	But if $ \pi_1(X) $ isn't finite,
	this too fails;
	for example, $ \pi_3(S^1 \vee S^2) $ is not finitely generated.

	It is still true that the category $ \Cat^V $ is
	$ \aleph_0 $-presentable.
\end{eg}

\begin{nul}
	Let $ \kappa \leq \lambda $ be regular cardinals.
	A $ \kappa $-small category is $ \lambda $-small.
	A $ \lambda $-filtered category is $ \kappa $-filtered.
	A $ \kappa $-continuous functor is $ \lambda $-continuous.
	In general, however, there are $ \kappa $-accessible categories
	that are not $ \lambda $-accessible.
\end{nul}

\begin{definition}
	Let $ \kappa $ and $ \lambda $ be regular cardinals.
	We write $ \kappa \ll \lambda $
	if and only if,
	for every pair of cardinals
	$ \kappa_0 < \kappa $ and $ \lambda_0 < \lambda $,
	one has $ \lambda_0^{\kappa_0} < \lambda $.
	Equivalently, $ \kappa \ll \lambda $ if and only if,
	for every $ \kappa $-small set $A$
	and every $ \lambda $-small set $ B $,
	the set $ \Map(A, B) $ is $ \lambda $-small.
\end{definition}

\begin{eg}
	For every regular cardinal $ \kappa $,
	one has $ 0 \ll \kappa $.
\end{eg}

\begin{eg}
	For every infinite regular cardinal $ \kappa $,
	one has $ \aleph_0 \ll \kappa $.
\end{eg}

\begin{nul}
	Let $ \kappa $ and $ \lambda $ be regular cardinals.
	How is the condition $ \kappa \ll \lambda $ used in practice?
	The answer comes down to the following pair of manoeuvres,
	which we can do whenever $ \kappa \ll \lambda $.

	If $ J $ is a $ \lambda $-small poset,
	then we can write
	\[ J = \bigcup_{\el \in \Lambda} J_{\el} \comma \]
	where $ \Lambda $ is a $ \lambda $-small and $ \kappa $-filtered poset,
	and each $ J_{\el} \subseteq J $ is a $ \kappa $-small poset.
	In this way, we may express
	any $ \lambda $-small colimit as
	a $ \lambda $-small and $ \kappa $-filtered colimit of
	$ \kappa $-small colimits:
	\[ \colim_{j \in J} X(j) \simeq \colim_{\el \in \Lambda} \colim_{j \in J_{\el}} X(j) \]
	\citep[Corollary 4.2.3.11]{Lurie2009}.

	On the other hand, if $ M $ is a $ \kappa $-filtered poset,
	then we can write
	\[ M = \bigcup_{k \in K} M_k \comma \]
	where $ K $ is a $ \lambda $-filtered poset,
	and each $ M_k \subseteq M $ is $ \lambda $-small and $ \kappa $-filtered.
	In this way, we may express
	any $ \kappa $-filtered colimit as
	a $ \lambda $-filtered colimit of
	$ \lambda $-small and $ \kappa $-filtered colimits:
	\[ \colim_{m \in M} Y(m) \simeq \colim_{k \in K} \colim_{m \in M_k} Y(m) \]
	\citep[Lemma 5.4.2.10]{Lurie2009}.
\end{nul}

\begin{proposition}[\protect{\citealp[Proposition 5.4.2.11]{Lurie2009}}]
	If $ \kappa \ll \lambda $ are regular cardinals,
	then every $ \kappa $-accessible category
	is $ \lambda $-accessible.
	Similarly, every $ \kappa $-presentable category
	is $ \lambda $-presentable.
\end{proposition}

\begin{definition}
	A large-category $ C $ is \defn{accessible} if and only if
	there exists a regular cardinal $ \kappa $ such that
	$ C $ is $ \kappa $-accessible.

	We shall say that $ C $ is \defn{presentable} if and only if
	there exists a regular cardinal $ \kappa $ such that
	$ C $ is $ \kappa $-presentable.
\end{definition}

\begin{eg}
	A small category is accessible if and only if it is idempotent-complete
	\citep[Corollary 5.4.3.6]{Lurie2009}.
\end{eg}

\begin{nul}
	A large-category is presentable if and only if
	it is accessible and has all colimits.
	A presentable large-category automatically has all limits as well.
\end{nul}

\begin{definition}
	A large-category $ C $ is \defn{locally presentable}%
	\footnote{In the $1$-category literature,
		the phrase \emph{locally presentable category} is used for
		what we call \emph{presentable category}.}
	if and only if, every object $ X \in C $
	is contained in a presentable full subcategory $ C' \subseteq C $
	such that the inclusion $ C' \inclusion C $ preserves colimits.
\end{definition}

%-------------------------------------------------------------------%
\subsection{Presheaf categories}%
\label{sub:presheaf_categories}
%-------------------------------------------------------------------%

Let $ C $ be a large-category.
What happens if we seek to make sense in $ \nbg $
of the category $ \tau_0\PP^V(C) $ of presheaves of sets
$ C^{\op} \to \Set^V $?

Right away we encounter a problem:
if the objects of $ C $ form a proper class $ C_ 0 $,
then there is no class of class maps $ \Map(C_0, V) $.
Indeed, on one hand, in $ \nbg $, every element of a class is itself a set,
and on the other hand, a class map $ f \colon C_0 \to V $ cannot be a set.%
\footnote{Worse still, the \enquote{very large}
	category of classes is not cartesian closed,
	so there's no hope of defining
	$ \Map(C_0, V) $ by means of some other artifice.}

\begin{nul}
	If $ C $ is a small category, then 
	the large-category $ \tau_0\PP^V(C) $ is locally small,
	and it enjoys many of the same good properties
	enjoyed by $ \Set^V $ itself.
	For every regular cardinal $ \kappa $,
	it is $ \kappa $-presentable,
	and it is \defn{cartesian closed}:
	for every pair of presheaves $ X, Y \colon C^{\op} \to \Set^V $,
	the morphisms $ X \to Y $
	form a presheaf $ \MOR(X, Y) \colon D^{\op} \to \Set^V $.
	The category $ \tau_0\PP^V(C) $ is a \defn{1-topos}.

	Similarly, the category $ \PP^V(C) $ of presheaves
	$ C^{\op} \to \An^V $
	is a $ \kappa $-presentable topos
	for every regular cardinal $ \kappa $.
\end{nul}

\begin{eg}
	Let $ C $ be a locally small category.
	If $ Y \in C $ is an object,
	then $ \yo_Y \colon C^{\op} \to \Set^V $ is
	the presheaf $ X \mapsto \Map_C(X,Y) $ represented by $ Y $.

	Dually, if $ X \in C $ is an objects,
	then $ \yo^X \colon C \to \Set^V $ is the functor
	$ Y \mapsto \Map_C(X,Y) $ corepresented by $ X $.
\end{eg}

\begin{definition}
	Let $ C $ be a locally small large-category.
	A \defn{small presheaf} of sets on $ C $ is
	a functor $ C^{\op} \to \Set^V $
	that is left Kan extended from its restriction
	to some small full subcategory $ D \subseteq C $.
	We write $ \tau_0\PP^V(C) $
	for the locally small large-category
	of small presheaves of sets.
	
	Similarly, a \defn{small presheaf} (of animae) is
	a functor $ C^{\op} \to \An^V $
	that is left Kan extended from its restriction to
	some small full subcategory $ D \subseteq C $.
	We write $ \PP^V(C) $
	for the locally small large-category
	of small presheaves.
\end{definition}

\begin{eg}
	Of course if $ C $ is a small category, then
	every presheaf on $ C $ is small.
	Thus the notation above does not conflict with the one established in
	Notation \ref{not:Vsetsanimaepresheavescategories}.
\end{eg}

\begin{nul}
	For any small full subcategory $ D \subseteq C $,
	we may contemplate the large-category $ \PP^V(D) $
	of presheaves $ D^{\op} \to \An^V $.
	If we have an inclusion of full subcategories
	$ D' \subseteq D \subset C $, then
	left Kan extension identifies $ \PP^V(D') $ with
	a full subcategory of $ \PP^V(D) $.
	
	The (class-indexed) filtered union
	$ \bigcup_D \tau_0\PP^V(D) $
	over the class of small full subcategories of $ C $
	is precisely the large-category $ \PP^V(C) $.
\end{nul}

\begin{eg}
	Assume that $ C $ is locally small.
	For any object $ Y \in C $,
	the representable presheaf $ \yo_Y $ is left Kan extended
	from any full subcategory that contains $ Y $.
	In particular, $ \yo_Y $ is small.

	Thus the assignment $ Y \mapsto \yo_Y $ is
	the fully faithful \defn{Yoneda embedding}
	\[ \yo \colon C \inclusion \PP^V(C) \period \]
\end{eg}

\begin{eg}
	If $ C^{\op} $ is accessible,
	then $ \tau_0\PP^V(C) $ and $ \PP^V(C) $
	are the categories of accessible functors
	$ C^{\op} \to \Set^V $ and $ C^{\op} \to \An^V $, respectively.
\end{eg}

\begin{nul}
	The categories $ \tau_0\PP^V(C) $ and $ \PP^V(C) $
	may not enjoy all the same good features that
	$ \Set^V $ and $ \An^V$ have.
	The categories $ \tau_0\PP^V(C) $ and $ \PP^V(C) $
	possess all colimits,
	but they do not generally have all limits.
	For example, if $C$ has no nonidentity arrows, then
	there is no terminal object in $ \tau_0\PP^V(C) $.
	
	If $ C^{\op} $ is accessible or small,
	then $ \tau_0\PP^V(C) $ and $ \PP^V(C) $ do have all limits.
\end{nul}

\begin{definition}
	Let $ A $ be a class of categories.
	Let $ C $ be a locally small large-category,
	and let $ C^0 \subseteq C $ be a full subcategory.
	Then we say that
	\defn{$ C^0 $ generates $ C $ freely under $ A $-shaped colimits}
	if and only if,
	for every large-category $ D $
	that has all $ A $-shaped colimits,
	the following assertions obtain.
	\begin{enumerate}
		\item Every functor $ C^0 \to D $ extends
			to a functor $ C \to D $
			that preserves $ A $ -shaped colimits.
		\item For every pair of functors
			$ F, G \colon C \to D $
			that preserve $ A $-shaped colimits,
			the map
			$ \Map(F, G) \to \Map(F|C^0, G|C^0 ) $
			is an equivalence.
	\end{enumerate}
	If $ f \colon C' \inclusion C $ is a fully faithful functor,
	then we will say that
	\defn{$ f $ generates $ C $ freely under $ A $-shaped colimits}
	if and only if its image $ f(C') \subseteq C $ does so.
\end{definition}

\begin{remark}
	If $ C $ is not small, then in $ \nbg $
	we can make sense neither of $ \Fun(C,D) $,
	nor of the full subcategory
	$ \Fun^{A}(C, D) \subseteq \Fun(C,D) $
	consisting of those functors that preserve
	$ A $-shaped colimits.
	If however we are in a situation in which
	these objects \emph{can} be made sensible,
	then $ C^0 $ generates $ C $ freely under
	$ A $-shaped colimits
	if and only if the restriction induces an equivalence
	\[ \Fun^A(C, D) \equivalence \Fun(C^0, D) \period \]
\end{remark}

\begin{proposition}%
	\label{prp:PPsmisfreelygenerated}
	Let $ C $ be a locally small large-category.
	Then the Yoneda embedding
	$ \yo \colon C \inclusion \PP^V(C) $
	generates $ \PP^V(C) $ freely under all colimits.
\end{proposition}

The theory of small presheaves can be relativized
to a regular cardinal $ \kappa $:

\begin{definition}
	Let $ \kappa $ be a regular cardinal.
	Let $ C $ be a locally $ \kappa $-small large-category.
	A \defn{$ \kappa $-small presheaf} of sets on $ C $
	is a functor $ C^{\op} \to \Set^{\kappa} $
	that is left Kan extended from its restriction to
	some $ \kappa $-small full subcategory $ D \subseteq C $.
	The large-category of $ \kappa $-small presheaves of sets
	will be denoted $ \tau_0\PP^{\kappa}(C) $.

	Similarly, a \defn{$ \kappa $-small presheaf}
	(of animae) is a functor $ C^{\op} \to \An^{\kappa} $
	that is left Kan extended from its restriction to 
	some $ \kappa $-small full subcategory $ D \subseteq C $.
	The large-category of $ \kappa $-small presheaves
	will be denoted $ \PP^{\kappa}(C) $.
\end{definition}

\begin{nul}
	If $ C $ is small, then so is $ \PP^{\kappa}(C) $.

	Since we have assumed that $ C $ is locally $ \kappa $-small,
	it follows that the Yoneda embedding lands in $ \PP^{\kappa}(C) $.
\end{nul}

\begin{proposition}%
	\label{prp:PPkappaisfreelygenerated}
	Let $ \kappa $ be a regular cardinal.
	Let $ C $ be a locally $\kappa$-small large-category.
	Then the Yoneda embedding
	$ \yo \colon C \inclusion \PP^{\kappa}(C) $
	generates $ \PP^{\kappa}(C) $ freely under
	$ \kappa $-small colimits.
\end{proposition}

\begin{nul}
	The category $ \PP^{\kappa}_0(C) $
	has all $ \kappa $-small colimits, but
	in general, it does not have $ \kappa $-small limits, and
	it is not cartesian closed.
	To ensure these properties as well,
	we must turn to a discussion of inaccessible cardinals.
\end{nul}

%-------------------------------------------------------------------%
\subsection{Strong limit and inaccessible cardinals}%
\label{sub:strong_limit_and_inaccessible_cardinals}
%-------------------------------------------------------------------%

\begin{definition}
	One says that $ \kappa $ is a \defn{weak limit cardinal} if and only if,
	for every cardinal $ \xi $,
	if $ \xi < \kappa $, then $ \xi^+ < \kappa $.

	A cardinal $\kappa$ is said to be a \defn{strong limit cardinal} if and only if,
	for every cardinal $ \xi $,
	if $ \xi < \kappa $, then $ 2^{\xi} < \kappa $ as well.
	Equivalently, $ \kappa $ is a strong limit cardinal if and only if,
	for every pair of $ \kappa $-small sets $ X $ and $ Y $,
	the set $ \Map(X, Y) $ of maps $ X \to Y $ is $ \kappa $-small as well. 

	One says that $\kappa$ is \defn{weakly inaccessible} if and only if
	it is a regular, uncountable, weak limit cardinal.

	One says that $\kappa$ is \defn{inaccessible}%
	\footnote{Some authors say \defn{strongly inaccessible}
	instead of \defn{inaccessible}.}
	if and only if it is a regular, uncountable,%
	\footnote{We include the condition of uncountability only for convenience.
		It is not unreasonable to regard $ 0 $ and $ \aleph_0 $
		as inaccessible as well.}
	strong limit cardinal.
	Equivalently, an uncountable cardinal $ \kappa $ is inaccessible if and only if
	$ \Set^{\kappa} $ has all $ \kappa $-small colimits and is cartesian closed.
	Equivalently again, an uncountable cardinal $ \kappa $ is inaccessible if and only if 
	$ \Set^{\kappa} $ has all $ \kappa $-small colimits and all $ \kappa $-small colimits.
\end{definition}

The Generalized Continuum Hypothesis ($ \gch $) is equivalent to the statement that
the classes of strong and weak limit cardinals coincide,
and similarly the classes of inaccessible and weakly inaccessible cardinal coincide.

\begin{eg}
	A cardinal $ \kappa $ is a weak limit cardinal if and only if,
	for some limit ordinal $ \alpha $, one has $ \kappa = \aleph_{\alpha} $.
\end{eg}

\begin{eg}
	The $\beth$ family of cardinals is defined by a function from the class of ordinal numbers to the class of cardinal numbers.
	It's defined by transfinite induction:
	\begin{enumerate}
		\item By definition, $ \beth_0 = \aleph_0 $.
		\item For any ordinal $ \alpha $, one defines $ \beth_{\alpha+1} \coloneq 2^{\beth_{\alpha}} $.
		\item For any limit ordinal $ \alpha $,
			one defines $ \beth_{\alpha} \coloneq \sup \left\{ \beth_{\beta} : \beta < \alpha \right\} $.
	\end{enumerate}
	The cardinal $ \beth_{\alpha} $ is the cardinality of $ V_{\omega+\alpha} $.

	The Generalized Continuum Hypothesis ($ \gch $) is equivalent to the statement that
	$ \aleph_{\alpha} = \beth_{\alpha} $ for each ordinal $ \alpha $,	
	
	A cardinal $ \kappa $ is a strong limit cardinal if and only if,
	for some limit ordinal $ \alpha $, one has $ \kappa = \beth_{\alpha} $.

	The cardinal $ \beth_{\omega} $ is
	the smallest uncountable strong limit cardinal.
	It is not inaccessible, however,
	because it is not regular.

	An inaccessible cardinal $ \kappa $ is
	a \defn{$\beth$-fixed point}:
	that is, $ \beth_{\kappa} = \kappa $.
\end{eg}

\begin{nul}
	A regular uncountable cardinal $ \kappa $
	is inaccessible if and only if
	one has $ \kappa \ll \kappa $.
\end{nul}

\begin{definition}[\protect{\citeauthor[Exposé I, \S 0 and Appendix]{SGA4-1}}]%
\label{dfn:uni}
	An uncountable set $ U $ is a \defn{Grothendieck universe} if it satisfies the following conditions.
	\begin{enumerate}
		\item The set $ U $ is \defn{transitive}:
			if $ X \in Y \in U $, then $ X \in U $ as well.
		\item If $ X, Y \in U $, then $ \{X,Y\} \in U $ as well.
		\item If $ X \in U $, then the powerset $ P(X) \in U $ as well.
		\item If $ A \in U $ and $ X \colon A \to U $ is a map, then
		\[ \bigcup_{\alpha\in A}X(\alpha) \in U \]
		as well.
	\end{enumerate}
\end{definition}

Grothendieck universes are essentially the same thing as inaccessible cardinals.
This was effectively proved by \cite{Tarski1938}.
See also Bourbaki, \citeauthor[Exposé I, Appendix]{SGA4-1}.
\begin{proposition}
	If $\kappa$ is an inaccessible cardinal, then
	the set $V_{\kappa}$ of all sets of rank less than $\kappa$ is
	a Grothendieck universe of rank and cardinality $\kappa$.

	If $ U $ is a Grothendieck universe, then
	there exists an inaccessible cardinal $ \kappa $ such that $ U = V_{\kappa} $.
\end{proposition}

\begin{theorem}
	If $ \kappa $ is an inaccessible cardinal,
	then $V_{\kappa} \models \zfc $, and
	$ V_{\kappa+1} \models \nbg $.
	
	Assuming that $ \zfc $ (respectively, $ \nbg $) is consistent, then
	the existence of inaccessible cardinals is not provable
	by methods formalizable in $ \zfc $ (resp., $ \nbg $).
\end{theorem}

\begin{axiom}%
\label{axm:AU}
	The \defn{Axiom of Universes} ($ \au $) is the assertion that
	every cardinal is dominated by an inaccessible cardinal,
	or, equivalently, every set is an element of some Grothendieck universe.
	\defn{Tarski--Grothendieck set theory} is the schema $\tg = \nbg+\au$.
\end{axiom}

Under $ \au $, the proper class of inaccessible cardinals
can be well ordered.
It will be helpful for us to have a notation for this.

\begin{definition}
	Assume $ \au $.
	Let us define the $ \daleth $ family of cardinals
	as a function from the class of ordinal numbers
	to the class of cardinal numbers:
	\begin{enumerate}
		\item By definition, $ \daleth_0 = \aleph_0 $.
		\item For any ordinal $ \alpha $,
			one defines $ \daleth_{\alpha+1} $ as
			the smallest inaccessible number
			greater than $ \daleth_{\alpha} $.
		\item For any limit ordinal $ \alpha $,
			one defines
			$ \daleth_{\alpha} \coloneq \sup \left\{ \daleth_{\beta} :
			\beta < \alpha \right\} $.
	\end{enumerate}
	Thus $ \daleth_0 = \aleph_0 $, and
	$ \daleth_{\alpha} $ is the
	\enquote{$ \alpha $-th inaccessible cardinal}.
\end{definition}

%-------------------------------------------------------------------%
\subsection{Echelons of accessibility}%
\label{sub:echelons_of_accessibility}
%-------------------------------------------------------------------%

The notions of smallness,
accessibility, and presentability of categories
can all be relativized to a Grothendieck universe.

\begin{definition}
	The \defn{echelon} of a category $ C $ is
	the smallest ordinal $ \alpha $ such that
	$ C $ is both
	locally $ \daleth_{\alpha} $-small
	and $ \daleth_{\alpha + 1} $-small.
\end{definition}

\begin{eg}
	The category of finite sets is 
	of echelon $ 0 $.
	More generally, for any ordinal number $ \alpha $,
	the category of $ \daleth_{\alpha} $-small sets is
	of echelon $ \alpha $.
\end{eg}

\begin{notation}
	Let $ \alpha $ be an ordinal number.
	We will denote by $ \Cat_{\alpha} $
	the category of categories that are
	$ \daleth_{\alpha} $-small.

	Accordingly, we will denote by
	$ \Set_{\alpha} $ and $ \An_{\alpha} $
	the categories of $ \daleth_{\alpha} $-small
	sets and animae, respectively.

	The categories $ \Cat_{\alpha} $, $ \Set_{\alpha} $, and $ \An_{\alpha} $
	are all of echelon $ \alpha $.
\end{notation}

\begin{definition}
	Let $ \alpha $ be an ordinal number, and
	let $ \kappa < \daleth_{\alpha} $ be a regular cardinal.
	Let $ C $ and $ D $ be categories
	of echelon $ \leq \alpha $.

	A functor $ f \colon C \to D $
	is \emph{$ \kappa $-continuous of echelon  $ \leq \alpha $} if and only if 
	it preserves all $ \daleth_{\alpha} $-small, $ \kappa $-filtered colimits.
	
	An object $ X $ of $ C $
	is said to be \defn{$ \kappa $-compact of echelon $ \leq \alpha $} if and only if
	the functor $ \yo^X \colon C \to \An_{\alpha} $ corepresented by $ X $
	is $ \kappa $-continuous of echelon $ \leq \alpha $.
	We write $ C^{(\kappa)}_{\alpha} \subseteq C $
	for the full subcategory of $ \kappa $-compact objects of echelon $ \leq \alpha $.

	A category $C$ is
	\defn{$ \kappa $-accessible of echelon $ \leq \alpha $} if and only if
	it satisfies the following trio of conditions:
	\begin{enumerate}
		\item The category $ C $ is of echelon $ \leq \alpha $.
		\item The category $ C $ has all
			$ \daleth_{\alpha} $-small, $ \kappa $-filtered colimits.
		\item The subcategory $ C^{(\kappa)}_{\alpha} $
			generates $ C $ under $ \daleth_{\alpha} $-small and 
			$ \kappa $-filtered colimits.
	\end{enumerate}
	
	A category $C$ is
	\defn{$\kappa$-presentable of echelon $ \leq \alpha $}
	if and only if
	it is $ \kappa $ accessible of echelon $ \leq \alpha $, and  
	$ C^{(\kappa)}_{\alpha} $ has all $ \kappa $-small colimits.

	A category $ C $ is
	\defn{accessible of echelon $ \leq \alpha $}
	if and only if
	there exists a regular cardinal $ \kappa < \daleth_{\alpha} $
	such that $ C $ is $ \kappa $-accessible of echelon $ \leq \alpha $.
	It is
	\defn{presentable of echelon $ \leq \alpha $}
	if and only if
	there exists a regular cardinal $ \kappa < \daleth_{\alpha} $
	such that $ C $ is $ \kappa $-presentable of echelon $ \leq \alpha $.
\end{definition}

\begin{eg}
	Let $ \alpha $ be an ordinal number.
	Let $ C $ be a $ \daleth_{\alpha} $-small category.
	Then we shall write
	\[ \PP_{\alpha}(\CC) \coloneq \PP^{\daleth_{\alpha}}(\CC) \comma \]
	the category of presheaves $ C^{\op} \to \An_{\alpha} $.
	The category $ \PP_{\alpha}(C) $ is then
	$ 0 $-presentable of echelon $ \leq \alpha $.

	Conversely, if $ D $ is a $ 0 $-presentable category
	of echelon $ \leq \alpha $, then 
	there exists a $ \daleth_{\alpha} $-small category $ C $
	and an equivalence $ D \simeq \PP_{\alpha}(C) $.
\end{eg}

\begin{notation}
	Let $ \alpha $ be an ordinal number, and
	let $ \kappa < \daleth_{\alpha} $ be a regular cardinal.
	We shall write $ \Acc^{\kappa}_{\alpha} \subset \Cat_{\alpha+1} $ for
	the following subcategory.
	\begin{enumerate}
		\item The objects of $ \Acc^{\kappa}_{\alpha} $ are
			the $ \kappa $-accessible categories of echelon $ \leq \alpha $.
		\item The morphisms $ f \colon C \to D $ of $ \Acc^{\kappa}_{\alpha} $ are
			the $ \kappa $-continuous functors of echelon $ \leq \alpha $
			such that $ f(C^{\kappa}_{\alpha}) \subseteq D^{\kappa}_{\alpha} $.
	\end{enumerate}
	Similarly, we shall write $ \Pr^{\kappa,L}_{\alpha} \subset \Acc^{\kappa}_{\alpha}$
	for the following subcategory.
	\begin{enumerate}
		\item The objects of $ \Pr^{\kappa,L}_{\alpha} $ are
			the $ \kappa $-presentable categories of echelon $ \leq \alpha $.
		\item The morphisms of $ \Pr^{\kappa,L}_{\alpha} $ are
			those functors in $ \Acc^{\kappa}_{\alpha} $
			that preserve all $ \daleth_{\alpha} $-small colimits.
	\end{enumerate}

	Now we may write
	\[
		\Acc_{\alpha} = \bigcup_{\kappa < \daleth_{\alpha}} \Acc^{\kappa}_{\alpha}
		\andeq 
		\Pr^L_{\alpha} = \bigcup_{\kappa < \daleth_{\alpha}} \Pr^{\kappa,L}_{\alpha}
	\]
	for the category of accessible categories of echelon $ \alpha $
	and the category of presentable categories of echelon $ \alpha $, respectively.
\end{notation}

For a fixed regular cardinal $ \kappa $,
the theory of $ \kappa $-accessible categories
does not depend upon the echelon $ \alpha $,
provided of course that $ \kappa < \daleth_{\alpha} $.

\begin{theorem}
	Let $ \alpha $ be an ordinal number, and
	let $ \kappa < \daleth_{\alpha} $ be a regular cardinal.

\end{theorem}

%-------------------------------------------------------------------%
\subsection{Ind and Pro}%
\label{sub:ind_and_pro}
%-------------------------------------------------------------------%

\begin{definition}
	Let $ \kappa \leq \lambda $ be regular cardinals.
	Let $ C $ be a locally $ \lambda $-small category.
	Then $ \Ind_{\kappa}^{\lambda}(C) $ is
	the smallest full subcategory $ D \subseteq \PP^{\lambda}(C) $
	such that $ D $ contains the image
	of the Yoneda embedding
	$ \yo \colon C \inclusion \PP^{\lambda}(C) $,
	and $ D $ is stable under
	$ \lambda $-small, $ \kappa $-filtered colimits.

	Accordingly, if $ C $ is a locally small large-category,
	then $ \Ind_{\kappa}^{\VV} $ is
	the smallest full subcategory $ D \subseteq \PP^V(C) $
	such that $ D $ contains the image
	of the Yoneda embedding
	$ \yo \colon C \inclusion \PP^V(C) $,
	and $ D $ is stable under $ \kappa $-filtered colimits.
\end{definition}

\begin{eg}
	The category $ \Ind_{\kappa}^{\kappa}(C) $ is
	equivalent to $ C $ itself.
\end{eg}

\begin{proposition}
	Let $ \kappa \leq \lambda $ be regular cardinals.
	Let $ C $ be a locally $ \lambda $-small category.
	Then the Yoneda embedding
	$ \yo \colon C \inclusion \Ind_{\kappa}^{\lambda}(C) $,
	generates $ \Ind_{\kappa}^{\lambda}(C) $ freely under
	$ \lambda $-small, $ \kappa $-filtered colimits.

	Similarly, if $ C $ is a locally small large-category,
	then the Yoneda embedding
	$ \yo \colon C \inclusion \Ind_{\kappa}^{V}(C) $,
	generates $ \Ind_{\kappa}^V(C) $ freely under
	$ \kappa $-filtered colimits.
\end{proposition}

\begin{notation}
	Let $ \kappa \leq \lambda $ be regular cardinals.
	Let $ C $ and $ D $  be locally $ \lambda $-small categories
	that contain all $ \lambda $-small, $ \kappa $-filtered colimits.
	Denote by $ \Fun^{\kappa \leq \lambda}(C, D) $
	the full subcategory of $ \Fun(C,D) $ consisting of
	the functors $ C \to D $ that preserve all 
	$ \lambda $-small, $ \kappa $-filtered colimits.

	If $ C^0 $ is a locally $ \lambda $-small category,
	then restriction along the Yoneda embedding
	induces and equivalence of categories
	\[
		\Fun^{\kappa \leq \lambda}(\Ind_{\kappa}^{\lambda}(C^0), D)
		\equivalence \Fun(C^0, D) \period
	\]
\end{notation}

\begin{eg}
	Let $ \kappa \leq \lambda \leq \mu $ be regular cardinals.
	Then for any 
\end{eg}

%-------------------------------------------------------------------%
\subsection{Higher inaccessibility}%
\label{sub:higher_inaccessibility}
%-------------------------------------------------------------------%

\begin{nul}
	We shall endow an ordinal with its order topology.
	This may be described recursively as follows:
	\begin{enumerate}
		\item The ordinal $ 0 $ is the empty topological space.
		\item For any ordinal $ \alpha $ with its order topology,
			the order topology on the ordinal $ \alpha + 1 $
			is the one-point compactification of $ \alpha $.
		\item For any limit ordinal $ \alpha $,
			the order topology is the colimit topology
			$ \colim_{\beta < \alpha} \beta $.
	\end{enumerate}
\end{nul}

We will use terminology 
that treats $ \Ord $ itself as a topological space,
even though it is not small.

\begin{definition}
	If $ W \subseteq \Ord $ is a subclass,
	then a \defn{limit point} of $ A $ is
	an ordinal $ \alpha $ such that $ \alpha = \sup (W \cap \alpha) $.
	The class $ W $ will be said to be \defn{closed} if and only if
	it contains all its limit points.

	An \defn{ordinal function} is a class map $ f \colon \Ord \to \Ord $.
	We say that $ f $ is \defn{continuous} if and only if
	its restriction to any subset is continuous.
	Equivalently, $ f $ is continuous if and only if,
	for every subclass $ W \subseteq \Ord $
	and every limit point $ \alpha $ of $ W $,
	the ordinal $ f(\alpha) $ is a limit point of $ f(W) $.

	We say that $ f $ is \defn{normal} if and only if
	it is continuous and strictly increasing.
\end{definition}

\begin{nul}
	If $ f $ is a normal ordinal function,
	then its image is a closed and unbounded class%
	\footnote{This is often abbreviated \defn{club class} in
	set theory literature.}
	of ordinals.
	Conversely, if $ W \subseteq \Ord $ is a closed and unbounded class,
	then we can define a normal ordinal function $ f $ by
	\[
		f(\alpha) =
		\min \left\{ \gamma \in W :
			(\forall \beta < \alpha)(f(\beta) < \gamma) \right\} \period
	\]
\end{nul}

\begin{definition}
	Let $ f $ be an ordinal function.
	A regular cardinal $ \kappa $ is said to be
	\defn{$ f $-inaccessible} if and only if,
	for every ordinal $ \alpha$, if $ \alpha < \kappa $,
	then $ f(\alpha) < \kappa $ as well.
\end{definition}

\begin{eg}
	If $ f $ is the ordinal function that carries
	an ordinal $ \alpha $ to the cardinal $ 2^{|\alpha|} $,
	then an $ f $-inaccessible cardinal is precisely
	an inaccessible cardinal.
\end{eg}

\begin{construction}
	Let $ f $ be an increasing ordinal function
	such that for every ordinal $ \beta $, one has $ \beta < f(\beta) $.
	For every ordinal $ \xi $,
	the normal ordinal function $ \alpha \mapsto f^{\alpha}(\xi) $
	in uniquely specified by the requirements that
	$ f^0(\xi) = \xi $ and $ f^{\alpha + 1}(\xi) =f(f^{\alpha}(\xi)) $.

	\cite{Jorgensen1970} proves that an $ f $-inaccessible cardinal
	greater than an ordinal $ \xi $
	is precisely a regular cardinal that is a \emph{fixed point}
	for the ordinal function $ \alpha \mapsto f^{\alpha}(\xi) $.
\end{construction}

\begin{eg}
	If $ f $ is the ordinal function $ \beta \mapsto 2^{|\beta|} $,
	then $ f^{\alpha}(\omega) = \beth_{\alpha} $.
	An inaccessible cardinal is thus precisely a regular $ \beth $-fixed point.
	
	If $ f $ is the ordinal function $ \beta \mapsto |\beta|^+ $,
	then $ f^{\alpha}(\omega) = \aleph_{\alpha} $. 
	A weakly inaccessible cardinal is precisely a regular $ \aleph $-fixed  point.
\end{eg}

\begin{eg}
	Assume $ \au $.
	Consider the ordinal function $ f $ that carries
	an ordinal $ \beta $ to the smallest inaccessible cardinal greater than $ \beta $.
	For any ordinal $ \alpha $,
	we have $ \daleth_{\alpha} = f^{\alpha}(\omega) $.

	An $ f $-inaccessible cardinal is precisely a $ \daleth $-fixed point.
	These are called \defn{$ 1 $-inaccessible cardinals}.
	If $ \kappa $ is $ 1 $-inaccessible, then
	$ V_{\kappa} \models (\zfc + \au) $.
	If $ \zfc + \au $ is consistent, then
	the existence of $ 1 $-inaccessible cardinals is not provable
	by methods formalizable in $ \zfc + \au $.
\end{eg}

Iterating this strategy,
one can now proceed to define $ \alpha $-inaccessibility
for every ordinal $ \alpha $.
Iterating the iteration,
one can define notions of
hyperinaccessibility, hyper${}^{\alpha}$inaccessibility, \emph{etc}.
We cut to the chase:

\begin{axiom}
	The \defn{Lévy scheme} ($ \levy $) is the assertion that
	for every ordinal function $ f $ and every ordinal $ \xi $,
	there exists an $ f $-inaccessible cardinal $ \kappa $ such that $ \xi < \kappa $.
\end{axiom}

\begin{theorem}[\cite{Levy1960, Montague1962, Jorgensen1970}]
	The following are equivalent.
	\begin{enumerate}
		\item The Lévy scheme.
		\item Every normal ordinal function
			has a regular cardinal in its image.
		\item Every closed unbounded subclass $ W \subseteq \Ord $
			contains a regular cardinal.
		\item Every normal ordinal function
			has an inaccessible cardinal in its image.
		\item Every closed unbounded subclass $ W \subseteq \Ord $
			contains an inaccessible cardinal.
	\end{enumerate}
\end{theorem}

\begin{nul}
	The Lévy scheme implies the Axiom of Universes,
	and the consistency strength of $ \nbg + \levy $ is strictly greater
	than that of $ \nbg + \au $.

	The consistency strength of the Lévy scheme is
	also strictly greater than
	the existence of $ \alpha $-inaccessible, hyperinaccessible,
	hyper${}^{\alpha}$inaccessible, \emph{etc}., cardinals.
\end{nul}

\begin{definition}
	Let $ \kappa $ be a regular cardinal.
	One says that $ \kappa $ is \defn{Mahlo} if and only if
	every closed unbounded subset $ W \subseteq \kappa $
	contains a regular cardinal.
\end{definition}

\begin{nul}
	Assume that $ \kappa $ is a Mahlo cardinal.
	Then $ \kappa $ is $ f $-inaccessible for every ordinal function $ f $.
	Accordingly, $ \kappa $ is a fixed point of every normal ordinal function.

	Additionally,
	if $ \kappa $ is a Mahlo cardinal, then
	$ V_{\kappa} \models (\zfc + \levy) $, and similarly
	$ V_{\kappa + 1} \models (\nbg + \levy) $.
	The consistency strength of the axiom
	\enquote{a Mahlo cardinal exists}
	is strictly greater than the Lévy scheme.
\end{nul}

\begin{nul}
	The Lévy scheme and its equivalents and slight variants
	have appeared under various names:
	\enquote{Mahlo's principle} \citep{Gloede1973},
	\enquote{Axiom F} \citep{Drake1974},
	\enquote{$ \Ord $ is Mahlo} \citep{Hamkins2003}.
\end{nul}

For our purposes,
one of the main appeals of the Lévy scheme
is the following.

\begin{theorem}
	Assume $ \levy $.
	
\end{theorem}

%-------------------------------------------------------------------%
\subsection{Universe polymorphism}%
\label{sub:universe_polymorphism}
%-------------------------------------------------------------------%




